\section{Consciousness (\textit{Citta})}

Welcome to the third lesson of this Practical Abhidhamma Course. This lesson describes consciousness, the first of the four Ultimate Realities: \textit{citta}, \textit{cetasika}, \textit{rūpa} and \textit{Nibbāna}.\footnote{See Chapter 1 of “A Comprehensive Manual of Abhidhamma” (see Footnote 2 for link).}

\subsection*{Definition of Consciousness}

Let’s start with a definition of consciousness. I downloaded the Wikipedia entry on “consciousness” and it filled 18 pages.\footnote{\url{http://en.wikipedia.org/wiki/Consciousness}} Obviously, “consciousness” is not a simple thing to define. I do, however, like the first sentence from the Wikipedia entry: “Consciousness is the quality or state of awareness, or, of being aware of an external object or something within oneself.”

There are two important things in this definition. First, consciousness and awareness are synonyms. Second, consciousness always takes an object; either an external object or an internal object.\footnote{Object condition (\textit{ārammaṇa-paccaya}); see Chapter 3 of “The Conditionality of Life” (see Footnote 2 for link).} A \textbf{Sound} is an example of an external object. When the mind is aware of itself, it takes an internal object such as \textbf{Attachment}.

The Commentary defines consciousness as having three roles. First, consciousness is that which is aware of an object; there is no Self that is aware, it is consciousness that is aware. Second, consciousness allows the associated Mental Factors to be aware of an object. And third, consciousness is an activity, a process of being aware of an object.

\subsection*{Definition of Mind Moment}

\begin{figure} [H]

\vspace{-1mm}

\begin{quote}
Life, personhood, pleasure, and pain; this is all that's bound together in a single mental event, a moment that quickly takes place.

Even the spirits who endure for 84,000 aeons, even these, do not live the same for any two moments of mind.

What ceases for one who is dead, or for one who's still standing here, are all just the same aggregates; gone, never to connect again.

The states which are vanishing now, and those which will vanish some day, have characteristics no different than those that have vanished before.

\end{quote}

\caption{The concept of a Mind Moment is captured in this extract from a beautiful poem found in the Suttas (Nm 2.4: \url{http://www.accesstoinsight.org/tipitaka/kn/nm/nm.2.04.olen.html}).}

\end{figure}

Consciousness never arises alone; it always arises as part of a Mind Moment. A Mind Moment is consciousness together with its associated Mental Factors. The Abhidhamma describes Mind Moments and Mental Factors in detail and these are the topics of this lesson and the next lesson.

\pagebreak

Within a Mind Moment, consciousness and each of its associated Mental Factors work as an inseparable team, each with their own characteristic. Consciousness has the characteristic of knowing the object. Each Mental Factor has its own individual characteristic such as grasping the object or not floating away from the object.\footnote{“Grasping the object” is the characteristic of \textbf{Attachment}, “not floating away from the object” is the characteristic of \textbf{Mindfulness}.}

As mentioned in the first lesson, according to the Commentaries, what is conventionally called the mind is modelled as a sequence of Mind Moments, like a “stream of consciousness.” Each Mind Moment arises based on conditions, performs its function and then falls away again.

\subsection*{Brief descriptions of each group of Mind Moments}

\begin{figure} [H]

\setlength{\tabcolsep}{0pt}
\renewcommand{\arraystretch}{1.0}
\noindent\begin{tabular}{C{.04\textwidth}C{0.16\textwidth}C{0.16\textwidth}C{0.16\textwidth}C{0.16\textwidth}C{0.16\textwidth}C{0.16\textwidth}}
\toprule
& \tablesubheader{{\small Unwholesome}} & \tablesubheader{{\small Rootless}} & \tablesubheader{{\small Wholesome}} & \tablesubheader{{\small Fine Material Sphere}} & \tablesubheader{{\small Immaterial Sphere}} & \tablesubheader{{\small Supra-\newline mundane}} \\
\midrule
\tablevsubheader{{\small \hspace{-2mm}Create New Kamma}} & Danger\newline Zone\newline \textbf{1}--\textbf{12} & & Faultless\newline Zone\newline \textbf{31}--\textbf{38} & {\small Fine Material Sphere Wholesome} \textbf{55}--\textbf{59} & {\small Immaterial Sphere Wholesome} \textbf{70}--\textbf{73} & {\small Supramundane Path}\newline \textbf{82}--\textbf{85} \\
\midrule
\tablevsubheader{{\small \hspace{-2mm}Result of Kamma}} & & Sensing\newline Zone\newline \textbf{13}--\textbf{27} & {\small Sense Sphere Resultant}\newline \textbf{39}--\textbf{46} & {\small Fine Material Sphere Resultant} \textbf{60}--\textbf{64} & {\small Immaterial Sphere Resultant} \textbf{74}--\textbf{77} & {\small Supramundane Fruit}\newline \textbf{86}--\textbf{89} \\
\midrule
\tablevsubheader{{\small \hspace{-2mm}Unrelated to Kamma}} & & Sensing\newline Zone\newline \textbf{28}--\textbf{29} & {\small Sense Sphere Functional}\newline \textbf{47}--\textbf{54} & {\small Fine Material Sphere Functional} \textbf{65}--\textbf{69} & {\small Immaterial Sphere Functional} \textbf{78}--\textbf{81} & \\
\bottomrule
\end{tabular}
\caption{A simplified overview of Appendix 2. Numbers in bold represent Mind Moments. The Danger Zone, the Faultless Zone, and the Sensing Zone are the most important areas.}
\label{Handout3}
\end{figure}

Appendix 2 lists Mind Moments numbered from \textbf{1}--\textbf{89}. We will refer back to it many times during other lessons as well. Consider it to be a ``map of the mind." With this map, you can explore without getting lost! 

Figure \ref{Handout3} summarizes the content of Appendix 2. In this map of the mind, Mind Moments are grouped together in rows and columns. I will first give a brief description of each group and then do an in-depth analysis of some of the groups.

\pagebreak

The first group is Mind Moments \textbf{1}--\textbf{12}. These create unwholesome\footnote{The Pāḷi term for unwholesome is \textit{akusala}, the opposite of \textit{kusala}; the prefix “\textit{a}” signifies the “active opposite of,” not merely “absence of.” The Commentary defines \textit{kusala} as “healthy,” “faultless” and “producing happy results.” Therefore, \textit{akusala} means “unhealthy,” “faulty,” and “producing unhappy results.”} kamma. I call this the “Danger Zone.”\footnote{Other Abhidhamma texts do not use the term, “Danger Zone.”} The Danger Zone is engulfed in a dense fog of \textbf{Delusion}, so it is impossible for the mind to see where it is, where it is going, or that this is the “Danger Zone.” In part of the Danger Zone sticky quicksand traps the mind, these are the \textbf{Attachment}-rooted Mind Moments. Another part of the Danger Zone is burning hot and painful for the mind; these are the \textbf{Aversion}-rooted Mind Moments.

The second group is Mind Moments \textbf{13}--\textbf{29}. These Mind Moments are involved in the sensing and processing of sense data and ideas. I call this the “Sensing Zone.”\footnote{Other Abhidhamma texts do not use the term, “Sensing Zone.”}

The third group is Mind Moments \textbf{31}--\textbf{38}. These create wholesome kamma. I call this the “Faultless Zone.”\footnote{The Abhidhammattha Sangaha classifies Mind Moments \textbf{31}--\textbf{54} as being “beautiful” (\textit{sobhana}). Abhidhamma texts do not use the term, “Faultless Zone,” but “faultless” is used as a translation of “\textit{kusala}.”} In this zone, there is no \textbf{Attachment}, only \textbf{Non-attachment}. There is no \textbf{Aversion}, only \textbf{Non-aversion}. In some parts of the Faultless Zone, details of everything can be clearly seen, and these parts are the Mind Moments associated with \textbf{Understanding}.

In a Sutta called “Two Sorts of Thinking,” the Buddha stressed the importance of knowing if the mind is in the Danger Zone or the Faultless Zone.\footnote{MN 19: \url{http://www.accesstoinsight.org/tipitaka/mn/mn.019.than.html}} If the mind is in the Danger Zone, the Buddha advises us to reflect on the disadvantages of such thinking. If it is in the Faultless Zone, just be passively aware. In this Sutta, the Buddha also said that whatever you keep thinking about will become the inclination of your awareness. In other words, unwholesome thinking accumulates into an unwholesome perspective and unwholesome habits. Similarly, wholesome thinking accumulates into a wholesome perspective and wholesome habits.

\begin{figure}[h]
\centering
\input{./Diagrams/Life_Cont.pdf_tex}
\caption{Many Life-continuum Mind Moments arise between each Sensing/Thinking Process and during dreamless sleep.}
\label{fig:Life_Cont}
\end{figure}

The fourth group (Sense Sphere Resultant) are Mind Moments \textbf{39}--\textbf{46}. The main function of these Mind Moments is Life-continuum.\footnote{The Pāḷi term for Life-continuum is \textit{bhavaṅga}; literally factor (\textit{aṅga}) of existence (\textit{bhava}). The only reference to \textit{bhavaṅga} in the Abhidhamma \textit{Piṭaka} is in “Conditional Relations” (\textit{Paṭṭhāna}), Volume I page 159 and elsewhere, where it is described as preceding five-sense-door adverting (Mind Moment \textbf{28}).} These Mind Moments arise during dreamless sleep, and between all the sensing and thinking that goes on while we are awake; they arise when the mind is not sensing, not thinking and not in a jhāna\footnote{\url{http://en.wikipedia.org/wiki/Dhyana_in_Buddhism}} meditative state.

The fifth group (Sense Sphere Functional) are Mind Moments \textbf{47}--\textbf{54}. These arise only in the mind of an Arahat.\footnote{\url{http://en.wikipedia.org/wiki/Arhat}} Mind Moments \textbf{47}--\textbf{54} correspond to Mind Moments \textbf{31}--\textbf{38}, except that Mind Moments \textbf{47}--\textbf{54} do not create new kamma as Arahats do not create new kamma.

The sixth group (Fine Material Sphere Wholesome) are Mind Moments \textbf{55}--\textbf{59}. These arise during the five Fine Material jhānas and create wholesome kamma.

The seventh group (Fine Material Sphere Resultant) are Mind Moments \textbf{60}--\textbf{64}. These do not arise in humans; they are the Life-continuum Mind Moments for beings in higher Realms of Existence.

The eighth group (Fine Material Sphere Functional) are Mind Moments \textbf{65}--\textbf{69}. These arise only in the mind of an Arahat, when an Arahat experiences a Fine Material jhāna.

The ninth group (Immaterial Sphere Wholesome) are Mind Moments \textbf{70}--\textbf{73}. These arise during the four Immaterial jhānas and create wholesome kamma.

The tenth group (Immaterial Sphere Resultant) are Mind Moments \textbf{74}--\textbf{77}. These do not arise in humans; they are the Life-continuum Mind Moments for beings in much higher Realms of Existence.

The eleventh group (Immaterial Sphere Functional) are Mind Moments \textbf{78}--\textbf{81}. These arise only when an Arahat experiences an Immaterial jhāna.

The twelfth group (Supramundane Path) are Mind Moments \textbf{82}--\textbf{85}. These path Mind Moments arise at the moment of transition to each degree of sainthood. 

The thirteenth group (Supramundane Fruit) are Mind Moments \textbf{86}--\textbf{89}. These fruit Mind Moments allow a saint to enjoy the bliss of \textit{Nibbāna}. 

The Pāḷi name for the twelfth and thirteenth groups is translated as Supramundane, literally “transcending the world.”\footnote{\textit{Lokuttara}, combines world (\textit{loka}) and transcending (\textit{uttara}).} These Mind Moments take \textit{Nibbāna} as their object and are reserved for saints and for the Buddha.

\subsection*{Using this ``map of the mind" (Appendix 2 and Figure \ref{Handout3})}

Before we go into the details of Mind Moments, let’s take a step back and consider the question, “How can I use this map of the mind?”

To develop spiritually we need to look at the mind. If the mind is agitated, we need first to steady, settle, unify and compose the mind; perhaps by focusing on a neutral object such as the breath. When we look inwards with a calm mind, the most obvious Mind Moments will be in the Danger Zone (Mind Moments \textbf{1}--\textbf{12}) or in the Faultless Zone (Mind Moments \textbf{31}--\textbf{38}).

If unwholesome Mind Moments are found when looking at the mind, don’t be frustrated. Frustration is another unwholesome Mind Moment. If unwholesome Mind Moments are found, be mindful of their characteristics; “Ah, this is the grasping of \textbf{Attachment}.” or “Ah, the mind is experiencing \textbf{Aversion}.” Mindfulness converts the unwholesome Mind Moment into a wholesome Mind Moment with \textbf{Understanding}. As mentioned earlier, when facing unwholesome mind moments, the Buddha recommended an active approach of reflecting on the disadvantages of such thinking.

If wholesome Mind Moments are found when looking at the mind, don’t be attached to them; craving wholesome Mind Moments brings craving, an unwholesome Mind Moment. If wholesome Mind Moments are found, just be mindful of their characteristics; “Ah, the mind is calm, not floating away.” As mentioned earlier, when facing wholesome mind moments, the Buddha recommended a passive approach of just being aware.

Here is a \textbf{\textit{RADICAL}} approach to ``moment to moment mindfulness" based on a modern author.\footnote{Sayālay Susīlā: \url{http://www.sayalaysusila.net/files/Moment-to-Moment-Practice.pdf}} It is a traditional approach, but the word \textbf{\textit{RADICAL}} is used as an acronym. The \textbf{\textit{RADICAL}} approach is \textbf{\textit{R}}ecognize, \textbf{\textit{A}}ccept, \textbf{\textit{D}}epersonalize, \textbf{\textit{I}}nvestigate, \textbf{\textit{C}}ontemplate \textbf{\textit{A}}nicca or \textbf{\textit{C}}ontemplate \textbf{\textit{A}}nattā, and \textbf{\textit{L}}et go.

\pagebreak

\textbf{\textit{R}}ecognize means identifying the current Mind Moment.

\textbf{\textit{A}}ccept means accepting what is experienced just as it is; resisting the unpleasant is \textbf{Aversion}, and clinging to the pleasant is \textbf{Attachment}. 

\textbf{\textit{D}}epersonalize means being a stable observer, separate and unbiased; what is experienced is not “happening to me,” it is not “myself,” it is not “mine,” it simply is something to be observed. 

\textbf{\textit{I}}nvestigate is the point at which \textbf{Understanding} starts to play an important role. Look closely; what is experienced has characteristics that can be understood. 

\textbf{\textit{CA}} means Contemplate \textit{Anicca} or Contemplate \textit{Anattā}. Notice that what is experienced is impermanent or is a natural process, not Self. 

Finally, \textbf{\textit{L}}et go because there is always a new experience waiting.

%To repeat, the \textbf{\textit{RADICAL}} approach is \textbf{\textit{R}}ecognize, \textbf{\textit{A}}ccept, \textbf{\textit{D}}epersonalize, \textbf{\textit{I}}nvestigate, \textbf{\textit{C}}ontemplate \textbf{\textit{A}}nicca or \textbf{\textit{C}}ontemplate \textbf{\textit{A}}nattā, and \textbf{\textit{L}}et go. 

We are so obsessed with the urgent that we neglect the important. Why not make a habit of regularly dedicating a few moments to mindfulness using the \textbf{\textit{RADICAL}} approach? This will slowly train the mind. Just as the body benefits from regular exercise, the mind can also benefit from regular 
use of the \textbf{\textit{RADICAL}} approach.

\subsection*{Mind Moments 1--12 (Danger Zone)}

\begin{figure}[H]

\setlength{\tabcolsep}{0pt}
\renewcommand{\arraystretch}{1.1}
\begin{center}
\begin{tabular}{P{.05\textwidth}L{.15\textwidth}L{.2\textwidth}C{.05\textwidth}}
\toprule
\multicolumn{4}{c}{\tablesubheader{\textbf{Attachment}-rooted}}\\
\textbf{1} & Unprompted & \textbf{Wrong view} & \smiley \\
\textbf{2} & Prompted & \textbf{Wrong view} & \smiley \\
\textbf{3} & Unprompted & No \textbf{Wrong view} & \smiley \\
\textbf{4} & Prompted & No \textbf{Wrong view} & \smiley \\
\textbf{5} & Unprompted & \textbf{Wrong view} & \neutral \\
\textbf{6} & Prompted & \textbf{Wrong view} & \neutral \\
\textbf{7} & Unprompted & No \textbf{Wrong view} & \neutral \\
\textbf{8} & Prompted & No \textbf{Wrong view} & \neutral \\
\multicolumn{4}{c}{\tablesubheader{\textbf{Aversion}-rooted}} \\
\textbf{9} & Unprompted & Ill will & \frowney \\
\textbf{10} & Prompted & Ill will & \frowney \\
\multicolumn{4}{c}{\tablesubheader{\textbf{Delusion}-rooted}} \\
\textbf{11} & \multicolumn{2}{l}{Associated with doubt} & \neutral \\
\textbf{12} & \multicolumn{2}{l}{Associated with restlessness} & \neutral \\
\bottomrule
\end{tabular}
\end{center}
\begin{center}
\smiley\hspace{2mm} Pleasant \textbf{Feeling}\hspace{5mm}\neutral\hspace{2mm} Indifferent \textbf{Feeling}\hspace{5mm}\frowney\hspace{2mm}Unpleasant \textbf{Feeling}
\end{center}
\caption{The Danger Zone: Mind Moments \textbf{1}--\textbf{12} create unwholesome kamma.}
\label{fig:Danger}
\end{figure}

Let’s look at the Danger Zone, Mind Moments \textbf{1}--\textbf{12}, in more detail. This group is subdivided into \textbf{Attachment}-rooted, \textbf{Aversion}-rooted, and \textbf{Delusion}-rooted. These three unwholesome roots of \textbf{Attachment}, \textbf{Aversion} and \textbf{Delusion} are frequently  mentioned in the Suttas.\footnote{Iti 3.50: \url{http://www.accesstoinsight.org/tipitaka/kn/iti/iti.3.050-099.than.html\#iti-050}} Just as roots provide a foundation, support, strength and nourishment to a tree, a Mind Moment gets a foundation, support, strength and nourishment from its roots.\footnote{This is root condition (\textit{hetu-paccaya}); see Chapter 2 of “The Conditionality of Life” (see Footnote 2 for link).}

\pagebreak

In the Kālāma Sutta, the Buddha was asked how to evaluate teachings.\footnote{AN 3.65: \url{http://www.accesstoinsight.org/tipitaka/an/an03/an03.065.than.html}} The Buddha first advised the Kālāmas not to judge teachings based on tradition, conjecture or the charisma of the teacher.\footnote{At the time of the Buddha, there were three approaches to knowledge: oral tradition (from the ancient \textit{Vedas}: \url{http://en.wikipedia.org/wiki/Vedas}), logical reasoning (the basis of the recently-developed Upaniṣads: \url{http://en.wikipedia.org/wiki/Upanishads}, teachings of religious philosophy), and direct intuition of a teacher. In MN 100 (\url{http://metta.lk/tipitaka/2Sutta-Pitaka/2Majjhima-Nikaya/Majjhima2/100-sangarava-e1.html}), the Buddha put himself in the third category. See \url{http://www.ahandfulofleaves.org/documents/Early Buddhist Theory of Knowledge_Jayatilleke.pdf}} The Buddha then asked if \textbf{Attachment}, \textbf{Aversion} and \textbf{Delusion} were beneficial or unbeneficial, and if \textbf{Non-attachment}, \textbf{Non-aversion} and \textbf{Understanding} were beneficial or unbeneficial. The Kālāmas said that \textbf{Attachment}, \textbf{Aversion} and \textbf{Delusion} were unbeneficial and \textbf{Non-attachment}, \textbf{Non-aversion} and \textbf{Understanding} were beneficial. The Buddha then told the Kālāmas that teachings were good if they led to a decrease in \textbf{Attachment}, \textbf{Aversion} and \textbf{Delusion} and led to an increase in \textbf{Non-attachment}, \textbf{Non-aversion} and \textbf{Understanding}. This simple guideline can be applied when considering the motivation behind our own actions.

\subsubsection*{\textbf{Attachment}-rooted}

Mind Moments \textbf{1}--\textbf{8} are \textbf{Attachment}-rooted. Earlier, I referred to these Mind Moments as the sticky quicksand that traps the mind in the foggy Danger Zone. Sticky quicksand refers to the root of \textbf{Attachment} which is the primary characteristic of these Mind Moments. These eight Mind Moments also have a root of \textbf{Delusion}; this is the thick fog. They are in the Danger Zone because they create unwholesome kamma.

Reminds me of a joke: the young novice monk asks his teacher, “Master, is it allowable for a monk to use email?” The teacher replies, “Sure, as long as there are no attachments.”\footnote{From my experience, \textbf{Attachment} to email is a significant distraction during a meditation retreat. Yogis may want to turn off data (to avoid email and instant messages) and tell people to send an SMS in case of emergency.}

\textbf{Attachment} has many grades, ranging from simply enjoying a coffee to intense passion. A subtle form of \textbf{Attachment} is \textbf{Attachment} to sense objects; the desire to see things, to hear things, etc. There is a story in the Suttas warning about sense-desires.\footnote{SN 47.7: \url{http://www.accesstoinsight.org/tipitaka/sn/sn47/sn47.007.than.html}. Suttas warning of the dangers of sense-desires are directed at monks, who have already overcome most of the attachments faced by laypeople.} A hunter puts tar on a branch. The foolish monkey is curious and touches the tar with one hand. The hand gets stuck, so the monkey uses his other hand, his two feet and finally his mouth to try to get free. The monkey is then trapped in five ways, corresponding to the five senses-desires. 

I prefer the modern version of this story in which the hunter attaches a hollowed-out coconut to a tree.\footnote{See \url{http://en.wiktionary.org/wiki/monkey_trap}} Food is put inside the coconut to attract the monkey. The monkey grabs the food but with a clenched fist full of food, it is unable to extract its hand so it is trapped. I prefer this version because to free itself, the monkey just has to overcome its instinctive \textbf{Attachment} to the food and open its hand. Of course, overcoming instinctive \textbf{Attachment} is easier said than done for the monkey, and for the monkey’s distant cousins, we humans!

As you can see from Figure \ref{fig:Danger}, the eight Mind Moments in the \textbf{Attachment}-rooted group include every combination of unprompted or prompted, associated or not associated with \textbf{Wrong view}, and pleasant or indifferent \textbf{Feeling}.

\pagebreak

The first differentiating factor is unprompted or prompted. An unprompted Mind Moment arises spontaneously; if I do something without anybody trying to convince me, then the Mind Moment behind the action is unprompted. A prompted Mind Moment has to be instigated or induced by another person or by myself; if I need convincing to do something, then the Mind Moment is prompted. Prompting is not triggered by the object, prompting is related to influence by another person or oneself.

The second differentiating factor, either “associated with \textbf{Wrong view}” or “not associated with \textbf{Wrong view},” is extremely important, so let’s discuss \textbf{Wrong view} in a bit more detail. \textbf{Wrong view} is an opinion, a bias or a judgement that can impact one’s outlook, perspective, paradigm or belief.

Almost 500 years ago, Copernicus wrote that observations and calculations showed that the earth orbited the sun; that the earth is not the centre of the universe.\footnote{\url{http://en.wikipedia.org/wiki/Nicolaus_Copernicus}} Initially, the idea was ignored. Sixty years after Copernicus’ death, only 15 astronomers in all of Europe agreed. The idea was ignored because it challenged the prevailing “common sense” and church doctrine.

In the first lesson, I shared a scientific experiment showing that a “Self who decides” is an illusion created after a decision has already been made; there is no Self at the centre of the universe. I suspect that after an initial discomfort with the results of the experiment, your mind ignored results of the experiment because the ego-centric view of the universe is “common sense.” “Ignored” is a verb; the related noun is “ignorance” or mental blindness. Refusing to see things as they truly are is clinging to ignorance or \textbf{Attachment} to \textbf{Wrong view}.

But we don’t need a fancy experiment to show the illusion of a controlling Self. Consider the last time that confusion arose in the mind. Was there a conscious decision at that time? Was there a Self who said, “I think that I will choose to be confused now?” Of course not! Confusion arose naturally. After confusion arose, Self-identification took place, “I am confused.”

\begin{figure}[H]
\centering
\input{./Diagrams/Latent.pdf_tex}
\caption{\textit{\small The process by which latent defilements lead to \textbf{Wrong view} and how this reinforces the latent defilements.}}
\label{fig:Latent}
\end{figure}

A distorted view arises because of the latent defilements.\footnote{Latent defilements (\textit{anusaya}) are sensual desire, \textbf{Aversion}, \textbf{Wrong view}, \textbf{Doubt}, \textbf{Conceit}, desire for existence and \textbf{Delusion}; see AN 7.11: \url{http://www.accesstoinsight.org/tipitaka/an/an07/an07.011.than.html}} I will illustrate this process using two analogies; a man afraid of snakes and a woman craving spiritual progress.

\pagebreak

Imagine a man with a fear of snakes.\footnote{This famous analogy is used by Candrakirti in his Commentary on Aryadeva’s “Catuḥśataka.”} Most of the time, this fear has no impact on his life. The fear is latent, not arisen, and he may not even be aware of this fear. The man is walking down a dimly-lit path at night and there is a coil of rope ahead. His latent fear of snakes causes him to misperceive the coil as a snake. This distorted perception, combined with his latent fear of snakes, causes him to think about snakes. This distorted thinking, combined with his latent fear of snakes, convinces the man that there is a snake ahead. This distorted view causes the fear of snakes to arise. It is no longer a latent fear. This actual fear reinforces his latent tendency to be afraid of snakes.\footnote{Distortion of \textbf{Perception} is \textit{saññāvipallāsa}, distortion of thought is \textit{cittavipallāsa} and distortion of view is \textit{diṭṭhivipallāsa}. Others have translated \textit{vipallāsa} as “inversions” or “perversions;” \textit{Vipallāsa} is literally \textit{vi} + \textit{pari} + \textit{āsa} = “turned upside down.”} In other words, actual fear comes from latent fear plus distorted view, distorted view comes from latent fear plus distorted thought, distorted thought comes from latent fear plus distorted perception, and distorted perception comes from latent fear.

Imagine a woman who is attached to the idea of spiritual progress. This is a latent \textbf{Attachment}, waiting for a trigger to arise. While sitting on her cushion one day, some mundane experience arises in the woman’s mind. Her \textbf{Attachment} to spiritual progress causes her to misperceive the mundane experience as an insight. Driven by this distorted perception, and supported by her \textbf{Attachment} to the idea of spiritual progress, she thinks she has experienced an insight. This distorted thought, supported by her \textbf{Attachment} to the idea of spiritual progress convinces her that she has experienced an insight. This distorted view causes \textbf{Attachment} to arise. It is no longer a latent \textbf{Attachment}. This actual \textbf{Attachment} reinforces her latent tendency to be attached to the idea of spiritual progress.

\begin{figure}[H]

\centering
\includegraphics[width=0.8\linewidth]{./Diagrams/Cats}
\caption{\textbf{Delusion} distorts reality. \textbf{Wrong view} sees something that does not really exist (for example, the ego creates a huge “Self”).}
\label{fig:Cats}
\end{figure}

\textbf{Delusion} is mental blindness while \textbf{Wrong view} involves conviction that something that is wrong is correct. As you can see from the examples of the snake and the woman's insight, \textbf{Delusion} triggers distorted perception, \textbf{Delusion} and distorted perception trigger distorted thought, then \textbf{Delusion} and distorted thought trigger \textbf{Attachment} to \textbf{Wrong view}. When \textbf{Attachment} to \textbf{Wrong view} arises, the underlying \textbf{Delusion} is reinforced.

\pagebreak

\begin{figure}[H]
\centering
\includegraphics[width=0.9\linewidth]{./Diagrams/Perception}
\caption{Latent tendencies influence how an image is perceived; do you perceive the first image as a vase or as two faces? Psychologists use the Rorschach inkblot test (\url{http://en.wikipedia.org/wiki/Rorschach_test}), such as the second image, to gain insights into a person’s latent tendencies.}
\label{fig:Perception}
\end{figure}

The Buddha listed four types of \textbf{Wrong view} that are harmful to spiritual progress.\footnote{AN 4.49: \url{http://www.accesstoinsight.org/tipitaka/an/an04/an04.049.than.html}} First, taking what is impermanent as permanent is \textbf{Wrong view}. Second, taking what is unsatisfactory as pleasant is \textbf{Wrong view}. Third, taking what is impure to be pure is \textbf{Wrong view}. Fourth, taking what is non-self as Self is \textbf{Wrong view}. The Buddha frequently highlighted the danger of taking what is non-self as Self.\footnote{According to Visuddhimagga XXII.68 (see Footnote 2 for link) the Sotāpanna uproots all four types of \textit{diṭṭhivipallāsa} (no speech or action because of \textit{vipallāsa}). \textit{Saññāvipallāsa} and \textit{cittavipallāsa} related to \textit{anicca} and related to \textit{anattā} are also uprooted by the Sotāpanna. \textit{Saññāvipallāsa} and \textit{cittavipallāsa} related to \textit{asubha} are uprooted by the Anāgāmī. \textit{Saññāvipallāsa} and \textit{cittavipallāsa} related to \textit{dukkha} are uprooted by the Arahat.} He explained how the ego of the untrained mind twists and distorts whatever is perceived, to place a Self at the centre by imagining a relationship with what is experienced.\footnote{MN 1: \url{http://www.accesstoinsight.org/tipitaka/mn/mn.001.than.html}} The experience of pain becomes “my pain,” “I am in pain” or “it is painful to me.” All of this happens in the blink of an eye. The Buddha’s teachings, our own experience, and scientific experiments all tell us that “Self is an illusion.” In spite of all this evidence, we stubbornly cling to the \textbf{Wrong view} of Self.

The third differentiating factor is pleasant \textbf{Feeling} or indifferent \textbf{Feeling}. Dictionary.com\footnote{\url{http://dictionary.reference.com/browse/feeling}} has 14 definitions for “feeling.” In Buddhism, \textbf{Feeling} has one definition; the most basic experience of pleasant, unpleasant or indifferent.\footnote{The Abhidhamma identifies five types of \textbf{Feeling}: mental pleasant, mental unpleasant, mental indifferent, physical pleasant (pleasure) and physical unpleasant (pain).} When we see the word \textbf{Feeling} in the Suttas or Abhidhamma, it is important that we put aside all other meanings associated with the English word “feeling,” and think of \textbf{Feeling} as describing only a pleasant, unpleasant or indifferent experience. Later, we will discuss the contemplation of \textbf{Feeling} in the \textit{Satipaṭṭhāna} Sutta.

The \textbf{Attachment}-rooted Mind Moments are subdivided according to prompting, \textbf{Wrong view}, and \textbf{Feeling}; this subdivision may have been done to reflect the relative weightiness of the associated kamma. A Mind Moment that is unprompted or spontaneous may create weightier kamma than a prompted Mind Moment that depends on somebody else, or depends on self-reflection. A Mind Moment associated with \textbf{Wrong view} may create weightier kamma than a Mind Moment not associated with \textbf{Wrong view}.\footnote{For example, Milindapañha III,7,8 explains that a person who picks up a hot object unknowingly will suffer greater burns than a person who picks up a hot object knowingly.} A Mind Moment accompanied by pleasant \textbf{Feeling} may create weightier kamma than a Mind Moment accompanied by indifferent \textbf{Feeling}. Spontaneity, association with \textbf{Wrong view} and pleasant \textbf{Feeling} may all contribute to increasing the \textbf{Volition} of a Mind Moment, and weightiness of kamma depends on the level of \textbf{Volition}.

\pagebreak

Consider a boy spontaneously stealing an apple, with joy, convinced that there is nothing wrong with stealing. The boy’s Mind Moment is unprompted, associated with \textbf{Wrong view} and accompanied by pleasant \textbf{Feeling}. It is Mind Moment \textbf{1}.

Imagine that, knowing lying is wrong, I reluctantly compliment someone to make them happy. My Mind Moment is prompted, not associated with \textbf{Wrong view}, and accompanied by indifferent \textbf{Feeling}. It is Mind Moment \textbf{8}.

I am not saying that stealing the apple resulted in weightier kamma than giving a false compliment. Rather, I am saying that stealing the apple spontaneously probably created weightier kamma than if the boy had to be convinced by his friend to steal it. The weightiness of kamma depends on the strength of the underlying \textbf{Volition}, so it is difficult to compare different actions.

This weightiness of kamma has a parallel in our legal system. Is a judge not more likely to give a stiffer punishment if a defendant is happy they committed a crime or think that they did nothing wrong? Is a judge not more likely to be more lenient if a defendant was forced to commit the crime or committed the crime reluctantly?

\subsubsection*{\textbf{Aversion}-rooted}

Now let’s look at Mind Moments rooted in \textbf{Aversion}. Mind Moment \textbf{9} is unprompted and Mind Moment \textbf{10} is prompted. Earlier, I referred to these Mind Moments as burning hot and painful because they are accompanied by unpleasant \textbf{Feeling}. While the eight \textbf{Attachment}-rooted Mind Moments are attracted to the object, these two Mind Moments do not like their object or are not satisfied with their object.

Just as \textbf{Attachment} has many grades, from simply enjoying coffee to intense passion, \textbf{Aversion} also has many grades. \textbf{Aversion} could be as subtle as not accepting the object as it is, or as intense as blinding hatred. When there is fear, there is \textbf{Aversion} toward a future situation. When there is \textbf{Remorse} or regret, there is \textbf{Aversion} toward a past situation. When there is boredom, there is \textbf{Aversion} toward the current situation. When there is \textbf{Envy}, there is \textbf{Aversion} because I do not like that somebody has something that I do not. Hatred, anger, fear, \textbf{Remorse}, boredom and \textbf{Envy} are all accompanied by unpleasant \textbf{Feeling}.

Which Mind Moment arises at the moment a hunter kills for sport? Before killing, the hunter was looking for the animal and these Mind Moments would have been rooted in craving or \textbf{Attachment}. At the moment of killing, the hunter does not want the animal to continue living, so the Mind Moment is rooted in \textbf{Aversion}; not necessarily \textbf{Aversion} to the animal, but rather \textbf{Aversion} to the situation that the animal is alive. Since the killing is premeditated, the Mind Moment is prompted. This is Mind Moment \textbf{10}. After the killing, the hunter may be proud of his accomplishment and again, the Mind Moments would be rooted in \textbf{Attachment}.

The Buddha was once insulted by a Brahmin. The Buddha replied, “If you offer food to a guest and the guest refuses it, does the food not then belong to you? Similarly, I do not accept the insults that you have offered, so they are all yours.”\footnote{SN 7.2: \url{http://www.accesstoinsight.org/tipitaka/sn/sn07/sn07.002.than.html}}

We all know that hatred is never conquered by hatred, only by loving-kindness.\footnote{Dhammapada verse 5: \url{http://www.tipitaka.net/tipitaka/dhp/verseload.php?verse=005}}

\pagebreak

Here is an analogy about dealing with anger. Imagine we want to stop the steam coming from a pot of boiling water on the stove. The short-term quick fix, the one that we are immediately drawn to, is to put a lid on the pot to keep the steam from escaping while pressure builds up inside the pot. The person who sees things as they truly are understands that it is the nature of steam to arise when there is a source of heat. This person stops the steam by turning off the stove. Imagine that we are angry and getting ready to blow off steam. Short-term quick fix techniques such as “firm resolution,” “considering the Buddha’s example” and “considering the harmful effects of anger” deal with the symptoms. It is impossible to overcome anger using a strategy based on \textbf{Aversion} to the current situation. Only beautiful Mental Factors such as \textbf{Mindfulness}, \textit{mettā} and \textbf{Understanding} deal with the root cause and turn off anger at the source.

We may blame external things such as a person, a situation, etc. when \textbf{Aversion} arises, but actually, the cause of \textbf{Aversion} is internal. Eleanor Roosevelt said, “No one can make you feel inferior without your consent.”\footnote{\url{http://en.wikipedia.org/wiki/Eleanor_Roosevelt}\linebreak {\url{http://www.goodreads.com/author/quotes/44566.Eleanor_Roosevelt}}} This applies to all forms of \textbf{Aversion}, not just feeling inferior.

Over the years, I have been asked many times, “How do I get rid of anger?” but nobody has ever asked, “How do I stop enjoying my coffee?” To quote Rudyard Kipling, we need to “meet with triumph and disaster and treat those two impostors just the same.”\footnote{\url{http://en.wikipedia.org/wiki/Rudyard_Kipling} \newline \url{http://www.poetryfoundation.org/poem/175772}} The Buddha identified eight worldly conditions as gain and loss, status and disgrace, praise and blame, pleasure and pain.\footnote{AN 8.6: \url{http://www.accesstoinsight.org/tipitaka/an/an08/an08.006.than.html}} The saint sees these eight worldly conditions as they truly are; impermanent. But when facing these eight worldly conditions, you and I get caught up in \textbf{Attachment} and \textbf{Aversion}.

\subsubsection*{\textbf{Delusion}-rooted}

All 12 of the Mind Moments in the Danger Zone include the root of \textbf{Delusion}. This is why I mentioned earlier that the entire Danger Zone was engulfed in a dense fog causing mental blindness. 

Reminds me of a joke: if ignorance is bliss, why aren’t more people happy?

\textbf{Attachment}-rooted Mind Moments have mental blindness working in the background, covering up the nature of the object. If the object is seen as impermanent, how can there be \textbf{Attachment}? \textbf{Aversion}-rooted Mind Moments also have mental blindness working in the background, covering up the nature of the object. If the object is seen as naturally arising because of conditions, how can there be \textbf{Aversion}? \textbf{Attachment} and \textbf{Aversion} cannot arise without \textbf{Delusion} doing its job in the background. \textbf{Delusion} is like a magician who can fool you. However, once you know how the magician performs his tricks, he can no longer deceive and captivate you; you are no longer fooled.

\textbf{Delusion} is also like the director of a film. You never see the director on the screen, but the director influences everything from the background. The director controls the script, actors, lighting, camera angle and the music of the film. Everything is coordinated by the director to manipulate the audience’s emotions; \textbf{Attachment} and \textbf{Aversion}. On the other hand, if you were on the set while they were making the movie, it would be obvious what was really happening.

\begin{figure}[H]
\centering
\includegraphics[width=1.0\linewidth]{./Diagrams/Actors}
\caption{The Buddha criticized actors for causing \textbf{Attachment}, \textbf{Aversion} and \textbf{Delusion} in other people. Many salespeople and politicians also cause \textbf{Attachment}, \textbf{Aversion} and \textbf{Delusion} in other people.}
\label{fig:Actors}
\end{figure}

The head of an acting troupe once asked the Buddha about the rebirth destination for an actor.\footnote{SN 42.2: \url{http://www.accesstoinsight.org/tipitaka/sn/sn42/sn42.002.than.html}} The Buddha replied that an actor could be reborn in hell or as an animal because they cause \textbf{Attachment}, \textbf{Aversion} and \textbf{Delusion} in other people.\footnote{In this Sutta, it is clear that the Buddha was referring to actors whose focus was entertainment; actors whose intention is to educate the audience regarding the human condition may not be included in this category.} 

Mind Moments \textbf{1}--\textbf{8} have both \textbf{Delusion} and \textbf{Attachment}. Mind Moments \textbf{9} and \textbf{10} have both \textbf{Delusion} and \textbf{Aversion}. Mind Moments \textbf{11} and \textbf{12} have \textbf{Delusion}, but no \textbf{Attachment} and no \textbf{Aversion}.

Mind Moment \textbf{11} is associated with \textbf{Doubt}. \textbf{Doubt} is not the same as not being sure about a person’s name. \textbf{Doubt} refers to the uncertainty in the benefit of generosity, uncertainty about the benefit of morality or uncertainty about the benefit of spiritual development. At the end of the previous lesson, I said that the Theravāda Suttas we have today contain the teachings of the Buddha, but they are \textbf{not} the literal, verbatim “word of the Buddha.” This may qualify under the general English language usage of the word “doubt,” but does not imply Mind Moment \textbf{11}.

\textbf{Doubt} cannot be reduced just by thinking; direct experience is the way to reduce \textbf{Doubt}. Here is a metaphor to show how \textbf{Doubt} is uprooted by the Sotāpanna’s experience of \textit{Nibbāna}. Imagine that you are on the near side of a stream and want to get to the far side.\footnote{The near side of the stream is Self-view; the Sotāpanna has eradicated Self-view. See SN 35.197: \url{http://www.accesstoinsight.org/tipitaka/sn/sn35/sn35.197.than.html}} You start by convincing yourself saying, “I can jump over this stream.” This is blind faith; there is still a lot of \textbf{Doubt} in the mind. You study books on stream jumping, you listen to talks on how to jump across streams, you read about previous stream-jumpers and now when you say, “I can jump over this stream,” there is less \textbf{Doubt}. On your side of the stream, you practise jumping and become incredibly skilled, an Olympic-level jumper. Now when you say, “I can jump over this stream,” there is very little \textbf{Doubt}, but \textbf{Doubt} remains because you have not yet jumped over this stream. You jump over it and are now on the far side. Having jumped over the stream, there are no more conditions for \textbf{Doubt} to arise. There is not even the most subtle \textbf{Doubt} remaining. \textbf{Doubt} has been truly uprooted. Before this point, \textbf{Doubt} was increasingly reduced, but now and forever in the future, it has been uprooted.

Figure \ref{fig:Danger} indicates that Mind Moment \textbf{12} is associated with \textbf{Restlessness}. Actually, \textbf{Restlessness} arises in all of the unwholesome Mind Moments, but in Mind Moments \textbf{1}--\textbf{11}, \textbf{Restlessness} is in the background whereas in Mind Moment \textbf{12}, \textbf{Restlessness} is dominant.

\textbf{Restlessness} is the opposite of steadiness or calmness; it is confusion, unsteadiness, agitation, mental distraction or mental excitement. A restless mind is not interested in the object and treats the object superficially. 

Reminds me of a joke: If I had a dollar for every  time I was distracted, I wish I had some ice cream!

\textbf{Restlessness} arises very frequently, but most people are unaware of it. When a person first tries to meditate, they are often surprised at the \textbf{Restlessness} in the mind. It seems that the mind never wants to penetrate the object; that the mind is wild and untamed. To help meditators deal with \textbf{Restlessness}, the Buddha delivered a Sutta titled “The removal of distracting thoughts.”\footnote{MN 20: \url{http://www.accesstoinsight.org/tipitaka/mn/mn.020.soma.html}} The good news is that the \textbf{Volition} associated with Mind Moment \textbf{12} is very weak, so the unwholesome kamma created by a restless mind is not weighty.

\subsection*{Mind Moments 31--38 (Faultless Zone)}

\begin{figure}[H]

\setlength{\tabcolsep}{0pt}
\renewcommand{\arraystretch}{1.1}
\begin{center}
\begin{tabular}{P{.05\textwidth}L{.15\textwidth}L{.25\textwidth}C{.05\textwidth}}
\toprule
\multicolumn{4}{c}{\tablesubheader{Sense Sphere Wholesome}}\\
\textbf{31} & Unprompted & \textbf{Understanding} & \smiley \\
\textbf{32} & Prompted & \textbf{Understanding} & \smiley \\
\textbf{33} & Unprompted & No \textbf{Understanding} & \smiley \\
\textbf{34} & Prompted & No \textbf{Understanding} & \smiley \\
\textbf{35} & Unprompted & \textbf{Understanding} & \neutral \\
\textbf{36} & Prompted & \textbf{Understanding} & \neutral \\
\textbf{37} & Unprompted & No \textbf{Understanding} & \neutral \\
\textbf{38} & Prompted & No \textbf{Understanding} & \neutral \\
\bottomrule
\end{tabular}
\end{center}
\begin{center}
\smiley\hspace{2mm} Pleasant \textbf{Feeling}\hspace{5mm}\neutral\hspace{2mm} Indifferent \textbf{Feeling}
\end{center}
\caption{Mind Moments \textbf{31}--\textbf{38}: Faultless Zone.}
\label{fig:Faultless}
\end{figure}

Let’s now jump to Mind Moments that create wholesome kamma, Mind Moments \textbf{31}--\textbf{38}. Earlier, I called this the Faultless Zone. This group's subdivisions are “prompted and unprompted,” “with and without \textbf{Understanding}” and “pleasant and indifferent \textbf{Feeling}.” We have already discussed prompted and unprompted, and pleasant and indifferent \textbf{Feeling}; let’s now discuss \textbf{Understanding}.

In Pāḷi, the word for \textbf{Understanding} is \textit{Paññā}. The Abhidhamma considers \textit{Paññā} to be the foundation of wisdom, investigation, insight, right view and \textit{vipassanā}.\footnote{One Mental Factor listed in the Abhidhamma is often equivalent to multiple terms from the Suttas.} Mind Moments \textbf{31}--\textbf{38} will create wholesome kamma, and those Mind Moments associated with \textbf{Understanding} (Mind Moments \textbf{31}, \textbf{32}, \textbf{35} and \textbf{36}) will contribute to spiritual development.\footnote{Faultless Mind Moments without \textit{Paññā} may also contribute to spiritual development, depending on how one defines “spiritual development.” Mind Moments without \textit{Paññā} can establish a habit supporting the arising of future Mind Moments with \textit{Paññā}.}

\pagebreak

Just to be clear, \textbf{Understanding} or wisdom does not imply intelligence. There is a story in the Commentary of a monk who was unable to memorize a single verse after trying for four months, yet was able to become an Arahat.\footnote{\url{http://www.tipitaka.net/tipitaka/dhp/verseload.php?verse=025}}

Morality and thoughts of generosity always create wholesome kamma, and when associated with \textbf{Understanding}, this can also directly contribute to spiritual development. All forms of mental cultivation directly contribute to spiritual development. This includes meditation, studying the Dhamma, teaching the Dhamma and straightening of views.

So how do we apply \textbf{Understanding} in our daily life?\footnote{Taken from Visuddhimagga I.13 (see Footnote 2 for link).} Bad speech and bad action come from bad thoughts. Bad thoughts arise from latent tendencies. Working backwards, we start by addressing bad speech and bad actions through precepts, morality or virtue; in Pāḷi, this is called \textit{Sīla}. Bad thoughts are addressed through having a composed or concentrated mind; in Pāḷi, this is called \textit{Samādhi}. Latent tendencies are addressed through \textbf{Understanding} or \textit{Paññā}.

\begin{figure}[H]
\centering
\input{./Diagrams/Sila.pdf_tex}
\caption{Latent tendencies give rise to thoughts which are manifested as actions and speech. Sīla is the restraint of actions and speech, Samādhi calms the mind and Paññā exposes the latent tendencies.}
\label{fig:Sila}
\end{figure}

\textit{Sīla}, \textit{Samādhi} and \textit{Paññā} are progressive steps. Start with \textit{Sīla}, and that will support \textit{Samādhi} which is a condition for \textit{Paññā}. \textit{Paññā} deepens the quality of \textit{Sīla} leading to deeper \textit{Samādhi} and clearer \textit{Paññā}, and the cycle continues. The Noble Eightfold Path can also be analyzed in terms of \textit{Sīla}, \textit{Samādhi} and \textit{Paññā}. \textit{Sīla} is Right Speech, Right Action and Right Livelihood. \textit{Samādhi} is Right Effort, Right \textbf{Mindfulness} and Right Concentration. \textit{Paññā} is Right View and Right Thought.

The Commentary gives an analogy to explain the difference between perceiving, thinking and \textbf{Understanding} to emphasize that \textbf{Understanding} is not superficial. A baby, a child and a money-changer all see a coin. The baby perceives a shiny, round object. The child thinks, “This is money” and thinks about what money can buy, how money can be used. The money-changer understands the qualities of the coin and can detect a counterfeit coin. The baby perceives, the child thinks and the money-changer understands.

The Buddha talked about the blind men who each touch different parts of an elephant and then argue over “what is an elephant.”\footnote{Ud 6.4: \url{http://www.accesstoinsight.org/tipitaka/kn/ud/ud.6.04.than.html}} The blind men’s different perceptions led to incomplete thinking and arguments. \textbf{Understanding} would be viewing the big picture like the sighted person who can see the entire elephant. 

Reminds me of a joke: a group of blind elephants each touched different parts of a human with their foot and they all agreed that a human was flat.

Earlier, I mentioned how \textbf{Wrong view} comes from distorted perception and distorted thinking. The analogies of the snake, the coin and the elephant all remind us that we all are subject to distorted and incomplete perceptions, distorted and incomplete thinking. Assumptions can lead to \textbf{Wrong view} but investigation can lead to \textbf{Understanding}.

\pagebreak

\begin{figure} [H]
\begin{center}
\includegraphics[width=0.4\linewidth]{./Diagrams/Dana}
\end{center}
\caption{A mother and her child offer dāna to a monk.}
\label{fig:Dana}
\end{figure}

Let’s consider another example and identify its associated Mind Moment. A mother sees a monk on alms round through the window. Immediately, she collects some food and takes her young child outside. They both offer food to the monk. At the time of offering, the Mind Moment of the mother and the Mind Moment of the young child are both wholesome; the Mind Moments are in the Faultless Zone. The Mind Moment of the mother is unprompted, spontaneous, while the Mind Moment of the child is prompted and depended on the inducement of the mother. The Mind Moment of the mother is with \textbf{Understanding}; she sees this as part of her spiritual development, and the child’s Mind Moment is without \textbf{Understanding}; the child is too young to comprehend. The Mind Moment of the mother is accompanied by pleasant \textbf{Feeling} while the child’s Mind Moment may be accompanied by indifferent \textbf{Feeling}. Mother and child perform exactly the same action, but the mother’s kamma will be much weightier because her \textbf{Volition} was much stronger. The mother’s Mind Moment will be \textbf{31} and the child’s will be \textbf{38}. The mother’s Mind Moment \textbf{31} will directly contribute to her spiritual development. The child’s Mind Moment \textbf{38} may indirectly contribute to his spiritual development by establishing a habit of generosity; as the child gets older, generosity may arise with \textbf{Understanding} and directly contribute to his spiritual development.

\pagebreak

\subsection*{Mind Moments 13--29 (Sensing Zone)}

\begin{figure}[H]

\setlength{\tabcolsep}{0pt}
\renewcommand{\arraystretch}{1.1}
\begin{center}
\begin{tabular}{P{.05\textwidth}L{.3\textwidth}C{.05\textwidth}}
\toprule
 \multicolumn{3}{c}{\tablesubheader{Unwholesome Resultant}} \\
 \textbf{13} & Eye-consciousness & \neutral \\
 \textbf{14} & Ear-consciousness & \neutral \\
 \textbf{15} & Nose-consciousness & \neutral \\
 \textbf{16} & Tongue-consciousness & \neutral \\
 \textbf{17} & Body-consciousness & \frowney \\
 \textbf{18} & Receiving consciousness & \neutral \\
 \textbf{19} & Investigating consciousness & \neutral \\
 \multicolumn{3}{c}{\tablesubheader{Wholesome Resultant}} \\
 \textbf{20} & Eye-consciousness & \neutral \\
 \textbf{21} & Ear-consciousness & \neutral \\
 \textbf{22} & Nose-consciousness & \neutral \\
 \textbf{23} & Tongue-consciousness & \neutral \\
 \textbf{24} & Body-consciousness & \smiley \\
 \textbf{25} & Receiving consciousness & \neutral \\
 \textbf{26} & Investigating consciousness & \smiley \\
 \textbf{27} & Investigating consciousness & \neutral \\
\midrule
 \multicolumn{3}{c}{\tablesubheader{Functional}} \\
 \textbf{28} & Five-sense-door adverting & \neutral \\
 \textbf{29} & Determining consciousness & \neutral \\
\bottomrule
\end{tabular}
\end{center}
\begin{center}
\smiley\hspace{2mm} Pleasant \textbf{Feeling}\hspace{5mm}\neutral\hspace{2mm} Indifferent \textbf{Feeling}\hspace{5mm}\frowney\hspace{2mm}Painful \textbf{Feeling}
\end{center}
\caption{Mind Moments \textbf{13}--\textbf{29}: Sensing Zone. Note that Mind Moment \textbf{17} is with painful (physical) feeling and Mind Moment \textbf{24} is with pleasurable (physical) feeling.}
\label{fig:Sensing}
\end{figure}

We have covered the Danger Zone, Mind Moments \textbf{1}--\textbf{12}, and the Faultless Zone, Mind Moments \textbf{31}--\textbf{38}. Now let’s look at the Sensing Zone, Mind Moments \textbf{13}--\textbf{29}. These are involved in the sensing and processing of sense data and ideas. In a later lesson, I will go into the details of each of these Mind Moments, but at this point I will give only a brief overview of a few of them.

Mind Moments \textbf{13}--\textbf{17} and Mind Moments \textbf{20}--\textbf{24} are the sense-consciousness Mind Moments corresponding to the five physical senses. The Mind Moment that captures an image at the back of the retina can be either Mind Moment \textbf{13} or Mind Moment \textbf{20}; both are eye-consciousness Mind Moments. The only difference between them is that Mind Moment \textbf{13} is the result of unwholesome kamma, while Mind Moment \textbf{20} is the result of wholesome kamma.

Once the sense-consciousness Mind Moment has performed its function, the sense data is processed. Mind Moments \textbf{18} and \textbf{19} will process sense data that arose because of unwholesome kamma. Mind Moments \textbf{25} and either Mind Moment \textbf{26} or Mind Moment \textbf{27} will process sense data that arose because of wholesome kamma.

\pagebreak
\subsubsection*{Decision box}

\begin{figure}[H]
\centering
\input{./Diagrams/Room.pdf_tex}
\caption{Decision directs the flow of the mind from the input/trigger to one of the outputs.}
\label{fig:Room}
\end{figure}

Once sense data has been captured and processed, a decision needs to be made as to how to proceed. That decision is the function of Mind Moment \textbf{29} (Determining consciousness). If the object is an idea, then Mind Moments \textbf{13}--\textbf{28} are bypassed, and the mind jumps directly to Mind Moment \textbf{29} to make a decision. You will notice that Mind Moment \textbf{29} is unrelated to kamma. The mind does not make decisions because of kamma.

Let me give you an analogy to explain the interaction between the Danger Zone, the Faultless Zone and the Sensing Zone. Figure \ref{fig:Room} shows a “Decision box.” There is one entrance to the box from below and this is how a stimulus such as a \textbf{Visible-form}, \textbf{Sound}, idea, etc. enters the mind. There are three exits from the box leading to the Danger Zone and one exit from the box leading to the Faultless Zone.\footnote{Appendix 2 also shows a ``Decision box" connecting the sensing zone to the Danger/Faultless Zone.}

The exits are the mind’s reaction to the stimulus; the result of the decision made by Mind Moment \textbf{29}. The three exits to the left are based on \textbf{Attachment}, \textbf{Aversion} and \textbf{Delusion}, and lead to the Danger Zone. The exit to the right leads to the Faultless Zone.

In the previous lesson, I mentioned that the mind makes decisions naturally. I referred to experiments that show how the mind decides first, and then later the idea of a Self who decides is created. I mentioned that a concept of a Self is added by the mind as a justification or rationalization of the decision making process. Actually, it is the decisions made by Mind Moment \textbf{29} that determines how much time is spent in the Faultless Zone and how much time is spent in the Danger Zone.

So how does Mind Moment \textbf{29} decide? The answer is very simple: accumulations, not kamma. Wholesome accumulations naturally cause Mind Moment \textbf{29} to open the door to the Faultless Zone. Unwholesome accumulations naturally cause Mind Moment \textbf{29} to open one of the doors to the Danger Zone. So how can accumulations be changed or reinforced? The answer is very simple: training.

Simple is not the same as easy. The five precepts, the rules of training, are simple to understand, but keeping the precepts is sometimes difficult. I try to recite the five precepts at least once a day. When I recite them, I reflect deeply upon them. Repetitive actions done with strong \textbf{Volition} train the mind and reinforce accumulations.

Before moving on to the jhānas (Mind Moments \textbf{55}--\textbf{81}), I should briefly mention Mind Moment \textbf{30}. Mind Moment \textbf{30} is not involved with Sensing. This Mind Moment arises only in an Arahat or a Buddha and causes a smile. 

\pagebreak

\subsection*{Mind Moments 55--81}

\begin{figure}[H]
\setlength{\tabcolsep}{0pt}
\renewcommand{\arraystretch}{1.1}

\begin{center}
\noindent\begin{tabular}{P{.05\textwidth}L{0.15\textwidth}C{.05\textwidth}C{0.0295\textwidth}P{.05\textwidth}L{0.265\textwidth}C{.05\textwidth}}
\toprule
\multicolumn{3}{C{0.287\textwidth}}{\tablesubheader{Fine Material Sphere Wholesome}} & & \multicolumn{3}{c}{\tablesubheader{Immaterial Sphere Wholesome}}\\
\textbf{55} & First Jhāna & \smiley & & \textbf{70} & Infinity of space & \neutral \\
\textbf{56} & Second Jhāna & \smiley & & \textbf{71} & Infinity of consciousness & \neutral \\
\textbf{57} & Third Jhāna & \smiley & & \textbf{72} & Nothingness & \neutral \\
\textbf{58} & Fourth Jhāna & \smiley & & \textbf{73} & Neither perception nor & \neutral \\
\textbf{59} & Fifth Jhāna & \neutral & & & non-perception & \\
\midrule
\multicolumn{3}{C{0.287\textwidth}}{\tablesubheader{Fine Material Sphere Resultant}} & & \multicolumn{3}{c}{\tablesubheader{Immaterial Sphere Resultant}} \\
\textbf{60} & First Jhāna & \smiley & & \textbf{74} & Infinity of space & \neutral \\
\textbf{61} & Second Jhāna & \smiley & & \textbf{75} & Infinity of consciousness & \neutral \\
\textbf{62} & Third Jhāna & \smiley & & \textbf{76} & Nothingness & \neutral \\
\textbf{63} & Fourth Jhāna & \smiley & & \textbf{77} & Neither perception nor & \neutral \\
\textbf{64} & Fifth Jhāna & \neutral & & & non-perception & \\
\midrule
\multicolumn{3}{C{0.287\textwidth}}{\tablesubheader{Fine Material Sphere Functional}} & & \multicolumn{3}{c}{\tablesubheader{Immaterial Sphere Functional}} \\
\textbf{65} & First Jhāna & \smiley & & \textbf{78} & Infinity of space & \neutral \\
\textbf{66} & Second Jhāna & \smiley & & \textbf{79} & Infinity of consciousness & \neutral \\
\textbf{67} & Third Jhāna & \smiley & & \textbf{80} & Nothingness & \neutral \\
\textbf{68} & Fourth Jhāna & \smiley & & \textbf{81} & Neither perception nor & \neutral \\
\textbf{69} & Fifth Jhāna & \neutral & & & non-perception \\
\bottomrule
\end{tabular}
\end{center}

\begin{center}
\smiley\hspace{2mm} Pleasant \textbf{Feeling}\hspace{5mm}\neutral\hspace{2mm} Indifferent \textbf{Feeling}
\end{center}

\caption{Mind Moments \textbf{55}--\textbf{81}: Fine Material Sphere and Immaterial Sphere}
\label{fig:55to81}
\end{figure}

Let’s move to Mind Moments \textbf{55}--\textbf{69}, which relate to the jhānas. As the focus of these lessons is applying Abhidhamma to daily life, I will not spend a lot of time discussing jhāna practice. Jhāna practice pre-dates the Buddha; the Buddha talked about practising the jhānas before he was enlightened.\footnote{MN 26: \url{http://www.accesstoinsight.org/tipitaka/mn/mn.026.than.html}} Jhāna practice is part of the Buddha’s teaching; the definition of “Right Concentration” as part of the Noble Eightfold Path is the practice of the four jhānas.\footnote{SN 45.8: \url{http://www.accesstoinsight.org/tipitaka/sn/sn45/sn45.008.than.html}}

A slight paraphrase of the standard description of jhāna found in the Suttas reads as follows: “Before entering the first jhāna, one must temporarily subdue the senses and temporarily subdue the unwholesome Mental Factors of sense desire, ill will, \textbf{Sloth} and \textbf{Torpor}, \textbf{Restlessness} and worry, and \textbf{Doubt} which block the first jhāna from arising. The first jhāna arises with the Mental Factors of \textbf{Zest}, pleasant \textbf{Feeling}, thought, examination and \textbf{One-pointedness}. When thought and examination fall away, one enters the second jhāna. When \textbf{Zest} falls away, one enters the third jhāna. When pleasant \textbf{Feeling} changes to indifferent \textbf{Feeling}, one enters the fourth jhāna.”

\begin{figure}[H]
\centering
\setlength{\tabcolsep}{0pt}
\renewcommand{\arraystretch}{1.1}

\noindent\begin{tabular}{C{.16\textwidth} C{.16\textwidth} |
p{.05\textwidth} 
p{.05\textwidth}
p{.05\textwidth}
p{.05\textwidth}
p{.05\textwidth}}
\toprule
\tablesubheader{According to\newline Abhidhamma} & \tablesubheader{According\newline to Suttas} & \tablevsubheader{\hspace{-4mm}\textbf{Initial application}}
& \tablevsubheader{\hspace{-4mm}\textbf{Sustained application}}
& \tablevsubheader{\hspace{-4mm}\textbf{Zest}}
& \tablevsubheader{\hspace{-4mm}\textbf{One-pointedness}}
& \tablevsubheader{\hspace{-4mm}\textbf{Feeling}}\\
\midrule
1\textsuperscript{st} Jhāna & 1\textsuperscript{st} Jhāna & \tm & \tm & \tm & \tm & \tmsmiley \\
2\textsuperscript{nd} Jhāna & & & \tm & \tm & \tm & \tmsmiley \\
3\textsuperscript{rd} Jhāna & 2\textsuperscript{nd} Jhāna & & & \tm & \tm & \tmsmiley \\
4\textsuperscript{th} Jhāna & 3\textsuperscript{rd} Jhāna & & & & \tm & \tmsmiley \\
5\textsuperscript{th} Jhāna & 4\textsuperscript{th} Jhāna & & & & \tm & \tmneutral \\

\bottomrule
\end{tabular}
\begin{center}
\smiley\hspace{2mm} Pleasant \textbf{Feeling} \hspace{5mm} \neutral\hspace{2mm} Indifferent \textbf{Feeling}
\end{center}
\caption{Comparison of jhānas according to the Abhidhamma and according to the Suttas.}
\label{fig:Jhana}
\end{figure}

I am sometimes asked, “The Suttas mention four jhānas, why does the Abhidhamma have five?” The Suttas describe the transition from the first jhāna to the second jhāna as dropping both \textbf{Initial application} (thought) and \textbf{Sustained application} (examination). But there is a Sutta that describes concentration without thought but with examination, kind of part way between the first and second jhāna.\footnote{DN 33: \url{http://suttacentral.net/en/dn33}} The Abhidhamma defines this “part way between” jhāna as the second jhāna and therefore the second jhāna according to the Suttas becomes the third jhāna according to the Abhidhamma. With this additional jhāna according to the Abhidhamma, the third jhāna according to the Suttas is the fourth jhāna according to the Abhidhamma and the fourth jhāna is the fifth jhāna according to the Abhidhamma.

The Suttas describe the second jhāna, without thought and without examination, as “Noble Silence” because there is no more mental chatter.\footnote{SN 21.1: \url{http://www.accesstoinsight.org/tipitaka/sn/sn21/sn21.001.than.html}} Remember this when you see a “Noble Silence” sign in a meditation hall: perhaps they are asking you to achieve the second jhāna!

Mind Moments \textbf{55}--\textbf{59} arise when experiencing the first to fifth jhāna. These fine material jhānas are analogous to being in an aeroplane; a special environment detached from what is happening on the ground.

For the immaterial jhānas, the meditator turns to increasingly subtle objects: infinite space, infinite consciousness, nothingness and “neither-perception-nor-non-perception.”\footnote{To attain Mind Moment \textbf{70}, the meditator who has mastered the fifth jhāna spreads the meditation object from the fifth jhāna until it has become immeasurable and then contemplates on the “infinity of space” it pervaded. Mind Moment \textbf{71} takes Mind Moment \textbf{70} as its object (since Mind Moment \textbf{70} takes “infinity of space” as an object, this is called “infinity of consciousness”). Mind Moment \textbf{72} takes the voidness/nothingness of Mind Moment \textbf{70} as its object. In Mind Moment \textbf{73}, \textbf{Perception} (\textit{saññā}) is so subtle that it cannot be said to exist; however \textbf{Perception} remains in residual form so the object is “neither perception nor non-perception.”} These are Mind Moments \textbf{70}--\textbf{73}. These immaterial jhānas are analogous to being in a spaceship; an even more specialized environment, even more detached from what is happening on the ground.

\subsection*{Mind Moments 82--89}

\begin{figure}[H]
\setlength{\tabcolsep}{0pt}
\renewcommand{\arraystretch}{1.1}

\begin{center}
\noindent\begin{tabular}{P{.05\textwidth}L{0.25\textwidth}C{.05\textwidth}}
\toprule
\multicolumn{3}{c}{\tablesubheader{Supramundane Path}} \\
\textbf{82} & Sotāpanna path & * \\
\textbf{83} & Sakadāgāmī path & * \\
\textbf{84} & Anāgāmī path & * \\
\textbf{85} & Arahat path & * \\
\midrule
\multicolumn{3}{c}{\tablesubheader{Supramundane Fruit}} \\
\textbf{86} & Sotāpanna fruit & * \\
\textbf{87} & Sakadāgāmī fruit & * \\
\textbf{88} & Anāgāmī fruit & * \\
\textbf{89} & Arahat fruit & * \\
\bottomrule
\end{tabular}
\end{center}

{\small \noindent * Note: If the Supramundane Path Mind Moment is attained by means of the First Jhāna to the Fourth Jhāna, the Supramundane Path Mind Moment and the corresponding Supramundane Fruit Mind Moment will be accompanied by pleasant \textbf{Feeling} (\smiley). If the Supramundane Path Mind Moment is attained by means of the Fifth Jhāna or an immaterial Jhāna, the Supramundane Path Mind Moment and the corresponding Supramundane Fruit Mind Moment will be accompanied by indifferent \textbf{Feeling} (\neutral). If the Supramundane Path Mind Moments and Supramundane Fruit Mind Moments are differentiated according to the Jhāna by which they are attained, the number of Path Mind Moments increases from 4 to 20, the number of Fruit Mind Moments increases from 4 to 20 and the total number of Mind Moments increases from 89 to 121.}

\caption{Mind Moments \textbf{82}--\textbf{89}: Supramundane.}
\label{fig:82to89}
\end{figure}

Mind Moments \textbf{82}--\textbf{89} are the “world transcending” or Supramundane Mind Moments. These Mind Moments take \textit{Nibbāna} as their object and are reserved for saints and for the Buddha.

\begin{figure}[H]
\centering
\input{./Diagrams/Magga.pdf_tex}
\caption{Mind Moment \textbf{82} is the transition to the purified mind of a Sotāpanna.}
\label{fig:Magga}
\end{figure}

The arising of Mind Moment \textbf{82} in the mental stream of an individual represents a “change of lineage;” that the being is now a Sotāpanna.\footnote{According to the Suttas (AN 3.86: \url{http://www.accesstoinsight.org/tipitaka/an/an03/an03.086.than.html}), a Sotāpanna will be reborn a maximum of seven times (never in a Woeful Plane) before becoming an Arahat.} This is the first experience of \textit{Nibbāna}, and this experience impacts all future Mind Moments, including Mind Moments in future lives. After this experience of \textit{Nibbāna}, there are no more conditions to support the arising of \textbf{Doubt}, personality-belief or \textbf{Attachment} to rules and rituals to arise in a Sotāpanna’s mind.

\pagebreak

Earlier, I gave the analogy of jumping over the stream. Having already jumped over the stream, there are no conditions for \textbf{Doubt} to arise such as “can I do it or not?” After the experience of \textit{Nibbāna}, there are no more conditions for \textbf{Doubt} to arise in the mind of a Sotāpanna.

I also referred to the man with a latent fear of snakes, and how distortion of perception leads to distortion of thinking, distortion of thinking leads to distortion of views and distortion of view leads to unwholesome speech and action. The Sotāpanna’s experience of \textit{Nibbāna} convinces him that everything is just mind and matter (\textit{nāmarūpa}), and that there is no underlying Self. Distortions of perception and distortions of thinking may still arise in a Sotāpanna, but there are no more conditions for distortion of views and therefore, no speech or action that could lead to rebirth in a woeful state such as hell or as an animal.

The Sotāpanna’s experience of \textit{Nibbāna} also uproots \textbf{Attachment} to rules and rituals. The Sotāpanna knows that performance of rituals does not lead to liberation. Rather, it is the development of the mind that allows one to experience \textit{Nibbāna}. Chanting or bowing can be worthwhile practices, but it is the Noble Eightfold Path that leads to liberation.

When a Sotāpanna experiences Mind Moment \textbf{83}, he is “promoted” to the level of Sakadāgāmī, and when a Sakadāgāmī experiences Mind Moment \textbf{84}, he is “promoted” to the level of Anāgāmī. Sensuous craving and ill will are significantly weakened by the Sakadāgāmī and uprooted by the Anāgāmī. 

When an Anāgāmī experiences Mind Moment \textbf{85}, he is “promoted” to the level of Arahat, and there is no more craving for fine-material existence, no more craving for immaterial existence, no \textbf{Conceit}, no \textbf{Restlessness} and no more \textbf{Delusion}.\footnote{The 10 fetters that bind beings to \textit{saṃsāra} are listed in AN 10.13: \url{http://www.accesstoinsight.org/tipitaka/an/an10/an10.013.than.html}}

Some people are surprised that Conceit is not uprooted until one is an Arahat. They ask me, “If a Sotāpanna has eliminated personality belief, how can he still have \textbf{Conceit}?” Though a Sotāpanna sees beings as just \textit{nāmarūpa}, he may still compare this set of \textit{nāmarūpa} with that set of \textit{nāmarūpa}. \textbf{Conceit} is any form of comparison.

Once Mind Moment \textbf{82} has arisen, there is no condition to support the arising of Mind Moments \textbf{1}, \textbf{2}, \textbf{5}, \textbf{6} and \textbf{11}. Once Mind Moment \textbf{84} has arisen, there is no condition to support the arising of Mind Moments \textbf{9} and \textbf{10}. An Arahat cannot have any unwholesome Mind Moments and does not create new kamma, so once Mind Moment \textbf{85} has arisen, there is no condition to support the arising of Mind Moments \textbf{1}--\textbf{12}, \textbf{31}--\textbf{38}, \textbf{55}--\textbf{59} or \textbf{70}--\textbf{73}.

A Sotāpanna can experience the “bliss of \textit{Nibbāna}.” Mind Moment \textbf{82} arises only once, when the ordinary being becomes a Sotāpanna.\footnote{ Some people claim that the Abhidhamma model of a Path Mind Moment arising only once does not agree with the Suttas. They claim that the Suttas speak of eight noble ones (one realizing the Sotāpanna path, one realizing the Sotāpanna fruit, etc.) and this would not make sense if the Path Mind Moment arose only for an instant. Actually, the Sutta (AN 8.59) does not use the phrase “one realizing the Sotāpanna path.” The Sutta uses the term, “\textit{sotāpattiphalasacchikiriyāya paṭipanno}” which Bhikkhu Bodhi translates as “the one practising for realization of the fruit of stream entry.”} Mind Moment \textbf{86} is the Sotāpanna fruit; it takes \textit{Nibbāna} as an object. So when a Sotāpanna experiences the “bliss of \textit{Nibbāna},” Mind Moment \textbf{86} is arising. When a Sakadāgāmī experiences the “bliss of \textit{Nibbāna},” Mind Moment \textbf{87} is arising. When an Anāgāmī experiences the “bliss of \textit{Nibbāna},” Mind Moment \textbf{88} is arising. And finally, when an Arahat experiences the “bliss of \textit{Nibbāna},” Mind Moment \textbf{89} is arising. 

One of the blessings sometimes used by monks is, “May this be a condition leading to path and fruit knowledge.” What they are saying is, “May this be a condition leading to sainthood.”

\pagebreak

\subsection*{Categories of Mind Moments}

At the beginning of this lesson, I gave a brief description of 13 groups of Mind Moments in Appendix 2. This grouping is only one way of categorizing the Mind Moments.

A second way of categorizing them is according to kamma. The Mind Moments in the top row create new kamma. The Mind Moments in the middle row are the result of past kamma, and the Mind Moments in the bottom row are unrelated to kamma.

A third way of categorizing Mind Moments is according to roots. Mind Moments \textbf{1}--\textbf{12} have unwholesome roots, Mind Moments \textbf{13}--\textbf{30} have no roots, and the remaining Mind Moments have beautiful roots, such as \textbf{Non-attachment}, \textbf{Non-aversion} and \textbf{Understanding}.

A fourth way of categorizing Mind Moments is according to sphere. We will discuss this during a later lesson, but you can see from Appendix 2 (and from Figure \ref{Handout3}) that Mind Moments \textbf{1}--\textbf{54} are in the Sense Sphere category, Mind Moments \textbf{55}--\textbf{69} are in the Fine Material sphere category, Mind Moments \textbf{70}--\textbf{81} are in the Immaterial Sphere category, and Mind Moments \textbf{82}--\textbf{89} are in the Supramundane category.

A fifth way of categorizing Mind Moments is according to \textbf{Feeling}.

The first book of the Abhidhamma lists more than 40 ways of classifying Mind Moments according to the Suttas, and more than 120 ways of classifying Mind Moments according to the Abhidhamma. As you can see, the Abhidhamma is very thorough, very detailed and includes lots of lists!

\subsection*{Linkage to \textit{Satipaṭṭhāna} Sutta}

\subsubsection*{Contemplation of \textbf{Feeling}}

Let’s now discuss the contemplation of \textbf{Feeling} from the \textit{Satipaṭṭhāna} Sutta. In paragraph 32, when the Sutta says “contemplating \textbf{Feelings} in \textbf{Feelings},” it means the meditator should repeatedly observe \textbf{Feeling}, but limit the involvement to only observing the \textbf{Feeling}.\footnote{The Pāḷi word for “contemplate” is \textit{anupassi}; The prefix \textit{anu} gives emphasis and \textit{passati} is the Pāḷi verb, “to see.” The Visuddhimagga XXI.14 (see Footnote 2 for link) explains this as “sees again and again in various modes.”} In other words, when observing pleasant \textbf{Feeling}, do not slip into \textbf{Attachment}; when observing unpleasant \textbf{Feeling}, do not slip into \textbf{Aversion}; when observing indifferent \textbf{Feeling}, do not slip into \textbf{Restlessness}. Observing \textbf{Attachment}, \textbf{Aversion} and \textbf{Restlessness} is part of the practice of contemplation of consciousness, but this is not the practice of contemplation of \textbf{Feeling}.

Paragraph 33 asks the meditator to observe pleasant, unpleasant and indifferent \textbf{Feeling} in general, and then asks the meditator to observe pleasant, unpleasant and indifferent \textbf{Feeling} according to their source. The two sources of \textbf{Feelings} are “worldly” and “spiritual.” Worldly \textbf{Feelings} arise from \textbf{Contact} with the six sense objects. Spiritual \textbf{Feelings} are related to spiritual development such as generosity, morality, mental cultivation and jhāna.

Spiritual pleasant \textbf{Feeling} arises with Mind Moments \textbf{31} and \textbf{32} as well as the jhāna Mind Moments, accompanied by pleasant \textbf{Feeling}. Spiritual indifferent \textbf{Feeling} arises with Mind Moments \textbf{35} and \textbf{36} as well as the jhāna Mind Moments, accompanied by indifferent \textbf{Feeling}. Spiritual unpleasant \textbf{Feeling} can arise at different stages of insight. For example, the Commentary mentions the contemplation of pain to abandon the perception of pleasure; during this practice, spiritual unpleasant \textbf{Feeling} will arise.\footnote{Visuddhimagga XX.90 (see Footnote 2 for link).}

\subsubsection*{Contemplation of Consciousness}

\begin{figure}[H]
\centering
\input{./Diagrams/Aversion.pdf_tex}
\caption{The mind with \textbf{Aversion} often spins out of control, fixing itself to ideas which lead to more \textbf{Aversion}. Contemplation of Consciousness is \textbf{Mindfulness} of a recent Mind Moment (which may have had \textbf{Aversion}). A Mind Moment with \textbf{Mindfulness} is wholesome. In this way, an unwholesome Mind Moment can be a condition for a wholesome Mind Moment.}
\label{fig:Aversion}
\end{figure}

Moving on to the contemplation of consciousness, please look at paragraph 36. When the Sutta says, “know the consciousness with lust, as with lust,” we are being asked to recognize Mind Moments \textbf{1}--\textbf{8}. We can recognize them because they are sticky. When the Sutta says, “know the consciousness without lust, as without lust,” we are being asked to recognize Mind Moments other than \textbf{1}--\textbf{8}. The consciousness with hate are Mind Moments \textbf{9} and \textbf{10}. We can recognize them because they burn. The consciousness with ignorance are Mind Moments \textbf{1}--\textbf{12}. We can recognize them because they are foggy.

The shrunken state refers to Mind Moments that are prompted, not spontaneous, “a rigid and indolent state of mind.” The distracted state refers to Mind Moments with \textbf{Restlessness}. As mentioned earlier, \textbf{Restlessness} arises in all unwholesome Mind Moments, but in Mind Moments \textbf{1}--\textbf{11}, \textbf{Restlessness} is more in the background whereas in Mind Moment \textbf{12}, \textbf{Restlessness} is dominant.

The final four classifications of consciousness, “the developed state,” “the state with nothing higher,” “the concentrated state,” and “the freed state,” all refer to monitoring the mind that is in advanced states of meditative development such as the jhānas.

\pagebreak

\subsection*{Summary of Key Points}

\begin{itemize}

\item A Mind Moment consists of consciousness and a collection of Mental Factors; within the Mind Moment, consciousness has the function of awareness, and each of the Mental Factors performs its own individual function.

\item Appendix 2 and Figure \ref{Handout3} show a map of the mind, consisting of 89 Mind Moments:

\begin{itemize}

\item Mind Moments \textbf{1}--\textbf{12} (Danger Zone) create new unwholesome kamma.

\begin{itemize}

\item Mind Moments \textbf{1}--\textbf{8} are \textbf{Attachment}-rooted (sticky quicksand) and are classified according to:

\begin{itemize}

\item Unprompted/prompted; spontaneous/induced.

\item Associated with \textbf{Wrong view}/not associated with \textbf{Wrong view}.

\item \textbf{Feeling} (pleasant/indifferent).

\end{itemize}

\item Mind Moments \textbf{9}--\textbf{10} are \textbf{Aversion}-rooted (hot and painful) and are accompanied by unpleasant \textbf{Feeling}.

\item Mind Moments \textbf{11}--\textbf{12} are \textbf{Delusion}-rooted (thick fog); Mind Moment \textbf{11} is with \textbf{Doubt}, Mind Moment \textbf{12} is with \textbf{Restlessness}.

\end{itemize}

\item Mind Moments \textbf{13}--\textbf{29} (Sensing Zone) process sense data.

\item Mind Moments \textbf{31}--\textbf{54} (Faultless Zone) includes Mind Moments \textbf{31}--\textbf{38}, which create new wholesome kamma.

\begin{itemize}

\item These Mind Moments are classified according to:

\begin{itemize}

\item Unprompted/prompted; spontaneous/induced.

\item Associated with \textbf{Understanding}/not associated with \textbf{Understanding}.

\item \textbf{Feeling} (pleasant/indifferent).

\end{itemize}

\end{itemize}

\item Mind Moments \textbf{55}--\textbf{81} are related to jhāna meditative states.

\item Mind Moments \textbf{82}--\textbf{89} include the attaining of the four degrees of Sainthood and enjoying the bliss of \textit{Nibbāna}.

\end{itemize}

\item The ``decision box" analogy shows the mind naturally deciding between the Danger Zone and the Faultless Zone when triggered by the Sensing Zone.

\item Appendix 2 helps us to \textbf{\textit{R}}ecognize the current Mind Moment; \textbf{\textit{R}}ecognize is the first step in the \textbf{\textit{R}} \textbf{\textit{A}} \textbf{\textit{D}} \textbf{\textit{I}} \textbf{\textit{CA}} \textbf{\textit{L}} process (\textbf{\textit{R}}ecognize, \textbf{\textit{A}}ccept, \textbf{\textit{D}}epersonalize, \textbf{\textit{I}}nvestigate, \textbf{\textit{C}}ontemplate \textbf{\textit{A}}\textit{nicca}/\textbf{\textit{C}}ontemplate \textbf{\textit{A}}\textit{nattā}, \textbf{\textit{L}}et go).

\end{itemize}

Finally, in my opinion, the most important thing to remember about consciousness and Mind Moments is that we should know if the mind is in the Danger Zone or in the Faultless Zone. If the mind is in the Danger Zone, reflect upon the disadvantages of such thinking. If the mind is in the Faultless Zone, just be passively aware.

\newpage

\subsection*{Questions \& Answers}

\question{What is the duration of a Mind Moment?}

According to the commentaries, more than a billion Mind Moments occur in the time occupied by a flash of lightning.\footnote{See entry for \textit{citta-kkhaṇa} in “Buddhist Dictionary” (see Footnote 2 for link).} The Buddha said he could not come up with a simile to describe how fast the mind changed.\footnote{AN 1.48: \url{http://www.accesstoinsight.org/tipitaka/an/an01/an01.048.than.html}} We may believe that our senses are working in parallel; that we see, hear and think at the same time, but according to the Abhidhamma, these processes happen in very fast succession. We have heard stories of people whose lives have flashed before their eyes in a few seconds. This suggests that under certain circumstances, the mind can be aware of a lot of precise details in a very short period of time. When observing the mind, experienced meditators can observe a great level of detail.

\question{How does the Abhidhamma explain the concepts of “subconscious” and the “unconscious mind?”}

The Wikipedia article\footnote{\url{http://en.wikipedia.org/wiki/Subconscious}} on “subconscious” explains that subconscious is the part of consciousness that is not currently in focal awareness, and that since there is a limit to what can be held in conscious focal awareness, an alternative storehouse of one’s knowledge and prior experience is needed. According to the Abhidhamma, there is only one Mind Moment at a time, and this Mind Moment includes consciousness or awareness. Prior experience influences the Mind Moment through natural decisive support, which will be explained in a later lesson.

The Wikipedia article\footnote{\url{http://en.wikipedia.org/wiki/Unconscious_mind}} on the “unconscious mind” defines it as consisting of the processes in the mind that occur automatically and are not available to introspection, including memory and motivation; though these processes exist well under the surface of conscious awareness, they are theorized to exert an impact on behaviour. This is another way of describing the effect of natural decisive support condition.

The Life-continuum Mind Moments are in no way related to the concepts of “subconscious” and the “unconscious mind.” Life-continuum Mind Moments merely fill in the gaps between sensing and thinking processes and ensure that the mind continues during dreamless sleep. 

\question{What kind of Mind Moments arise in a person who is in a coma?}

I once attended a Buddhist conference where Ajahn Brahm\footnote{\url{http://en.wikipedia.org/wiki/Ajahn_Brahm}} discussed euthanasia. He spoke of a woman who had been in a coma for two years. The doctor told her son that she would never recover and asked about euthanasia. A few months later, the woman came out of her coma. 

While Ajahn Brahm spoke, there was a commotion in the audience because the woman involved was in the audience. She slowly made her way to the stage and explained that she could hear the conversation between the doctor and her son but could not communicate with them. It was very emotional. Clearly, in addition to Life-continuum Mind Moments, sensing and thinking may also happen during a coma.

\pagebreak

\question{In Figure \ref{Handout3}, is the top row labelled “Create New Kamma” directly linked to the middle row labelled “Result of Kamma?”}

The Life-continuum Mind Moment \textbf{60} is always the result of Mind Moment \textbf{55} arising in a previous life. In other words, to be reborn in a Brahma Realm, it is necessary to have attained the first jhāna in a previous existence. It is not possible to draw the same kind of connections involving Sense Sphere Mind Moments. For example, it is not possible to say that Mind Moment \textbf{39} is the result of Mind Moment \textbf{31}. As will be discussed in the lesson on Realms of Existence, it is actually much more complicated than this.

\question{Is the Abhidhamma useful for “socially-engaged Buddhists”?}

“Socially-engaged Buddhists” seek ways to apply the insights from the Suttas and meditation practice to situations of social, political, environmental, and economic suffering and injustice.\footnote{\url{http://en.wikipedia.org/wiki/Engaged_Buddhism}}

The Abhidhamma helps us recognize when the mind is in the Danger Zone, under the influence of the roots of \textbf{Attachment}, \textbf{Aversion} and \textbf{Delusion}. Consumerism, terrorism and the entertainment industry are examples of how the roots of \textbf{Attachment}, \textbf{Aversion} and \textbf{Delusion} have become institutionalized in today's society. The problems that the world faces today such as global hunger, terrorism and global warming have their roots in the Danger Zone.

The three unwholesome roots can be overcome by developing the three wholesome roots of \textbf{Non-attachment}, \textbf{Non-aversion} and \textbf{Understanding}. In my opinion, a “socially-engaged Buddhist” should focus on root causes rather than symptoms. I have signed petitions for causes that I believe in, but I see this as a “quick fix solution.” In my opinion, movements such as “Metta Round the World”\footnote{\url{http://mettaroundtheworld.org/}} (an initiative to encourage people to meditate and pray for world peace, harmony and stability) are examples of how a “socially-engaged Buddhist” can focus on root causes. Perhaps by making this Practical Abhidhamma Course available online, I am making a small contribution to increasing Understanding in the world. \smiley

\question{Does Sotāpannas know they are saints?}

I am of the opinion that nobody can be 100\% sure of their level of spiritual attainment. 

During the time of the Buddha, the Buddha himself could confirm the attainment of another; this happened at the conclusion to the Dhammacakkappavattana Sutta when the Buddha confirmed that Kondañña had become a Sotāpanna.

The “Points of Controversy” (page 184) clearly states that a disciple of the Buddha cannot have knowledge regarding the attainment of another person. According to the Theravāda doctrine, only a Buddha has this kind of knowledge.

