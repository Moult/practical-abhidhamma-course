\begin{center}
\textbf{\textit{Namo Tassa Bhagavato Arahato Sammā Sambuddhassa}} \\
\end{center}

This course is for Theravāda\footnote{\url{http://en.wikipedia.org/wiki/Theravada}} Buddhists with inquiring minds who want an introduction\footnote{Additional details can be found in the following texts (these texts are referenced in footnotes):\\
“\textbf{A Comprehensive Manual of Abhidhamma}” (\textit{Abhidhammattha Sangaha}) Bhikkhu Bodhi (\url{http://store.pariyatti.org/Comprehensive-Manual-of-Abhidhamma-A--PDF-eBook_p_4362.html})\\
“\textbf{Path of Purification}” (\textit{Visuddhimagga}) Bhikkhu Ñāṇamoli (\url{http://www.accesstoinsight.org/lib/authors/nanamoli/PathofPurification2011.pdf})\\
“\textbf{Buddhist Dictionary}” Nyanatiloka (Ñāṇatiloka) (\url{http://urbandharma.org/pdf/palidict.pdf})\\
“\textbf{Cetasikas}” Nina van Gorkom (\url{http://archive.org/details/Cetasikas})\\
“\textbf{The Buddhist Teaching on Physical Phenomena}” Nina van Gorkom (\url{http://archive.org/details/TheBuddhistTeachingOnPhysicalPhenomena})\\
“\textbf{The Conditionality of Life (Outline of the 24 conditions as taught in the Abhidhamma)}” Nina van Gorkom (\url{http://archive.org/details/TheConditionalityOfLife})\\
When publications of the Pali Text Society are referenced, page numbers refer to the English translation.} to the Abhidhamma with minimal Pāḷi.\footnote{\url{http://en.wikipedia.org/wiki/Pali}} It includes audio recordings of eight talks, handouts to accompany the talks and a document. The recordings, handouts and the document can be downloaded from \url{http://practicalabhidhamma.com/}.

Some people may prefer to listen to the recordings with the handouts in front of them\footnote{When listening to the talks for the first time, you may have an impression of “information overload”, like trying to take a drink from a fire hose! \smiley  If so, you may want to listen to the talks a second time or read the document.}. Others may find they absorb more by reading the document\footnote{The pdf document can also be annotated so you can add your own notes.} with the handouts in front of them (the document includes a transcript of the talks). For those interested in a lot more information, the document includes footnotes\footnote{These footnotes are not merely an academic convention, they are my invitation to you to explore further.} and links\footnote{When viewing this document on a laptop or a tablet, the links are active; clicking a link shows the online website.} to online resources such as Suttas, stories from the Dhammapada Commentary and Wikipedia articles. The document also includes Questions \& Answers for each talk. 

The talks cover selected topics from the Abhidhamma that are most practical and relevant to daily life. Though it is called a “Practical Abhidhamma Course,” it is also a practical Dhamma\footnote{In this context, “Dhamma” (\url{http://en.wikipedia.org/wiki/Dharma}) means both the Buddha’s teachings (capitalized “Dhamma”) and the Ultimate Realities of the Abhidhamma (lower case “dhamma”).} course using themes from the Abhidhamma.

The Dhamma and the Abhidhamma are \textbf{not} meant for abstract theorizing; the Dhamma and the Abhidhamma are meant for practical application. I hope you approach this course not only to learn new facts, but also to consider how you can improve yourself spiritually.\footnote{The Commentary speaks of a three-step progression: study (\textit{pariyatti}), practice (\textit{paṭipatti}) and realization (\textit{paṭivedha}).}

\color{red}

\subsubsection*{Acknowledgements}

The suggestion to prepare this Practical Abhidhamma Course came from Ayyā Medhānandī Bhikkhunī, after a talk I gave at Sati Sārāṇīya Hermitage (\url{http://satisaraniya.ca/}). I am very grateful to the Venerables who reviewed the draft and corrected technical inaccuracies. Many of the questions came from my friends/students. Oli Cosgrove proofread the document (twice). My son Dion helped with aesthetics and all things digital (\LaTeX, audio and webpage).


\color{black}