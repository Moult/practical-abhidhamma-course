\begin{center}
\textbf{\textit{Namo Tassa Bhagavato Arahato Sammā Sambuddhassa}} \\
\end{center}

This course is for Theravāda\footnote{\url{http://en.wikipedia.org/wiki/Theravada}} Buddhists with inquiring minds who want an introduction\footnote{Additional details can be found in the following texts (these texts are referenced in footnotes):\\
“\textbf{A Comprehensive Manual of Abhidhamma}” (\textit{Abhidhammattha Sangaha}) Bhikkhu Bodhi (\url{http://store.pariyatti.org/Comprehensive-Manual-of-Abhidhamma-A--PDF-eBook_p_4362.html})\\
“\textbf{Path of Purification}” (\textit{Visuddhimagga}) Bhikkhu Ñāṇamoli (\url{http://www.accesstoinsight.org/lib/authors/nanamoli/PathofPurification2011.pdf})\\
“\textbf{Buddhist Dictionary}” Nyanatiloka (Ñāṇatiloka) (\url{http://urbandharma.org/pdf/palidict.pdf})\\
“\textbf{Cetasikas}” Nina van Gorkom (\url{http://archive.org/details/Cetasikas})\\
“\textbf{The Buddhist Teaching on Physical Phenomena}” Nina van Gorkom (\url{http://archive.org/details/TheBuddhistTeachingOnPhysicalPhenomena})\\
“\textbf{The Conditionality of Life (Outline of the 24 conditions as taught in the Abhidhamma)}” Nina van Gorkom (\url{http://archive.org/details/TheConditionalityOfLife})\\
When publications of the Pali Text Society are referenced, page numbers refer to the English translation.} to the Abhidhamma with minimal Pāḷi.\footnote{\url{http://en.wikipedia.org/wiki/Pali}} 

The eight lessons in this course cover selected topics from the Abhidhamma that are most practical and relevant to daily life. Though it is called a “Practical Abhidhamma Course,” it is also a practical Dhamma\footnote{In this context, “Dhamma” (\url{http://en.wikipedia.org/wiki/Dharma}) means both the Buddha’s teachings (capitalized “Dhamma”) and the Ultimate Realities of the Abhidhamma (lower case “dhamma”).} course using themes from the Abhidhamma.

The Dhamma and the Abhidhamma are \textbf{not} meant for abstract theorizing; they are meant for practical application. I hope you approach this course not only to learn new facts, but also to consider how you can improve yourself spiritually.\footnote{The Commentary speaks of a progression: study (\textit{pariyatti}), practice (\textit{paṭipatti}) and realization (\textit{paṭivedha}).}

This document includes many diagrams and footnotes with links to online resources such as Suttas, stories from the Dhammapada Commentary and Wikipedia articles. The footnotes are not merely an academic convention, they are my invitation to you to explore further. When viewing this document on a laptop or a tablet, the links are active; clicking a link shows the online website. The document also includes Questions \& Answers for each lesson. You may find it convenient to print the appendices so you can refer to them while reading the document. This document can be downloaded from \url{http://practicalabhidhamma.com/}.

\subsubsection*{Acknowledgements}

Ayyā Medhānandī Bhikkhunī suggested that I prepare this Practical Abhidhamma Course after a talk I gave at Sati Sārāṇīya Hermitage (\url{http://satisaraniya.ca/}). I am very grateful to the Venerables who reviewed the draft and corrected inaccuracies. Many of the questions came from my friends/students. Oli Cosgrove proofread the document and improved the writing style. My son Dion helped with aesthetics and all things digital (\LaTeX, Inkscape and the webpage).
