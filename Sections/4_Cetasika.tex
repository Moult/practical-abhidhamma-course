\section{Mental Factors (\textit{Cetasika})}

Welcome to the fourth lesson of this Practical Abhidhamma Course. This lesson describes Mental Factors, the second of the four Ultimate Realities: \textit{citta}, \textit{cetasika}, \textit{rūpa} and \textit{Nibbāna}.\footnote{See Chapter 2 of “A Comprehensive Manual of Abhidhamma” and “Cetasikas” (see Footnote 2 for links).}

In the previous lesson, we discussed consciousness and Mind Moments. Within a Mind Moment, consciousness and the associated Mental Factors do not arise sequentially; they arise together, cease together and take the same object. Consciousness and the Mental Factors support each other and interact with each other. 

There are two ways of visualizing the relationship between Mind Moments, consciousness and Mental Factors. One perspective is that consciousness and Mental Factors are components of Mind Moments. Abhidhamma charts usually give this impression. A second perspective is that consciousness and Mental Factors are interdependent activities within a Mind Moment.\footnote{As an analogy, when we can say that a company is made up of management, sales and finance. One perspective is that management, sales and finance are positions or people (a company is a collection of nouns). Another perspective is that management, sales and finance are interdependent activities (a company is a collection of verbs). Each perspective provides a different insight into the nature of the thing called a company.}

Just because all Mental Factors within a Mind Moment are interdependent and function as a team, does not imply that they are all equal. Certain Mental Factors such as \textbf{Wish to do}, \textbf{Energy} and \textbf{Understanding} can dominate the other Mental Factors.\footnote{This is “conascence predominance condition” (\textit{sahajātādhipati-paccaya}); see Chapter 4 of “The Conditionality of Life” (see Footnote 2 for link).} 

Mental Factors arise within a Mind Moment with varying intensity. For example, Mind Moment \textbf{9} can arise with slight \textbf{Aversion} when you are bored and do not accept the current situation. At the other extreme, Mind Moment \textbf{9} can arise with strong \textbf{Aversion} during moments of intense hatred. Boredom and intense hatred are both Mind Moment \textbf{9}.

The Abhidhamma identifies 52 Mental Factors and the Visuddhimagga provides a structured definition for each Mental Factor. It defines a Mental Factor in four ways: the “characteristic” is the main quality of this Mental Factor; the “function” defines the role that this Mental Factor plays or the goal that this Mental Factor achieves in the Mind Moment, the “manifestation” describes how this Mental Factor is experienced during meditation, and the “proximate cause” is the main thing that typically causes this Mental Factor to come into prominence.

In this lesson, we will review each Mental Factor in a series of charts.\footnote{See Figure \ref{VarUni} as an example of a chart.} Each chart will include the primary English translation (in bold font), alternate English translations (in regular font),\footnote{Primary and alternative translations are provided because it is often difficult to translate jargon from one language to another. For example, the meaning of the Buddhist term “\textbf{Contact}” is a combination of the English words “contact” and “sense impression.” From our discussion of “\textbf{Feeling}” in the last lesson, you can see how this Buddhist term has a meaning that combines the English words “feeling,” “sensation” and “experience.”} the Pāḷi term,\footnote{If you want to start learning a bit of Pāḷi, the names of the Mental Factors are a good starting point because these are important words in the Suttas.} and the details from the Visuddhimagga definition. If part of the Visuddhimagga definition is unclear  to you, I suggest that you ignore it; these details are not very important for a beginner.

\pagebreak

\begin{figure}[H]
\centering
\input{./Diagrams/Ment_Fact.pdf_tex}
\caption{{\small \textit{The types of Mental Factors that arise in each Mind Moment will depend on the type of Mind Moment }(Unwholesome/Rootless/Wholesome).}}
\label{fig:MentalFactors}
\end{figure}

In the last lesson, we classified Mind Moments as Unwholesome (Mind Moments \textbf{1}--\textbf{12}), Rootless/Ethically Neutral (Mind Moments \textbf{13}--\textbf{30}) or Wholesome (Mind Moments \textbf{31}--\textbf{89}). There are three classifications of Mental Factors: Ethically-variable, Unwholesome and Beautiful. Within each of these three classifications, there is a group of Universal Mental Factors and a group of Occasional Mental Factors.\footnote{\textbf{Contact} is an example of a universal ethically-variable Mental Factor. \textbf{Initial application} is an example of an occasional ethically-variable Mental Factor. \textbf{Delusion} is an example of a universal unwholesome Mental Factor. \textbf{Attachment} is an example of an occasional unwholesome Mental Factor. \textbf{Faith} is an example of a universal beautiful Mental Factor. \textbf{Understanding} is an example of an occasional beautiful Mental Factor.}

Ethically-variable Mental Factors arise in all Mind Moments. The ethically-variable Mental Factors take on the character of the Mind Moment; unwholesome, wholesome or neutral. So when \textbf{Contact} arises in Mind Moment \textbf{1}--\textbf{12} it is an unwholesome \textbf{Contact}, when \textbf{Contact} arises in Mind Moment \textbf{13}--\textbf{30} it is an ethically-neutral \textbf{Contact} and when \textbf{Contact} arises in Mind Moment \textbf{31}--\textbf{89} it is a wholesome \textbf{Contact}.

Unwholesome Mental Factors arise only in unwholesome Mind Moments, so \textbf{Delusion} will arise only in Mind Moments \textbf{1}--\textbf{12}. Similarly, beautiful Mental Factors arise only in wholesome Mind Moments, so \textbf{Faith} will arise only in Mind Moments \textbf{31}--\textbf{89}.

The group of universal Mental Factors will always arise in the appropriate type of Mind Moment while the group of occasional Mental Factors will sometimes arise in the appropriate type of Mind Moment. For example, \textbf{Contact} arises in all 89 Mind Moments, \textbf{Initial application} arises in some of the 89 Mind Moments, \textbf{Delusion} arises in all Mind Moments \textbf{1}--\textbf{12}, \textbf{Attachment} arises in some of the Mind Moments \textbf{1}--\textbf{12}, \textbf{Faith} arises in all Mind Moments \textbf{31}--\textbf{89} and \textbf{Understanding} arises in some of the Mind Moments \textbf{31}--\textbf{89}. 

\pagebreak

\subsection*{Benefits of Learning the Mental Factors}

There are two benefits of learning the Mental Factors. 

The Mental Factors were extracted from the Suttas as the most important doctrinal terms used to describe the mind. The Mental Factors are technical jargon used in the Suttas, and learning them improves our understanding of the Suttas. Dhamma talks related to the mind also draw upon these terms, and knowing them helps us to a better understanding of Dhamma talks. This is especially true of Dhamma talks given during meditation retreats.

The second benefit of learning the Mental Factors is that they can help you identify the current Mind Moment. You may remember that in the previous lesson, I shared a \textbf{\textit{RADICAL}} practice: \textbf{\textit{R}} is for Recognize, \textbf{\textit{A}} is for Accept, \textbf{\textit{D}} is for Depersonalize, \textbf{\textit{I}} is for Investigate, \textbf{\textit{CA}} is for Contemplate \textit{Anicca} or Contemplate \textit{Anattā} and \textit{\textbf{L}} is for Let go. 

The first step is to recognize the Mind Moment. If there is strong \textbf{Attachment} or strong \textbf{Aversion}, it is easy to recognize the Mind Moment as being unwholesome. However, most of the time there is no strong \textbf{Attachment} and no strong \textbf{Aversion}, so recognizing the Mind Moment can be more challenging.

The Commentary gives an example of a soup that contains a variety of spices. Once the spices have been mixed, it is impossible to completely isolate one flavour from the others, but it is possible to distinguish the individual spices in the combination through their individual characteristics. In a similar way, Mental Factors can be identified through their individual characteristics. If a Mental Factor is unwholesome, we know the mind is in the Danger Zone. If a Mental Factor is beautiful, we know the mind is in the Faultless Zone.

During this lesson, I will focus mainly on recognizing Mind Moments \textbf{1}--\textbf{12}, the unwholesome Mind Moments in the Danger Zone, and recognizing Mind Moments \textbf{31}--\textbf{38} in the Faultless Zone. These are the Mind Moments that are most obvious. The other Mind Moments tend to be more subtle and difficult to observe.\footnote{Kamma-result condition (\textit{vipāka paccaya}) causes consciousness and Mental Factors in these Mind Moments to be “passive and quiescent.” See Chapter 12 of “The Conditionality of Life” (see Footnote 2 for link).}

For example, you may sit motionless and think, “I am doing this to have wholesome Mind Moments,” but even a chicken can sit motionless. Perhaps while sitting, you are craving some meditative experience, craving some \textbf{Understanding}, not even realizing that you are lost in \textbf{Attachment}. If you crave \textbf{Understanding}, you get craving, not \textbf{Understanding}. Having a clear understanding of the Mental Factors will help you to identify and recognize the current Mind Moment may be in the Danger Zone, rooted in \textbf{Attachment}.

Recognizing Mind Moments supports the development of Right Effort which is part of the Noble Eightfold Path. Right Effort involves recognizing, stopping and preventing unwholesome Mind Moments, and recognizing, initiating and extending wholesome Mind Moments. I hope that the material from these lessons will help you to develop Right Effort.

\subsection*{Ethically-variable Mental Factors}

\subsubsection*{Universals}

\begin{figure} [H]

\setlength{\tabcolsep}{0pt}
\renewcommand{\arraystretch}{1.1}

\noindent\begin{tabular}{P{0.25\textwidth} | C{.185\textwidth} C{0.185\textwidth} C{0.185\textwidth} C{0.195\textwidth} m{0mm}}
\toprule
 & \tableheader{Characteristic} & \tableheader{Function} & \tableheader{Manifestation} & \tableheader{Proximate Cause}\\
\midrule
\textbf{Contact}/\newline Sense impression\newline \textit{Phassa} & Mentally touching object & Impact/\newline impingement & Base + object + consciousness & Object in avenue of awareness &\\[12mm]
\textbf{Feeling}/\newline Sensation/Experience\newline \textit{Vedanā} & Being felt & Experiencing object & Relishing of the Mental Factors & Tranquillity &\\[12mm]
\textbf{Perception}/\newline Recognition\newline \textit{Saññā} & Perceiving object’s\newline qualities/Noting & Recognizing/\newline Marking & Brief interpretation\newline of object & Object as it appears &\\[12mm]
\textbf{Volition}/\newline Intention/Will\newline \textit{Cetanā} & State of willing & Accumulate kamma & Directing/\newline Organizing & Consciousness/\newline Mental Factors &\\[12mm]
\textbf{One-pointedness}/ Concentration\newline \textit{Ekagattā} & Non-scattering/\newline Non-distraction & Uniting Mental Factors & Peace & Happiness &\\[12mm]
\textbf{Attention}/\newline Reflection\newline \textit{Manasikāra} & Driving Mental Factors to object & Joining Mental Factors to object & Facing object & Object &\\[12mm]
\textbf{Life faculty}/\newline Vitality\newline \textit{Jīvitindriya} & Ceaseless watching & Maintaining life & Establishment & Consciousness/\newline Mental Factors &\\[12mm]
\bottomrule
\end{tabular} 

\caption{The seven universal ethically-variable Mental Factors that arise in all 89 Mind Moments.}
\label{VarUni}
\end{figure}

The first ethically-variable Mental Factor is \textbf{Contact}.\footnote{Defined in Visuddhimagga XIV.134, explained in Chapter 2 of “Cetasikas” (see Footnote 2 for links).} The Suttas explain that when a \textbf{Visible-form} strikes the sensitive part of the eye, eye-consciousness arises naturally. The meeting of the three (\textbf{Visible-form}, sensitive part of the eye and eye-consciousness), is defined as \textbf{Contact}.\footnote{MN 48: \url{http://www.accesstoinsight.org/tipitaka/mn/mn.148.than.html\#phassa}} From this definition, we can see that \textbf{Contact} is not physical impact, but rather sense impression. This is why the Suttas define \textbf{Contact} according to which sense is involved: eye-\textbf{Contact}, ear-\textbf{Contact}, nose-\textbf{Contact}, tongue-\textbf{Contact}, body-\textbf{Contact} and mind-\textbf{Contact}. Dependent origination states that the six senses are a condition for the arising of \textbf{Contact} and, in turn, \textbf{Contact} is a condition for the arising of the next ethically-variable Mental Factor, \textbf{Feeling}.

We discussed the Mental Factor of \textbf{Feeling} in the previous lesson.\footnote{Defined in Visuddhimagga XIV.125, explained in Chapter 3 of “Cetasikas” (see Footnote 2 for links).}

\pagebreak

\textbf{Perception} is the next Mental Factor.\footnote{Defined in Visuddhimagga XIV.129, explained in Chapter 4 of “Cetasikas” (see Footnote 2 for links).} In the previous lesson, I discussed perception working at a high level in the story of the snake. The man's misperceiving a coil of rope as a snake happened in a fraction of a second, but this was long enough for millions of processes to occur. Within each of these processes there were multiple Mind Moments and \textbf{Perception} arose in each Mind Moment. So \textbf{Perception}, or in this case misperception, works at a high level, but as a Mental Factor, \textbf{Perception} also works on a microscopic level within a Mind Moment.\footnote{AN 6.63: \url{http://www.accesstoinsight.org/tipitaka/an/an06/an06.063.than.html\#part-3}\\SN 27.6: \url{http://www.accesstoinsight.org/tipitaka/sn/sn27/sn27.001-010.than.html\#sn27.006}}

Within a Mind Moment, there are two roles for the Mental Factor \textbf{Perception}: first, to mark the object so it can be recognized later and second, to recognize the object that has been previously marked. These two roles work at a microscopic level to allow continuity; an object is ``passed" between Mind Moments and processes.\footnote{The relationship between \textbf{Perception} and “memory" is explored in pages 111--118 of “Abhidhamma Studies" (\url{http://www.dhammatalks.net/Books9/Nyanaponika_Thera_Abhidhamma_studies.pdf}).}

Now on to \textbf{Volition}, the next Mental Factor.\footnote{Defined in Visuddhimagga XIV.135, explained in Chapter 5 of “Cetasikas” (see Footnote 2 for links).} \textbf{Volition} organizes and coordinates consciousness and all the Mental Factors so that they work as a team. \textbf{Volition} plays this managerial role in all Mind Moments. In kamma-producing Mind Moments, \textbf{Volition} has an additional responsibility of creating new kamma. The Buddha said, “\textbf{Volition} is kamma,” so kamma comes from the mind, not from speech or action.\footnote{AN 6.63: \url{http://www.accesstoinsight.org/tipitaka/an/an06/an06.063.than.html\#part-5}} All speech and actions start in the mind, and it is in the mind that kamma is created.\footnote{Dhammapada verses 1 \& 2:  \url{http://www.accesstoinsight.org/tipitaka/kn/dhp/dhp.01.budd.html}} In kamma-creating Mind Moments, \textbf{Volition} is like the manager who also rolls up his sleeves and gets things done; the extra force generates kamma.

Moving on to \textbf{One-pointedness}.\footnote{Defined in Visuddhimagga XIV.139, explained in Chapter 7 of “Cetasikas” (see Footnote 2 for links).} The commentaries equate \textbf{One-pointedness} with concentration.\footnote{Visuddhimagga III.2 (See Footnote 2 for link).} “Unification of mind” and “singleness of mind” are other translations of \textbf{One-pointedness}. \textbf{One-pointedness} is like water that binds together substances to form one concrete mass; it allows the Mind Moment to focus on one object. Unwholesome \textbf{One-pointedness} arises when a hunter aims his gun and wholesome \textbf{One-pointedness} arises during meditation.\footnote{The \textit{Dhammasaṅgaṇī} mentions that all Mind Moments that create new kamma except Mind Moment \textbf{11} (associated with \textbf{Doubt}) include self-collectedness, faculty of concentration, power of concentration, right concentration path factor, quiet and balance (all of these are synonyms for \textbf{One-pointedness}). Mind Moments \textbf{11}, \textbf{13}--\textbf{27} and \textbf{39}--\textbf{46} only include self-collectedness but not the other items; this suggests that \textbf{One-pointedness} is present, but quite weak in Mind Moments \textbf{11}, \textbf{13}--\textbf{27} and \textbf{39}--\textbf{46}.}

The next Mental Factor is \textbf{Attention}.\footnote{Defined in Visuddhimagga XIV.152, explained in Chapter 8 of “Cetasikas” (see Footnote 2 for links).} At a high level, wise \textbf{Attention} plays an important role in our practice; the Buddha named wise \textbf{Attention} as the most important internal factor for spiritual development.\footnote{Iti 1.16: \url{http://www.accesstoinsight.org/tipitaka/kn/iti/iti.1.001-027.than.html\#iti-016} \\ The most important external factor for spiritual development is a good friend in the Dhamma (\textit{kalyāṇa-mitta}).} Wise \textbf{Attention} means looking at the situation through the lens of the Dhamma. Wise \textbf{Attention} can be applied at all times, but I find it useful before eating to reflect on the Buddha’s advice to “use this food not for intoxication, not for physical beauty, but simply for the survival and continuance of the body, for assisting in the practice.”\footnote{See page 437 of \url{http://www.accesstoinsight.org/lib/authors/thanissaro/bmc1.pdf}}

When things happen to us, it is useful to apply wise \textbf{Attention} and reflect on the eight worldly conditions mentioned by the Buddha: gain and loss, fame and obscurity, praise and blame, pleasure and pain; not to be attracted to one worldly condition or repelled by the other.\footnote{AN 8.6: \url{http://www.accesstoinsight.org/tipitaka/an/an08/an08.006.than.html}}

Wise \textbf{Attention} supports the arising of beautiful Mental Factors. Wise \textbf{Attention} is a proximate cause for many of the beautiful Mental Factors, and unwise \textbf{Attention} is a proximate cause for many of the unwholesome Mental Factors.

At the microscopic level, inside the Mind Moment, \textbf{Attention} acts like the rudder of a ship, directing, steering or turning consciousness and all of the Mental Factors toward the object. 

The final universal ethically-variable Mental Factor is \textbf{Life faculty}.\footnote{Defined in Visuddhimagga XIV.138, explained in Chapter 8 of “Cetasikas” (see Footnote 2 for links).} The function of \textbf{Life faculty} is to sustain the life of consciousness and the Mental Factors, but only for the brief instant that they exist.

\begin{figure}[H]
\centering
\input{./Diagrams/Eye.pdf_tex}
\caption{The seven universal ethically-variable Mental Factors in Mind Moment \textbf{13}.}
\label{fig:Eye-cons}
\end{figure}

The seven universal ethically-variable Mental Factors arise in all Mind Moments. Some Mind Moments, such as Mind Moment \textbf{13}, have only the universal ethically-variable Mental Factors. Let’s consider how consciousness and the Mental Factors in Mind Moment \textbf{13}, eye-consciousness, work as a team. According to the Suttas, eye-consciousness arises naturally when there is a working eye and \textbf{Visible-form}, something to be seen. Using modern jargon, the function of eye-consciousness is to capture the ``photograph" as it appears at the retina.

Within this eye-consciousness Mind Moment, consciousness is aware of the \textbf{Visible-form}. \textbf{Contact} makes a mental connection to the \textbf{Visible-form}. \textbf{Feeling} experiences the flavour of the \textbf{Visible-form}; as we can see from Figure \ref{SensingZone}, the \textbf{Feeling} is indifferent. \textbf{Perception} marks and recognizes this \textbf{Visible-form}; multiple related photographs may be necessary to support later thinking, so \textbf{Perception} bundles past, present and future photographs into a set. This is not what we normally refer to as memory; \textbf{Perception} is much more superficial. \textbf{Volition} coordinates the activities of consciousness and the associated Mental Factors so they work as a team; the eye-consciousness Mind Moment does not create new kamma, so in this Mind Moment, \textbf{Volition} only coordinates and does not create new kamma. \textbf{One-pointedness} unifies consciousness and the Mental Factors so they are focused. \textbf{Attention} directs the focus provided by \textbf{One-pointedness} to the \textbf{Visible-form}. \textbf{Life faculty} breathes life into consciousness and into the Mental Factors. 

%Though I am presenting them in sequence, consciousness and all the Mental Factors arise together and perform their individual activities at the same time.

\subsubsection*{Occasionals}

\begin{figure} [H]

\setlength{\tabcolsep}{0pt}
\renewcommand{\arraystretch}{1.1}

\begin{tabular}{P{0.25\textwidth} | C{.185\textwidth} C{0.185\textwidth} C{0.185\textwidth} C{0.195\textwidth} m{0mm}}
\toprule
 & \tableheader{Characteristic} & \tableheader{Function} & \tableheader{Manifestation} & \tableheader{Proximate Cause}\\
\midrule
\textbf{Initial application}/\newline Thought\newline \textit{Vitakka} & Directing mind onto object & Striking at object & Leading of the mind to object & Object &\\[12mm]
\textbf{Sustained application}/\newline Examination\newline \textit{Vicāra} & Continued pressure on object & Sustained application\newline on object & Anchoring of Mental Factors\newline on object & Object &\\[12mm]
\textbf{Certainty}/\newline Commitment\newline \textit{Adhimokkha} & Conviction/\newline Being convinced & Not groping & Decisiveness & A thing to be convinced about &\\[12mm]
\textbf{Energy}/\newline Effort/Exertion\newline \textit{Viriya} & Supporting, exerting and marshalling & Supporting Mental Factors & Non-collapse & A sense\newline of urgency &\\[12mm]
\textbf{Zest}/\newline Rapture/Enthusiasm\newline \textit{Pīti} & Endearing & Refreshes\newline body \& mind & Elation & Mind \& body (\textit{nāmarūpa}) &\\[12mm]
\textbf{Wish to do}/\newline Desire/Zeal\newline \textit{Chanda} & Desire to act & Searching\newline for object & Need for object & Object &\\[12mm]
\bottomrule
\end{tabular}

\caption{The six occasional ethically-variable Mental Factors.}

\end{figure}

We now consider the six occasional ethically-variable Mental Factors, starting with \textbf{Initial application} and \textbf{Sustained application}.\footnote{Defined in Visuddhimagga IV.88, explained in Chapter 9 of “Cetasikas” (see Footnote 2 for links).} These two Mental Factors are closely related. When working at a high level, \textbf{Initial application} and \textbf{Sustained application} are sometimes translated as “Thought” and “Examination,” but at a microscopic level, \textbf{Initial application} directs the mind onto the object and \textbf{Sustained application} keeps the mind on the object. As an analogy, when I am washing a plate, one hand holds the plate while the other hand rubs the wash cloth over its surface. The hand holding the plate is \textbf{Initial application} while the hand rubbing with the wash cloth is \textbf{Sustained application}.

The next Mental Factor is \textbf{Certainty}.\footnote{Defined in Visuddhimagga XIV.151, explained in Chapter 10 of “Cetasikas” (see Footnote 2 for links).} \textbf{Certainty} is decisiveness. If one decides to study the Dhamma, there is wholesome \textbf{Certainty}. If one complains, unwholesome \textbf{Certainty} arises which is convinced that the object of \textbf{Aversion} is unacceptable in some way.

\pagebreak

\textbf{Energy} is the next Mental Factor.\footnote{Defined in Visuddhimagga XIV.137, explained in Chapter 10 of “Cetasikas” (see Footnote 2 for links).} The Commentary explains that when rightly initiated, \textbf{Energy} is the root of all attainments, but strong Energy can also increase the weightiness of the kamma generated by unwholesome Mind Moments. Just as a string of a musical instrument needs to be “not too tight and not too loose,” \textbf{Energy} must be exerted at the appropriate level for spiritual development.\footnote{AN 6.55: \url{http://www.accesstoinsight.org/tipitaka/an/an06/an06.055.than.html}}

Now let’s look at \textbf{Zest}.\footnote{Defined in Visuddhimagga IV.94, explained in Chapter 12 of “Cetasikas” (see Footnote 2 for links).} \textbf{Zest} takes an interest in the object and refreshes the mind. \textbf{Zest} and pleasant \textbf{Feeling} are closely related; an exhausted man in a desert experiences \textbf{Zest} when seeing an oasis, he experiences pleasant \textbf{Feeling} when going into the shade and drinking water.

The final ethically-variable Mental Factor is \textbf{Wish to do}.\footnote{Defined in Visuddhimagga XIV.150, explained in Chapter 13 of “Cetasikas” (see Footnote 2 for )links.} Every action begins with a \textbf{Wish to do}. A journey of 1000 miles starts with a single step, and that single step arises because of a \textbf{Wish to do}. The Commentary lists eight things that can be reflected upon to raise one’s sense of spiritual urgency: birth, ageing, sickness, death, suffering in the woeful planes, suffering from past lives, suffering in future lives, suffering in the present life rooted in the search for food.\footnote{See Visuddhimagga IV.63 (see Footnote 2 for link). Also in AN 3.38 (\url{http://www.accesstoinsight.org/tipitaka/an/an03/an03.038.than.html}) the Buddha explained his motivations for renunciation; reflecting that he was subject to ageing, illness and death, he overcame the “intoxication of youth”, the “intoxication of health” and the “intoxication of life”.}

Reminds me of a joke: how many Buddhas does it take to change a light bulb? None. The change must come from within, the Buddhas only show the way to change the light bulb.

Looking at the Mind Moments in the Danger Zone, Mind Moments \textbf{1}--\textbf{12}, we can see that all of the occasional ethically-variable Mental Factors arise in all of the unwholesome Mind Moments with a few exceptions. \textbf{Certainty} does not arise in Mind Moment \textbf{11}, which is associated with \textbf{Doubt}. \textbf{Zest} arises only in Mind Moments with pleasant \textbf{Feeling}. \textbf{Wish to do} arises only in \textbf{Attachment}-rooted or \textbf{Aversion}-rooted Mind Moments.

Looking at the Mind Moments in the Faultless Zone, Mind Moments \textbf{31}--\textbf{38}, we can see that all occasional ethically-variable Mental Factors arise in all wholesome Mind Moments with one exception. \textbf{Zest} arises only in Mind Moments with pleasant \textbf{Feeling}.

Now let’s look at the five jhānas, Mind Moments \textbf{55}--\textbf{59}. The first jhāna has all of the occasional ethically-variable Mental Factors.\footnote{In MN 111 (\url{http://www.accesstoinsight.org/tipitaka/mn/mn.111.than.html}), the Buddha praises Sāriputta because he is able to identify Mental Factors when examining his own mind. The Mental Factors listed in the Sutta includes all of the ethically-variable Mental Factors except \textbf{Life faculty}.} The second jhāna drops the Mental Factor of \textbf{Initial application}. The third jhāna drops the Mental Factor of \textbf{Sustained application}. The fourth jhāna drops the Mental Factor of \textbf{Zest}. The fourth jhāna is the only Mind Moment accompanied by pleasant \textbf{Feeling} in which \textbf{Zest} does not arise. In the fifth jhāna, pleasant \textbf{Feeling} is replaced with indifferent \textbf{Feeling} so there is no change in the number of Mental Factors.

\pagebreak

\subsubsection*{Mental Factors arising in the Sensing Zone}

\begin{figure} [H]

\setlength{\tabcolsep}{0pt}
\renewcommand{\arraystretch}{1.1}

\begin{center}

\noindent\begin{tabular}{P{.05\textwidth}C{.04\textwidth}L{.19\textwidth}C{.04\textwidth}|p{.04\textwidth}p{.04\textwidth}p{.04\textwidth}p{.04\textwidth}p{.04\textwidth}p{.04\textwidth}p{.04\textwidth}}
\toprule
& & & & \tablevsubheaderhacksmall{Universal Ethically-variable} & \tablevsubheaderhacksmall{\textbf{Initial application}} & \tablevsubheaderhacksmall{\textbf{Sustained application}} & \tablevsubheaderhacksmall{\textbf{Certainty}} & \tablevsubheaderhacksmall{\textbf{Energy}} & \tablevsubheaderhacksmall{\textbf{Zest}} & \tablevsubheaderhacksmall{\textbf{Wish to do}} \\
\midrule
\multicolumn{4}{c|}{\tablesubheader{\textbf{Unwholesome Resultant}}} & & & & & & &  \\
\textbf{13} & \multicolumn{2}{l}{Eye-consciousness} & \neutral & \tmsmall & & & & & & \\
\textbf{14} & \multicolumn{2}{l}{Ear-consciousness} & \neutral & \tmsmall & & & & & &  \\
\textbf{15} & \multicolumn{2}{l}{Nose-consciousness} & \neutral & \tmsmall & & & & & &  \\
\textbf{16} & \multicolumn{2}{l}{Tongue-consciousness} & \neutral & \tmsmall & & & & & &  \\
\textbf{17} & \multicolumn{2}{l}{Body-consciousness} & \frowney & \tmsmall & & & & & & \\
\textbf{18} & \multicolumn{2}{l}{Receiving consciousness} & \neutral & \tmsmall & \tmsmall & \tmsmall & \tmsmall & & &  \\
\textbf{19} & \multicolumn{2}{l}{Investigating consciousness} & \neutral & \tmsmall & \tmsmall & \tmsmall & \tmsmall & & &  \\

\midrule

\multicolumn{4}{c|}{\tablesubheader{\textbf{Wholesome Resultant}}} & & & & & & &  \\
\textbf{20} & \multicolumn{2}{l}{Eye-consciousness} & \neutral & \tmsmall & & & & & & \\
\textbf{21} & \multicolumn{2}{l}{Ear-consciousness} & \neutral & \tmsmall & & & & & & \\
\textbf{22} & \multicolumn{2}{l}{Nose-consciousness} & \neutral & \tmsmall & & & & & &  \\
\textbf{23} & \multicolumn{2}{l}{Tongue-consciousness} & \neutral & \tmsmall & & & & & &  \\
\textbf{24} & \multicolumn{2}{l}{Body-consciousness} & \smiley & \tmsmall & & & & & &  \\
\textbf{25} & \multicolumn{2}{l}{Receiving consciousness} & \neutral & \tmsmall & \tmsmall & \tmsmall & \tmsmall & & &  \\
\textbf{26} & \multicolumn{2}{l}{Investigating consciousness} & \smiley & \tmsmall & \tmsmall & \tmsmall & \tmsmall & & \tmsmall &  \\
\textbf{27} & \multicolumn{2}{l}{Investigating consciousness} & \neutral & \tmsmall & \tmsmall & \tmsmall & \tmsmall & & &  \\
\midrule
\multicolumn{4}{c|}{\tablesubheader{\textbf{Unrelated to Kamma}}} & & & & & & &  \\
\textbf{28} & \multicolumn{2}{l}{Five-sense-door adverting} & \neutral & \tmsmall & \tmsmall & \tmsmall & \tmsmall & & & \\
\textbf{29} & \multicolumn{2}{l}{Determining consciousness} & \neutral & \tmsmall & \tmsmall & \tmsmall & \tmsmall & \tmsmall & & \\
\textbf{30} & \multicolumn{2}{l}{Smile producing (Arahat)} & \smiley & \tmsmall & \tmsmall & \tmsmall & \tmsmall & \tmsmall & \tmsmall & \\

\bottomrule
\end{tabular}


\end{center}

\begin{center}
\noindent
\smiley \hspace {2mm} Pleasurable physical \textbf{Feeling} (Mind Moment \textbf{24})\\ \smiley \hspace {2mm} Pleasant mental \textbf{Feeling} (Mind Moment \textbf{26} \& \textbf{30}) \\ \neutral \hspace{2mm} Indifferent \textbf{Feeling} \\ \frowney \hspace{2mm} Painful physical \textbf{Feeling} (Mind Moment \textbf{17})

\end{center}


\caption{Mental Factors arising in Mind Moments \textbf{13} -- \textbf{30}.}

\label{SensingZone}

\end{figure}

\pagebreak
\subsection*{Unwholesome Mental Factors}

\subsubsection*{Universals}

\begin{figure} [H]

\setlength{\tabcolsep}{0pt}
\renewcommand{\arraystretch}{1.1}

\begin{tabular}{P{0.25\textwidth} | C{.185\textwidth} C{0.185\textwidth} C{0.185\textwidth} C{0.195\textwidth} m{0mm}}
\toprule
 & \tableheader{Characteristic} & \tableheader{Function} & \tableheader{Manifestation} & \tableheader{Proximate Cause}\\
\midrule
\textbf{Delusion}/\newline Ignorance\newline \textit{Moha} & Mental blindness, unknowing & Concealment of object’s nature & Absence of right understanding & Unwise attention &\\[12mm]
\textbf{Shamelessness}/\newline Immodesty\newline \textit{Ahirika} & No disgust over misconduct & Doing evil without shame & Not shrinking away from evil & Lack of respect for self &\\[12mm]
\textbf{Recklessness}/\newline Lack of moral dread\newline \textit{Anottappa} & No dread over misconduct & Doing evil without dread & Not shrinking away from evil & Lack of respect for others &\\[12mm]
\textbf{Restlessness}/\newline Distraction/Wavering\newline \textit{Uddhacca} & Excitement/\newline No mindfulness & Make the mind steady & Turmoil/\newline Whirling & Unwise attention &\\[12mm]
\bottomrule
\end{tabular}

\caption{The four universal unwholesome Mental Factors.}

\end{figure}

Let’s move on to the four universal unwholesome Mental Factors of \textbf{Delusion}, \textbf{Shamelessness}, \textbf{Recklessness} and \textbf{Restlessness}. These arise in all unwholesome Mind Moments.

We discussed the Mental Factor of \textbf{Delusion} in the previous lesson; it is considered to be the mother of all that is unwholesome.\footnote{Defined in Visuddhimagga XIV.163, explained in Chapter 15 of “Cetasikas” (see Footnote 2 for links).}

The next Mental Factor is \textbf{Shamelessness}.\footnote{Defined in Visuddhimagga XIV.160, explained in Chapter 15 of “Cetasikas” (see Footnote 2 for links).} Just as a pig is not ashamed to roll in sewage, the mind is not disgusted with unwholesome actions, speech or thought. The Buddha said to his seven-year-old son, “When anyone feels no shame in telling a deliberate lie, there is no evil he will not do. Thus, Rahula, you should train yourself, `I will not tell a deliberate lie even in jest.’”\footnote{MN 61: \url{http://www.accesstoinsight.org/tipitaka/mn/mn.061.than.html\#fnt-2}} In other words, there is no room for “white lies.” To check if there is \textbf{Shamelessness} in the mind, I ask myself, “Is this the kind of Mind Moment that could arise in an Arahat?” or I ask myself, “Would I be proud if my thought were reported in tomorrow’s newspaper?”

\textbf{Recklessness} is the next Mental Factor.\footnote{Defined in Visuddhimagga XIV.160, explained in Chapter 15 of “Cetasikas” (see Footnote 2 for links).} Just as a moth is attracted to fire and is burned, \textbf{Recklessness} is unaware of consequences, is attracted by the unwholesome and plunges into the Danger Zone. To check if there is \textbf{Recklessness} in the mind, I ask myself, “Is this Mind Moment going to be the wind under my wings to lift me up, or a weight around my neck to drag me down?” or I ask myself, “What kind of kamma is this Mind Moment creating?”

We discussed the Mental Factor of \textbf{Restlessness}, in the previous lesson.\footnote{Defined in Visuddhimagga XIV.165, explained in Chapter 15 of “Cetasikas” (see Footnote 2 for links).}

\subsubsection*{Occasionals}

\begin{figure} [H]

\setlength{\tabcolsep}{0pt}
\renewcommand{\arraystretch}{1.1}

\begin{tabular}{P{0.25\textwidth} | C{.185\textwidth} C{0.185\textwidth} C{0.185\textwidth} C{0.195\textwidth} m{0mm}}
\toprule
 & \tableheader{Characteristic} & \tableheader{Function} & \tableheader{Manifestation} & \tableheader{Proximate Cause}\\
\midrule
\textbf{Attachment}/\newline Greed\newline \textit{Lobha} & Grasping an object & Sticking & Not giving up & Seeing enjoyment in what leads to bondage &\\[12mm]
\textbf{Wrong view}/\newline Evil opinion\newline \textit{Diṭṭhi} & Unjustified interpretation & Pre-assume/\newline Misapprehend & Wrong interpretation & Unwillingness to listen to Dhamma &\\[12mm]
\textbf{Conceit}/\newline Pride\newline \textit{Māna} & Haughtiness & Self-praise & Desire to advertise oneself & Greed disassociated from \textbf{Wrong view} &\\[12mm]
\textbf{Aversion}/\newline Anger/Hatred/Fear\newline \textit{Dosa} & Ferocity/\newline Savageness & Burn up its\newline own support (\textbf{Heart-base}) & Persecuting/\newline Injuring/\newline Offending & A ground for annoyance &\\[12mm]
\textbf{Envy}/\newline Jealousy\newline \textit{Issā} & Aversion to other’s success & Dissatisfied with other’s success & Uncomfortable with other’s success & Other’s success &\\[12mm]
\textbf{Selfishness}\newline \textit{Macchariya} & Concealing\newline one’s success & Unwilling to\newline share with others & Shrinking away from sharing & One’s own success &\\[12mm]
\textbf{Remorse}/\newline Worry/Regret\newline \textit{Kukkucca} & Subsequent\newline regret, repentance & Sorrow over what has been done & Remorse/\newline Regret & Past unwholesome kamma &\\[12mm]
\textbf{Sloth}/\newline Sluggishness\newline \textit{Thīna} & Resistance to trying/\newline No striving & Destruction of energy & Sinking of the mind & Unwise attention to drowsiness &\\[12mm]
\textbf{Torpor}/\newline Laziness\newline \textit{Middha} & Unwieldiness & Closing the doors of consciousness & Drooping, nodding \& sleepiness & Unwise attention to drowsiness &\\[12mm]
\textbf{Doubt}\newline \textit{Vicikicchā} & Doubting/\newline Shifting about & Mental wavering & Indecisiveness/\newline Indecision & Unwise attention &\\[12mm]
\bottomrule
\end{tabular}

\caption{The ten occasional unwholesome Mental Factors.}

\end{figure}

\pagebreak

The first two occasional unwholesome Mental Factors are \textbf{Attachment}\footnote{Defined in Visuddhimagga XIV.162, explained in Chapter 16 of “Cetasikas” (see Footnote 2 for links).} and \textbf{Wrong view}.\footnote{Defined in Visuddhimagga XIV.164, explained in Chapter 17 of “Cetasikas” (see Footnote 2 for links).} We discussed these in the previous lesson.

\textbf{Conceit} is the next Mental Factor.\footnote{Defined in Visuddhimagga XIV.168, explained in Chapter 18 of “Cetasikas” (see Footnote 2 for links).} \textbf{Conceit} includes all forms of comparison; “better than,” “equal to” and “inferior to.” Racism, bigotry, prejudice and competitiveness are all forms of \textbf{Conceit}.

We discussed the Mental Factor of \textbf{Aversion}, in the previous lesson.\footnote{Defined in Visuddhimagga XIV.171, explained in Chapter 19 of “Cetasikas” (see Footnote 2 for links).} Aversion arises together with the Mental Factors \textbf{Envy}, \textbf{Selfishness} and \textbf{Remorse}.

\textbf{Envy} is outward looking, it focuses on others.\footnote{Defined in Visuddhimagga XIV.172, explained in Chapter 20 of “Cetasikas” (see Footnote 2 for links).} For example, \textbf{Envy} arises when we are dissatisfied because we feel that somebody’s life is better than ours. The mind can be trained to reduce the negative mental habit of \textbf{Envy} by developing the positive habit of \textbf{Sympathetic joy}.

The next Mental Factor is \textbf{Selfishness}; \textbf{Selfishness} is inward looking, it focuses on ourselves.\footnote{Defined in Visuddhimagga XIV.173, explained in Chapter 20 of “Cetasikas” (see Footnote 2 for links).} When I detect \textbf{Selfishness} arising, I consider that there is nothing I can possess; phenomena that rise and fall away cannot belong to me.

I still remember a line of dialogue from the film Crocodile Dundee, “Aborigines don’t own the land. They belong to it. It’s like their mother. See those rocks? Been standing there for 600 million years. Still be there when you and I are gone. So arguing who owns them is like two fleas arguing over who owns the dog they live on.”\footnote{\url{http://en.wikipedia.org/wiki/Crocodile_Dundee}}

Why are we stingy about what does not belong to us? We cannot take our possessions with us when we die. Life is so short; we waste many opportunities for wholesome actions because of our stinginess. We can reduce the accumulation of \textbf{Selfishness} by practising generosity.

\textbf{Remorse} is the next Mental Factor.\footnote{Defined in Visuddhimagga XIV.174, explained in Chapter 20 of “Cetasikas” (see Footnote 2 for links).} Remorse is the thought, ``I should have done this" or ``I shouldn't have done that." When there is \textbf{Remorse}, there is \textbf{Aversion} toward a past object.\footnote{Iti 30: \url{http://www.accesstoinsight.org/tipitaka/kn/iti/iti.2.028-049.than.html\#iti-030}} The Commentary refers to \textbf{Remorse} as a state of bondage. The proper way of dealing with \textbf{Remorse} is not to dwell on it but rather to acknowledge, forgive yourself and learn. Acknowledge, forgive and learn are the underlying principles of the Vinaya. Repentance is considered a virtue, but \textbf{Remorse} is unwholesome. \textbf{Remorse} that regrets unwholesome deeds and the non-arising of wholesome deeds is different from thinking about the disadvantages of unwholesome deeds and the value of wholesome deeds. The Suttas explain that virtuous behaviour leads to freedom from \textbf{Remorse} and freedom from \textbf{Remorse} can lead to becoming an Arahat.\footnote{AN 11.1: \url{http://www.accesstoinsight.org/tipitaka/an/an11/an11.001.than.html}}

Now let’s look at \textbf{Sloth} and \textbf{Torpor}.\footnote{Defined in Visuddhimagga XIV.167, explained in Chapter 21 of “Cetasikas” (see Footnote 2 for links).} These two Mental Factors always arise together and make the mind unwieldy and lazy. The Commentary refers to \textbf{Sloth} and \textbf{Torpor} as paralysis due to lack of urgency and lack of energy. \textbf{Sloth} is called a sickness of consciousness while \textbf{Torpor} is called a sickness of the other Mental Factors. \textbf{Sloth} and \textbf{Torpor} arise when the Mental Factor of \textbf{Energy} is weak. In the case of the Danger Zone, these Mind Moments are prompted.

In the previous lesson, we discussed the final unwholesome Mental Factor of \textbf{Doubt}.\footnote{Defined in Visuddhimagga XIV.177, explained in Chapter 21 of “Cetasikas” (see Footnote 2 for links).}
\pagebreak

Imagine that the unwholesome Mental Factors were people and they were having a discussion after a retreat. \textbf{Delusion} says, “Things would be better if only...;” \textbf{Shamelessness} says, “I’m not embarrassed if I disturb other yogis;” \textbf{Recklessness} says, “The teacher didn’t catch me when I stayed in bed;” \textbf{Restlessness} says, “When is the next retreat?” \textbf{Attachment} says, “I enjoyed the food at this centre;” \textbf{Wrong view} says, “There is only one way to practise;” \textbf{Conceit} says, “I am the best practitioner of the group;” \textbf{Aversion} says, “I did not like the attitude of the teacher;” \textbf{Envy} says, “I wish I could practise like the other yogi;” \textbf{Selfishness} keeps quiet because he wants to keep what he has learned to himself, does not want to share; \textbf{Remorse} says, “I should have been more diligent;” \textbf{Sloth} says, “I couldn’t be bothered to practise;” \textbf{Torpor} says, “I was too tired to practise” and \textbf{Doubt} says, “Not sure if practise has any benefit.”

\subsubsection*{Overcoming the Hindrances}

The Suttas often list five factors that are hindrances\footnote{\url{http://en.wikipedia.org/wiki/Five_hindrances}} to spiritual progress, and provide an analogy of how conditions make it difficult to see things as they truly are.\footnote{SN 46.55: \url{http://www.accesstoinsight.org/tipitaka/sn/sn46/sn46.055.wlsh.html}} The hindrance of sense-desire is \textbf{Attachment} to the current situation. Sense-desire is like mixing many colours in a bowl of water, making it difficult to see one’s reflection. The hindrance of ill will is \textbf{Aversion}, not accepting the current situation. Ill will is like boiling a bowl of water, making it difficult to see one’s reflection. The hindrance of \textbf{Sloth} and \textbf{Torpor} is sluggishness and laziness. \textbf{Sloth} and \textbf{Torpor} are like the bowl of water being covered in moss, making it difficult to see one’s reflection. The next hindrance is \textbf{Restlessness} and \textbf{Remorse}. \textbf{Restlessness} and \textbf{Remorse} are like the wind agitating the surface of the bowl of water, making it difficult to see one’s reflection. The last hindrance is \textbf{Doubt}. \textbf{Doubt} is like muddy water in the bowl, making it difficult to see one’s reflection.

The Commentary gives a useful summary of practice to overcome the hindrances.\footnote{This list was taken from page 200 of \url{http://www.buddhismuskunde.uni-hamburg.de/pdf/5-personen/analayo/direct-path.pdf}} The one practice that is common to overcoming all five hindrances is to have “good friends and suitable conversation.” In one Sutta, Ānanda said to the Buddha that having good friends was half of the holy life.\footnote{SN 45.2: \url{http://www.accesstoinsight.org/tipitaka/sn/sn45/sn45.002.than.html}; the footnote to this Sutta explains that “good friends” means not only associating with good people, but also learning from them and emulating their good qualities.} The Buddha replied that having good friends was the whole of the holy life; because the Buddha was a good friend, others can gain liberation.

The Commentary suggests that to overcome sensual desire one should learn and practise the 32 parts of the body mediation (as explained in paragraph 14 of the \textit{Satipaṭṭhāna} Sutta), should guard the senses and practise moderation in eating.\footnote{See also Iti 29: \url{http://www.accesstoinsight.org/tipitaka/kn/iti/iti.2.028-049.than.html}} To overcome ill will, one should learn and practise \textit{mettā} meditation, reflect on the kammic consequences of one’s actions, and repeatedly look at things with wise \textbf{Attention} through the lens of the Dhamma. To overcome \textbf{Sloth} and \textbf{Torpor}, one should lessen food intake, occasionally change your meditation posture or stay outdoors. To overcome \textbf{Restlessness} and \textbf{Remorse}, one should study the Suttas and Vinaya, visit experienced elders and ask them questions about the Dhamma. To overcome \textbf{Doubt}, one should study the Suttas and Vinaya, ask questions about the Dhamma and have a strong commitment.

\subsubsection*{Uprooting Unwholesome Mental Factors}

\begin{figure}[H]
\centering
\setlength{\tabcolsep}{0pt}
\renewcommand{\arraystretch}{1.1}

\noindent\begin{tabular}{R{.18\textwidth} |
p{.045\textwidth} 
p{.045\textwidth} 
p{.045\textwidth}
p{.045\textwidth}} 
\toprule
& \tablevsubheader{Sotāpanna}
& \tablevsubheader{Sakadāgāmī}
& \tablevsubheader{Anāgāmī}
& \tablevsubheader{Arahat}
\\
\midrule
\hpadright{\textbf{Delusion}} & & & & \tm \\
\hpadright{\textbf{Shamelessness}} & & & & \tm \\
\hpadright{\textbf{Recklessness}} & & & & \tm \\
\hpadright{\textbf{Restlessness}} & & & & \tm \\
\hpadright{\textbf{Attachment}} & & & \lc & \tm \\
\hpadright{\textbf{Wrong view}} & \tm & & & \\
\hpadright{\textbf{Conceit}} & & & & \tm \\
\hpadright{\textbf{Aversion}} & & & \tm & \\
\hpadright{\textbf{Envy}} & \tm & & & \\
\hpadright{\textbf{Selfishness}} & \tm & & & \\
\hpadright{\textbf{Remorse}} & & & \tm & \\
\hpadright{\textbf{Sloth}} & & & & \tm \\
\hpadright{\textbf{Torpor}} & & & & \tm \\
\hpadright{\textbf{Doubt}} & \tm & & & \\

\bottomrule
\end{tabular}
\begin{center}
\lc \hspace{2mm} Partially uprooted \hspace{5mm} \tm\hspace{2mm} Fully uprooted
\end{center}
\caption{The stages at which the saint becomes free from the 14 unwholesome Mental Factors. The Anāgāmī is free from \textbf{Attachment} to sense objects and an Arahat is free from all \textbf{Attachment}, including craving for fine-material existence and craving for immaterial existence.}
\end{figure}

\subsubsection*{Mental Factors arising in the Danger Zone}

\begin{figure}[H]
\setlength{\tabcolsep}{0pt}
\renewcommand{\arraystretch}{1.1}

\noindent\begin{tabular}{P{.05\textwidth}C{.04\textwidth}L{.19\textwidth}C{.04\textwidth}|p{.04\textwidth}p{.04\textwidth}p{.04\textwidth}p{.04\textwidth}p{.04\textwidth}p{.04\textwidth}p{.04\textwidth}|p{.04\textwidth}p{.04\textwidth}p{.04\textwidth}p{.04\textwidth}p{.04\textwidth}p{.04\textwidth}p{.04\textwidth}p{.04\textwidth}p{.04\textwidth}p{.04\textwidth}}
\toprule
& & & & \tablevsubheaderhacksmall{Universal Ethically-variable} & \tablevsubheaderhacksmall{\textbf{Initial application}} & \tablevsubheaderhacksmall{\textbf{Sustained application}} & \tablevsubheaderhacksmall{\textbf{Certainty}} & \tablevsubheaderhacksmall{\textbf{Energy}} & \tablevsubheaderhacksmall{\textbf{Zest}} & \tablevsubheaderhacksmall{\textbf{Wish to do}} & \tablevsubheaderhacksmall{Universal unwholesome} & \tablevsubheaderhacksmall{\textbf{Attachment}} & \tablevsubheaderhacksmall{\textbf{Wrong view}} & \tablevsubheaderhacksmall{\textbf{Conceit}} & \tablevsubheaderhacksmall{\textbf{Aversion}} & \tablevsubheaderhacksmall{\textbf{Envy}} & \tablevsubheaderhacksmall{\textbf{Selfishness}} & \tablevsubheaderhacksmall{\textbf{Remorse}} & \tablevsubheaderhacksmall{\textbf{Sloth} \& \textbf{Torpor}} & \tablevsubheaderhacksmall{\textbf{Doubt}}\\
\midrule
\multicolumn{4}{c|}{\tablesubheader{\textbf{Attachment}-rooted}} & & & & & & & & & & & & & & & & \\
\textbf{1} & U & \textbf{Wrong view} & \smiley & \tmsmall & \tmsmall & \tmsmall & \tmsmall & \tmsmall & \tmsmall & \tmsmall & \tmsmall & \tmsmall & \tmsmall & & & & & & & \\
\textbf{2} & P & \textbf{Wrong view} & \smiley & \tmsmall & \tmsmall & \tmsmall & \tmsmall & \tmsmall & \tmsmall & \tmsmall & \tmsmall & \tmsmall & \tmsmall & & & & & & \tmsmall & \\
\textbf{3} & U & No \textbf{Wrong view} & \smiley & \tmsmall & \tmsmall & \tmsmall & \tmsmall & \tmsmall & \tmsmall & \tmsmall & \tmsmall & \tmsmall & & \lcsmall & & & & & & \\
\textbf{4} & P & No \textbf{Wrong view} & \smiley & \tmsmall & \tmsmall & \tmsmall & \tmsmall & \tmsmall & \tmsmall & \tmsmall & \tmsmall & \tmsmall & & \lcsmall & & & & & \tmsmall & \\
\textbf{5} & U & \textbf{Wrong view} & \neutral & \tmsmall & \tmsmall & \tmsmall & \tmsmall & \tmsmall & & \tmsmall & \tmsmall & \tmsmall & \tmsmall & & & & & & & \\
\textbf{6} & P & \textbf{Wrong view} & \neutral & \tmsmall & \tmsmall & \tmsmall & \tmsmall & \tmsmall & & \tmsmall & \tmsmall & \tmsmall & \tmsmall & & & & & & \tmsmall & \\
\textbf{7} & U & No \textbf{Wrong view} & \neutral & \tmsmall & \tmsmall & \tmsmall & \tmsmall & \tmsmall & & \tmsmall & \tmsmall & \tmsmall & & \lcsmall & & & & & & \\
\textbf{8} & P & No \textbf{Wrong view} & \neutral & \tmsmall & \tmsmall & \tmsmall & \tmsmall & \tmsmall & & \tmsmall & \tmsmall & \tmsmall & & \lcsmall & & & & & \tmsmall & \\
\multicolumn{4}{c|}{\tablesubheader{\textbf{Aversion}-rooted}} & & & & & & & & & & & & & & & & & \\
\textbf{9} & U & Ill will & \frowney & \tmsmall & \tmsmall & \tmsmall & \tmsmall & \tmsmall & & \tmsmall & \tmsmall & & & & \tmsmall & \lcsmall & \lcsmall & \lcsmall & & \\
\textbf{10} & P & Ill will & \frowney & \tmsmall & \tmsmall & \tmsmall & \tmsmall & \tmsmall & & \tmsmall & \tmsmall & & & & \tmsmall & \lcsmall & \lcsmall & \lcsmall & \tmsmall & \\
\multicolumn{4}{c|}{\tablesubheader{\textbf{Delusion}-rooted}} & & & & & & & & & & & & & & & & & \\
\textbf{11} & \multicolumn{2}{l}{With \textbf{Doubt}} & \neutral & \tmsmall & \tmsmall & \tmsmall & & \tmsmall & & & \tmsmall & & & & & & & & & \tmsmall \\
\textbf{12} & \multicolumn{2}{l}{With \textbf{Restlessness}} & \neutral & \tmsmall & \tmsmall & \tmsmall & \tmsmall & \tmsmall & & & \tmsmall & & & & & & & & & \\

\bottomrule
\end{tabular}

\begin{center}
\noindent
U \hspace{2mm} Unprompted\hspace{5mm} P \hspace{2mm} Prompted

\smiley \hspace {2mm} Pleasant \textbf{Feeling} \hspace{5mm} \neutral \hspace{2mm} Indifferent \textbf{Feeling} \hspace{5mm} \frowney \hspace{2mm} Unpleasant \textbf{Feeling}

\tmsmall \hspace{2mm} Always in Mind Moment\hspace{5mm} \lcsmall \hspace{2mm} Sometimes in Mind Moment

\end{center}

\caption{Mental Factors arising in Mind Moments \textbf{1}--\textbf{12}.}
\label{fig:Unwholesome}
\end{figure}

The unwholesome Mental Factors arise only in the Danger Zone, Mind Moments \textbf{1}--\textbf{12}.

\textbf{Attachment} arises only in the first eight Mind Moments. \textbf{Wrong view} arises in Mind Moments \textbf{1}, \textbf{2}, \textbf{5} and \textbf{6}; this can also be seen in Figure \ref{fig:Danger}. \textbf{Wrong view} and \textbf{Conceit} cannot arise in the same Mind Moment because their way of taking the object is different. 

\textbf{Wrong view} is convinced that an object has qualities of permanence, beauty or Self, when in fact, the object has the opposite qualities of impermanence, not-to-be-clung-to and non-self. 

\textbf{Conceit} is an activity of comparison of beings, this being is better or worse than, or equal to another being. So, depending on the object, Mind Moments \textbf{3}, \textbf{4}, \textbf{7} and \textbf{8} can arise with or without \textbf{Conceit}. 

\pagebreak

\textbf{Envy}, \textbf{Selfishness} and \textbf{Remorse} always arise together with \textbf{Aversion} because none of them accept the object as it is. \textbf{Envy}, \textbf{Selfishness} and \textbf{Remorse} each take different types of objects, so they cannot arise together. \textbf{Sloth} and \textbf{Torpor} arise in prompted Mind Moments, Mind Moments \textbf{2}, \textbf{4}, \textbf{6}, \textbf{8} and \textbf{10}. \textbf{Doubt} arises only in one Mind Moment, Mind Moment \textbf{11}.

\subsubsection*{Groupings of Unwholesome Mental Factors in the Suttas}

\begin{figure}[H]
\setlength{\tabcolsep}{0pt}
\renewcommand{\arraystretch}{1.1}

\noindent\begin{tabular} {L{.426\textwidth} |
p{.041\textwidth} 
p{.041\textwidth} 
p{.041\textwidth}
p{.041\textwidth} 
p{.041\textwidth} 
p{.041\textwidth}
p{.041\textwidth} 
p{.041\textwidth} 
p{.041\textwidth}
p{.041\textwidth} 
p{.041\textwidth} 
p{.041\textwidth} 
p{.041\textwidth}
p{.041\textwidth}} 
\toprule

& \tablevsubheader{\textbf{Delusion}}
& \tablevsubheader{\textbf{Shamelessness}}
& \tablevsubheader{\textbf{Recklessness}}
& \tablevsubheader{\textbf{Restlessness}}
& \tablevsubheader{\textbf{Attachment}}
& \tablevsubheader{\textbf{Wrong view}}
& \tablevsubheader{\textbf{Conceit}}
& \tablevsubheader{\textbf{Aversion}}
& \tablevsubheader{\textbf{Envy}}
& \tablevsubheader{\textbf{Selfishness}}
& \tablevsubheader{\textbf{Remorse}}
& \tablevsubheader{\textbf{Sloth}}
& \tablevsubheader{\textbf{Torpor}}
& \tablevsubheader{\textbf{Doubt}}
\\
\midrule
\tablesubheader{Taints}: Oozing pus/intoxicants flowing up to the topmost plane of existence. & \tm & & & & \tm & \tm & & & & & & & & \\
\tablesubheader{Floods}: Sweep away beings into the ocean of existence; hard to cross. & \tm & & & & \tm & \tm & & & & & & & & \\
\tablesubheader{Bonds}: Yoke beings to suffering; they do not allow them to escape. & \tm & & & & \tm & \tm & & & & & & & & \\
\tablesubheader{Knots}: Tie the mind to the body; they tie the current body to bodies in past/future existences. & & & & & \tm & \tm & \tm & & & & & & & \\
\tablesubheader{Clingings}: Clinging to sense pleasures, wrong views, rites and rituals, doctrine of self. & & & & & \tm & \tm & & & & & & & & \\
\tablesubheader{Hindrances}: Obstruct the way to a heavenly rebirth \& the way to \textit{Nibbāna}. & \tm & & & \tm & \tm & & & \tm & & & \tm & \tm & \tm & \tm \\
\tablesubheader{Latent Dispositions}: Lie with mental processes, rising up as obsessions\newline when they meet suitable conditions. & \tm & & & & \tm & \tm & \tm & \tm & & & & & & \tm \\
\tablesubheader{Fetters}: Bind beings to the round of existence. & \tm & & & \tm & \tm & \tm & \tm & \tm & \tm & \tm & & & & \tm \\
\tablesubheader{Defilements}: Afflict and torment the mind; drag beings down to a mentally defiled condition. & \tm & \tm & \tm & \tm & \tm & \tm & \tm & \tm & & & & \tm & & \tm \\
\bottomrule
\end{tabular}

\begin{center}
\tm\hspace{2mm} Unwholesome Mental Factor included in grouping from the Suttas
\end{center}

\caption{Correlation between groupings found in the Suttas and the Unwholesome Mental Factors.}
\end{figure}

\subsection*{Beautiful Mental Factors}

\subsubsection*{Universals}

\begin{figure} [H]

\setlength{\tabcolsep}{0pt}
\renewcommand{\arraystretch}{1.0}

\begin{tabular}{P{0.25\textwidth} | C{.185\textwidth} C{0.185\textwidth} C{0.185\textwidth} C{0.195\textwidth} m{0mm}}
\toprule
 & \tableheader{Characteristic} & \tableheader{Function} & \tableheader{Manifestation} & \tableheader{Proximate Cause}\\
\midrule
\textbf{Faith}/\newline Confidence\newline \textit{Saddhā} & Placing faith/\newline Aspiring & Clarifying/\newline Purifying & Non-fogginess/\newline Lack of pollution & A worthy object &\\[12mm]
\textbf{Mindfulness}/\newline Attentiveness\newline \textit{Sati} & Not floating\newline away from\newline object & Non-forgetfulness/\newline Non-confusion & Being “face to face” with object & Firm remembrance/\newline Four foundations &\\[12mm]
\textbf{Conscience}/\newline Shame/Scruples\newline \textit{Hiri} & Disgust at misconduct & Not doing evil because of modesty & Shrinking away from evil & Self-respect &\\[12mm]
\textbf{Fear of blame}/\newline Moral dread\newline \textit{Ottappa} & Dread of evil & Not doing evil because of dread & Shrinking away from evil & Respect\newline for others&\\[12mm]
\textbf{Non-attachment}/\newline Non-greed\newline \textit{Alobha} & No attachment\newline to object & Not appropriating & Detachment & Wise attention &\\[12mm]
\textbf{Non-aversion}/\newline Non-anger\newline \textit{Adosa} & Not opposing & Removing annoyance & Being pleasing/\newline Agreeableness & Wise attention &\\[12mm]
\textbf{Equanimity}/\newline Mental balance\newline \textit{Tatramajjhattatā} & Promoting neutrality\newline toward beings & Inhibiting partiality/\newline Seeing equality & No approval or resentment & Wise attention &\\[12mm]
2 x \textbf{Tranquillity}\newline \textit{Passaddhi} & Quietening mental disturbances & Crushing mental disturbances & Neutrality/\newline Peacefulness & Consciousness/\newline Mental Factors &\\[12mm]
2 x \textbf{Agility}/\newline Lightness/Buoyancy\newline \textit{Lahutā} & Opposing mental heaviness & Crushing mental heaviness & Oppose\newline \textbf{Sloth} \& \textbf{Torpor} & Consciousness/\newline Mental Factors &\\[12mm]
2 x \textbf{Pliancy}/\newline Elasticity/Malleability\newline \textit{Mudutā} & Opposing mental rigidity & Crushing mental rigidity & Oppose\newline \textbf{Wrong view}\newline \& \textbf{Conceit} & Consciousness/\newline Mental Factors &\\[12mm]
2 x \textbf{Adaptability}/\newline Workableness\newline \textit{Kammaññatā} & Opposing mental unwieldiness & Crushing mental unwieldiness & Oppose\newline Sense desire\newline \& \textbf{Aversion} & Consciousness/\newline Mental Factors &\\[12mm]
2 x \textbf{Proficiency}/\newline Skill\newline \textit{Pāguññatā} & Healthiness/\newline Fitness/\newline Competence & Crushing mental unhealthiness & Oppose lack\newline of \textbf{Faith}\newline (no disability) & Consciousness/\newline Mental Factors &\\[12mm]
2 x \textbf{Uprightness}/\newline Rectitude\newline \textit{Ujjukatā} & Mental uprightness & Crushing mental crookedness & Oppose hypocrisy \& fraudulence & Consciousness/\newline Mental Factors &\\[12mm]
\bottomrule
\end{tabular} 

\caption{The 19 universal beautiful Mental Factors.}

\end{figure}

\pagebreak

\begin{figure}[H]
\centering
\input{./Diagrams/Key.pdf_tex}
\caption{“Manussa” is the Pāḷi word for human. “Manussa” is derived from the words for “abundance” and “mind.”}
\label{fig:Key}
\end{figure}

The first beautiful Mental Factor is \textbf{Faith}.\footnote{Defined in Visuddhimagga XIV.140, explained in Chapter 26 of “Cetasikas” (see Footnote 2 for links).} The Commentary regards \textbf{Faith} as the leader of all beautiful Mental Factors.\footnote{AN 5.38: \url{http://www.accesstoinsight.org/tipitaka/an/an05/an05.038.than.html}} When you have confidence in the value of \textit{dāna} (generosity), \textit{sīla} (discipline), or \textit{bhāvanā} (mental cultivation), you will apply yourself to it. When Buddhists take refuge in the Triple Gem, their \textbf{Faith} should be reasoned and rooted in \textbf{Understanding}. A Buddhist’s \textbf{Faith} is not in conflict with the spirit of enquiry. The Suttas discourage blind \textbf{Faith} and encourage reasoned confidence. People who have negative associations with the word “\textbf{Faith}” can substitute it with “confidence” when reading translations of Buddhist Suttas.

Shortly before his \textit{parinibbāna}, the Buddha said there were four places that a Buddhist should visit with \textbf{Faith} to increase their sense of spiritual urgency: the places of the Buddha’s birth, enlightenment, first sermon and \textit{parinibbāna}.\footnote{DN 16: \url{http://www.accesstoinsight.org/tipitaka/dn/dn.16.5-6.than.html\#pilgrim}} I have visited these pilgrimage sites and they inspired deep reverence in me.\footnote{The leader of my tour group has published a book about Buddhist pilgrimages: \url{http://www.urbandharma.org/udharma14/pilgrim.html}}

The main benefit that I get from studying the Abhidhamma is a better understanding of the Suttas. For me, the Abhidhamma clarifies the meaning of the Suttas and this clarity instills strong confidence and \textbf{Faith} in me. This strong confidence and \textbf{Faith} motivates me to practise.

\begin{figure}[H]
\centering
\input{./Diagrams/Fortress.pdf_tex}
\begin{quote}
“Suppose, monk, that there were a fortress with strong walls (the body) and six gates (senses). In it would be a wise, experienced, intelligent gatekeeper (\textbf{Mindfulness}) to keep out those he didn't know and to let in those he did. A swift pair of messengers (tranquillity/\textit{samatha} \& insight/\textit{vipassanā}), would say to the gatekeeper, ‘Where, my good man, is the commander (consciousness) of this fortress?’ He would say, ‘There he is, sitting in the central square.’ The swift pair of messengers, delivering their accurate report (\textit{Nibbāna}) to the commander of the fortress, would then go back by the route by which they had come (Noble Eightfold Path).”
\end{quote}
\caption{In this Sutta (SN 35.204: \url{http://www.accesstoinsight.org/tipitaka/sn/sn35/sn35.204.than.html\#fort}), the Buddha describes \textbf{Mindfulness} as a “wise, experienced, intelligent gatekeeper who keeps out those he doesn’t know and lets in those he does.”}
\label{fig:Fortress}
\end{figure}

\pagebreak

The next Mental Factor is \textbf{Mindfulness}.\footnote{Defined in Visuddhimagga XIV.141, explained in Chapter 27 of “Cetasikas” (see Footnote 2 for links).} A modern Buddhist teacher describes \textbf{Mindfulness} in many ways:\footnote{\url{http://en.wikipedia.org/wiki/Henepola_Gunaratana} \\ \url{http://www.urbandharma.org/udharma4/mpe13.html}} Mirror-thought accurately reflects what is presently happening;\footnote{Remember the cat photos (Figure \ref{fig:Cats}) illustrating \textbf{Delusion} and \textbf{Wrong view}?} there are no biases. Impartial watchfulness doesn’t take sides; no clinging to pleasant or fleeing from unpleasant. Non-conceptual awareness pays bare attention; it registers experiences, but does not compare. Present-time awareness stays forever in the present moment. Non-egoistic alertness has no reference to Self. Goalless awareness does not try to accomplish anything. Awareness of change watches the flow of the show. Non-judgemental observation observes things in a natural state; no criticism or judgement. This is \textbf{\textit{A}}ccept in the \textbf{\textit{RADICAL}} acronym.

Mindfulness is more than just being in the present moment. As an analogy, when driving on an open highway, the mind may be in the present moment (not thinking about the past or the future) but there is no mindfulness because there is superficiality; the mind is not face to face with the object. On the other hand, mindfulness is like when you are trying to park your car in a tight spot; the mind does not float away from the object and pays close attention. Being in the present moment without mindfulness is superficial, like watching something on TV, whereas being in the present moment with mindfulness is like experiencing something in real life; experiencing it with the heart.

Consciousness is the simple awareness of what is happening. \textbf{Mindfulness} is the breakthrough observing power that comes face to face with the object. For example, when we are watching a movie in a theatre, there is consciousness or awareness of what is happening in the movie. \textbf{Mindfulness} is the remembering of “I am watching a movie in a theatre.” 

Our six senses usually lead us around as if we were on a leash. \textbf{Mindfulness} wakes us up to what is really happening. \textbf{Understanding} uses \textbf{Mindfulness} as a platform and builds on \textbf{Mindfulness} through investigation. \textbf{Understanding} learns through investigation. \textbf{Understanding} can ask, “Is this wholesome or unwholesome, necessary or unnecessary?” \textbf{Understanding} can investigate what we are being mindful of, and sees that it is all just nature unfolding.

\textbf{Conscience} and \textbf{Fear of blame}\footnote{Defined in Visuddhimagga XIV.142, explained in Chapter 28 of “Cetasikas” (see Footnote 2 for links).} are the next Mental Factors. Their opposites, \textbf{Shamelessness} and \textbf{Recklessness}, were mentioned earlier as universal unwholesome Mental Factors. The Buddha referred to \textbf{Conscience} and \textbf{Fear of blame} as the bright guardians of the world because they protect society from degrading to a level of animals.\footnote{AN 2.9: \url{http://www.accesstoinsight.org/tipitaka/an/an02/an02.009.than.html}} In other words, it is not laws that protect society, it is each individual’s internal moral compass and sense of responsibility.

\begin{figure}[H]
\centering
\includegraphics[width=0.3\linewidth]{./Diagrams/AngelDevil}
\caption{\textbf{Conscience} is the angel on your shoulder and \textbf{Shamelessness} is the devil.}
\label{fig:AngelDevil}
\end{figure}

In cartoons, \textbf{Conscience} is often represented as a small angel sitting on one shoulder whispering into one ear, while bad habits are shown as a small devil sitting on the other shoulder whispering into the other ear. Constantly feeding the bad habits while ignoring \textbf{Conscience} leads to an undernourished \textbf{Conscience}. The effects of an undernourished \textbf{Conscience} are \textbf{Remorse}, reduced self-image and sometimes depression (all of these are types of \textbf{Aversion}). 

Reminds me of a joke: a clear conscience is usually the sign of a bad memory.

Now let’s look at \textbf{Non-attachment}.\footnote{Defined in Visuddhimagga XIV.143, explained in Chapter 29 of “Cetasikas” (see Footnote 2 for links).} \textbf{Non-attachment} is not merely the absence of \textbf{Attachment}, it counteracts \textbf{Attachment} and is the foundation of the acts of generosity, detachment and renunciation. Planning the act of giving, performing the act of giving and recalling the act of giving are all wholesome Mind Moments with \textbf{Non-attachment} toward the object that is given. The giving of the Dhamma is superior to giving material things.\footnote{Iti 3.98: \url{http://www.accesstoinsight.org/tipitaka/kn/iti/iti.3.050-099.than.html}; also Dhammapada 354, “The gift of Dhamma excels all other gifts” (\textit{sabba dānam dhamma dānam jināti}).} Some believe that renunciation requires one to become a monk or a nun but this is not true. Renunciation arises when you withdraw from sense pleasures, motivated by a sense of spiritual urgency. For example, at this very moment you could be gratifying your senses, indulging in sense pleasures, but you chose to study Abhidhamma. This is a form of renunciation.

In Mind Moments \textbf{33}, \textbf{34}, \textbf{37} and \textbf{38} (Mind Moments without \textbf{Understanding}) there is \textbf{Non-attachment} to the current object; for example, when offering dāna, there is \textbf{Non-attachment} to what is being given. In Mind Moments with \textbf{Understanding}, Mind Moments \textbf{31}, \textbf{32}, \textbf{35} and \textbf{36}, there is \textbf{Non-attachment} to the current object, together with an appreciation of the value of wholesome Mind Moments or a realization of non-self.

The next Mental Factor, \textbf{Non-aversion}, is the foundation for patience and loving-kindness.\footnote{Defined in Visuddhimagga XIV.143, explained in Chapter 30 of “Cetasikas” (see Footnote 2 for links).} \textbf{Non-aversion} counteracts \textbf{Aversion}. Again, this is not merely the absence of \textbf{Aversion}, it is the active opposite of \textbf{Aversion}. For example, when I am enjoying my coffee, there is \textbf{Attachment} and there is an absence of \textbf{Aversion}, but there is no patience or loving-kindness.

Patience means acceptance; it allows us to endure and tolerate both the undesirable and the desirable. We tolerate the desirable by not clinging to it.

It is valuable to instill a habit of loving-kindness, \textit{mettā}. This can be done by silently repeating a phrase such as “May all beings be well and happy.” Sometimes this is combined with \textbf{Compassion}, “May all beings be free from suffering, may they be well and happy.” When cultivating loving-kindness, we need to be careful to ask ourselves, “Do we want to be kind only to those we like, or can we be kind to whomever we meet because we are truly concerned for their welfare?”

I once attended a Dhamma talk on \textit{mettā} given by a monk. During the Q \& A session a lady asked, “One of my co-workers is nasty. I have been radiating \textit{mettā} to her for a month and she still hasn’t changed. What to do now?” The monk smiled and said, “\textit{Mettā} is supposed to change you, not change the other person!”

The next Mental Factor is \textbf{Equanimity}.\footnote{Defined in Visuddhimagga XIV.153, explained in Chapter 31 of “Cetasikas” (see Footnote 2 for links).} \textbf{Equanimity} is even-mindedness and stability that does not give a foothold to Mental Factors such as \textbf{Attachment} or \textbf{Aversion}. \textbf{Equanimity} can arise when reflecting that all things arise based on conditions, and that beings inherit their kamma. It is valuable to cultivate a habit of \textbf{Equanimity} by silently repeating a phrase such as, “May we all accept things as they are.”

\pagebreak

We will treat the final 12 universal beautiful Mental Factors as a set. Included in this set are six pairs of Mental Factors; one of the pair applies to consciousness and the other applies to the remaining Mental Factors. For example, the first pair is “\textbf{Tranquillity} of consciousness” and “\textbf{Tranquillity} of Mental Factors.”

Imagine two people who have just finished listening to a Dhamma talk by Ajahn Brahm. To provide some context, in his talks, Ajahn Brahm supplements the Dhamma with humorous stories (a spoonful of sugar helps the medicine go down).

The Mind Moment of person A is in the Faultless Zone while the Mind Moment of person B is in the Danger Zone. Both people experienced pleasant \textbf{Feeling}, but person A listened with joy while person B enjoyed listening. Please take a moment to  consider the differences in experience between “listening with joy” and “enjoying listening.”

Person A experiences \textbf{Tranquillity}.\footnote{Defined in Visuddhimagga XIV.144, explained in Chapter 32 of “Cetasikas” (see Footnote 2 for links).} He is filled with calmness and warmth from being in the presence of something beautiful. He listens patiently to the Dhamma so that he will have more \textbf{Understanding}. Person B does not have \textbf{Tranquillity}. He is restless as his mind jumps between the amusing stories in the talk.

Person A experiences \textbf{Agility}.\footnote{Defined in Visuddhimagga XIV.145, explained in Chapter 32 of “Cetasikas” (see Footnote 2 for links).} He is inspired to take positive action. His mind is light and nimble, ready to quickly seize an opportunity for wholesome actions. Person B does not have \textbf{Agility}. This creates conditions for sluggishness and laziness, \textbf{Sloth} and \textbf{Torpor}, because the talk is finished and “the show is over.”

Person A experiences \textbf{Pliancy}.\footnote{Defined in Visuddhimagga XIV.146, explained in Chapter 32 of “Cetasikas” (see Footnote 2 for links).} He focuses on the application of the Dhamma as his mind naturally spreads the Dhamma learned to various aspects of his life. Person B does not have \textbf{Pliancy}. He focuses on his own enjoyment of the experience. His mind is rigid and his focus is on himself and his own viewpoints, not on the Dhamma.

Person A experiences \textbf{Adaptability}.\footnote{Defined in Visuddhimagga XIV.147, explained in Chapter 32 of “Cetasikas” (see Footnote 2 for links).} His mind is workable and flexible with just the right amount of \textbf{Pliancy}. Too much \textbf{Pliancy} makes the mind easily influenced so the Dhamma is overwritten by the next thought. Too little \textbf{Pliancy} makes the mind stubborn so that the Dhamma is ignored. Person B does not have \textbf{Adaptability}. Though he enjoyed the talk, he judges that there were both fun parts and boring parts. He likes the fun parts and dislikes the boring parts thereby making his mind less workable.

Person A experiences \textbf{Proficiency}.\footnote{Defined in Visuddhimagga XIV.148, explained in Chapter 32 of “Cetasikas” (see Footnote 2 for links).} His mind is healthy, fit and competent. It is ready to apply the Dhamma from the talk faultlessly and spontaneously into his life. Person B does not have \textbf{Proficiency}. His mind is sickly, lacking confidence to push himself to integrate the Dhamma from the talk into his own life.

Person A experiences \textbf{Uprightness}.\footnote{Defined in Visuddhimagga XIV.149, explained in Chapter 32 of “Cetasikas” (see Footnote 2 for links).} His mind is sincere. He wants to use the Dhamma to improve his spiritual development. Person B does not have \textbf{Uprightness}. He has a crafty mind, thinking about how to repeat the content of the talk to make himself look good.

To summarize, Person A listens with joy and his mind is calm, inspired, elastic, flexible, healthy and sincere. Person B enjoys listening but his mind is agitated, sluggish, rigid, judgemental, sickly and crafty. Our mind changes very rapidly; at one moment, our mind can be wholesome like Person A and the next instant, it can be unwholesome like Person B.

\subsubsection*{Occasionals}

\begin{figure} [H]

\setlength{\tabcolsep}{0pt}
\renewcommand{\arraystretch}{1.1}

\begin{tabular}{P{0.25\textwidth} | C{.185\textwidth} C{0.185\textwidth} C{0.185\textwidth} C{0.195\textwidth} m{0mm}}
\toprule
 & \tableheader{Characteristic} & \tableheader{Function} & \tableheader{Manifestation} & \tableheader{Proximate Cause}\\
\midrule
\textbf{Understanding}/\newline Wisdom\newline \textit{Paññā} & Penetrating intrinsic nature\newline of object & Illuminate the object & Non-bewilderment & Wise attention &\\[12mm]
\textbf{Compassion}\newline \textit{Karuṇā} & Promoting removal of\newline other’s suffering & Unable to bear other’s suffering & Non-cruelty & Seeing helplessness &\\[12mm]
\textbf{Sympathetic joy}/\newline Altruistic joy\newline \textit{Muditā} & Gladness at the success of others & Being unenvious at other’s success & Elimination of aversion & Seeing the success of others &\\[12mm]
\textbf{Abstinence from\newline wrong speech}\newline \textit{Vaci-duccarita virati} & Non-transgression by wrong speech & Shrink back\newline from evil deeds & Abstinence\newline from evil deeds & \textbf{Faith}, shame and fewness of wishes &\\[12mm]
\textbf{Abstinence from\newline wrong action}\newline \textit{Kāya-duccarita virati} & Non-transgression by wrong action & Shrink back\newline from evil deeds & Abstinence\newline from evil deeds & \textbf{Faith}, shame and fewness of wishes &\\[12mm]
\textbf{Abstinence from\newline wrong livelihood}\newline \textit{Ājīva-duccarita virati} & Non-transgression by wrong livelihood & Shrink back\newline from evil deeds & Abstinence\newline from evil deeds & \textbf{Faith}, shame and fewness of wishes &\\[12mm]
\bottomrule
\end{tabular}

\caption{The six occasional beautiful Mental Factors.}

\end{figure}

The first occasional beautiful Mental Factor is \textbf{Understanding} or wisdom.\footnote{Defined in Visuddhimagga XIV.143, explained in Chapter 35 of “Cetasikas” (see Footnote 2 for links).} We discussed \textbf{Understanding} in some detail in the previous lesson so we will not review it here.

\textbf{Compassion} and \textbf{Sympathetic joy}\footnote{Defined in Visuddhimagga XIV.154, explained in Chapter 34 of “Cetasikas” (see Footnote 2 for links).} can arise when there is a suitable object.

\textbf{Compassion} arises when the mind cannot bear others’ suffering and wants to remove that suffering. It is valuable to instill a habit of \textbf{Compassion} by silently repeating a phrase such as “May all beings be free from suffering.” This may sound like loving-kindness, but the motives of loving-kindness and \textbf{Compassion} are different.

When we visit a sick person, we may offer them flowers and wish them well; these are moments of loving-kindness. When we notice their suffering, moments of \textbf{Compassion} may arise. Pity and \textbf{Aversion} sometimes displace \textbf{Compassion} without our noticing. When visiting a sick person we may have moments of \textbf{Compassion} where we wish that the person’s suffering be reduced. The next moment, we may be thinking of the person’s sickness with \textbf{Aversion} or fear. The following moment, we may be thinking with \textbf{Aversion} about the injustice of the situation. After that, we may have pity toward the person. True \textbf{Compassion} does not involve \textbf{Aversion}, fear or pity.

\pagebreak

When I am being scolded, I try not to allow \textbf{Aversion} to cloud my mind. I first consider how I can improve myself and also feel \textbf{Compassion} for the person doing the scolding; it is obvious that they are experiencing a painful \textbf{Feeling}.

The Mahāyāna tradition places a lot of emphasis on \textbf{Compassion}. Here is a story from that tradition. Two monks were washing their bowls in the river when they noticed a scorpion that was drowning. One monk immediately scooped it up and set it upon the bank; in the process he was stung. He went back to washing his bowl and again the scorpion fell in, the monk saved the scorpion and was again stung. The second monk asked him, “Friend, why do you continue to save the scorpion when you know its nature is to sting?” “Because,” the first monk replied, “to save is my nature.” I have heard variations on this story, but my favourite is when the second monk takes a leaf from the ground and uses this leaf to scoop the scorpion out of the water. He showed the first monk how to save the scorpion without being stung.

%I share the Abhidhamma because I believe that, if properly understood, the Abhidhamma can be useful to support one’s spiritual development. So it is out of \textbf{Compassion} for you, the listener, that I share what I hope will be helpful to you.

\textbf{Sympathetic joy} arises when others have good fortune. When this happens, we can silently and sincerely repeat a phrase such as “may your good fortune continue.” This must be done without judgement, without comparing, and especially without \textbf{Attachment}.

The final beautiful Mental Factors are the three abstinences.\footnote{Defined in Visuddhimagga XIV.155, explained in Chapter 33 of “Cetasikas” (see Footnote 2 for links).} \textbf{Abstinence from wrong speech} means avoiding lying, slander, harsh speech and idle chatter. \textbf{Abstinence from wrong action} means avoiding killing, stealing and sexual misconduct. \textbf{Abstinence from wrong livelihood} means avoiding jobs dealing in weapons, living beings, butchery, poisons and intoxicants. 

In mundane Mind Moments, abstinences arise one at a time when the opportunity presents itself and we can abstain from only one thing at a time. There are different degrees of abstinence. Abstaining in spite of opportunity is momentary. Abstaining because of observance of precepts is temporary. Abstaining by way of eradication is permanent for saints. We may refrain from wrong speech, action or livelihood because of \textbf{Delusion} or with \textbf{Aversion}, but this is not a wholesome Mind Moment. When we abstain from wrong speech, action or livelihood with kindness and patience, this is a wholesome Mind Moment.

As the spiritual path progresses, five Mental Factors, referred to as spiritual faculties, tend to dominate the mind. They are \textbf{Faith}, \textbf{Energy}, \textbf{Mindfulness}, Concentration\footnote{Concentration is the Mental Factor of \textbf{One-pointedness}.} and \textbf{Understanding}. The Commentary stresses the importance of balancing the spiritual faculties as part of meditation.\footnote{See Visuddhimagga IV.45--49 (See Footnote 2 for link).} \textbf{Faith} and \textbf{Understanding} must be balanced, \textbf{Energy} and concentration must be balanced. I feel that we can apply a similar principle on a broader scale when considering our practice. \textbf{Faith}-based practice may involve chanting and attending ceremonies. \textbf{Energy}-based practice may involve volunteering and charity works. \textbf{Mindfulness}-based practice may involve \textit{vipassanā} meditation. Concentration-based practice may involve \textit{samatha} meditation. \textbf{Understanding}-based practice may involve studying and teaching the Dhamma. Just as all five fingers of the hand need to be strong and coordinated, so too the five types of practice can best support each other when each of them is equally developed. For example, you may look at yourself and say, “My practice puts a lot of emphasis on \textbf{Understanding}, but not so much on \textbf{Energy}. Perhaps I should look at ways of increasing my \textbf{Energy}-based practice by doing volunteer work.”

\pagebreak

\subsubsection*{Mental Factors arising in the Mind Moments \textbf{31}--\textbf{89}}

\begin{figure}[H]

\begin{center}
\setlength{\tabcolsep}{0pt}
\renewcommand{\arraystretch}{1.1}

\noindent\begin{tabular}{P{.05\textwidth}C{.04\textwidth}L{.22\textwidth}C{.04\textwidth}|p{.04\textwidth}p{.04\textwidth}p{.04\textwidth}p{.04\textwidth}p{.04\textwidth}p{.04\textwidth}p{.04\textwidth}|p{.04\textwidth}p{.04\textwidth}p{.04\textwidth}p{.04\textwidth}p{.04\textwidth}}
\toprule
& & & & \tablevsubheaderhacksmall{Universal Ethically-variable} & \tablevsubheaderhacksmall{\textbf{Initial application}} & \tablevsubheaderhacksmall{\textbf{Sustained application}} & \tablevsubheaderhacksmall{\textbf{Certainty}} & \tablevsubheaderhacksmall{\textbf{Energy}} & \tablevsubheaderhacksmall{\textbf{Zest}} & \tablevsubheaderhacksmall{\textbf{Wish to do}} & \tablevsubheaderhacksmall{Universal beautiful} & \tablevsubheaderhacksmall{\textbf{Understanding}} & \tablevsubheaderhacksmall{\textbf{Compassion}} & \tablevsubheaderhacksmall{\textbf{Sympathetic joy}} & \tablevsubheaderhacksmall{One of three abstinences} \\
\midrule
\textbf{31} & U & \textbf{Understanding} & \smiley & \tmsmall & \tmsmall & \tmsmall & \tmsmall & \tmsmall & \tmsmall & \tmsmall & \tmsmall & \tmsmall & \lcsmall & \lcsmall & \lcsmall \\
\textbf{32} & P & \textbf{Understanding} & \smiley & \tmsmall & \tmsmall & \tmsmall & \tmsmall & \tmsmall & \tmsmall & \tmsmall & \tmsmall & \tmsmall & \lcsmall & \lcsmall & \lcsmall \\
\textbf{33} & U & No \textbf{Understanding} & \smiley & \tmsmall & \tmsmall & \tmsmall & \tmsmall & \tmsmall & \tmsmall & \tmsmall & \tmsmall & & \lcsmall & \lcsmall & \lcsmall \\
\textbf{34} & P & No \textbf{Understanding} & \smiley & \tmsmall & \tmsmall & \tmsmall & \tmsmall & \tmsmall & \tmsmall & \tmsmall & \tmsmall & & \lcsmall & \lcsmall & \lcsmall \\
\textbf{35} & U & \textbf{Understanding} & \neutral & \tmsmall & \tmsmall & \tmsmall & \tmsmall & \tmsmall & & \tmsmall & \tmsmall & \tmsmall & \lcsmall & \lcsmall & \lcsmall \\
\textbf{36} & P & \textbf{Understanding} & \neutral & \tmsmall & \tmsmall & \tmsmall & \tmsmall & \tmsmall & & \tmsmall & \tmsmall & \tmsmall & \lcsmall & \lcsmall & \lcsmall \\
\textbf{37} & U & No \textbf{Understanding} & \neutral & \tmsmall & \tmsmall & \tmsmall & \tmsmall & \tmsmall & & \tmsmall & \tmsmall & & \lcsmall & \lcsmall & \lcsmall \\
\textbf{38} & P & No \textbf{Understanding} & \neutral & \tmsmall & \tmsmall & \tmsmall & \tmsmall & \tmsmall & & \tmsmall & \tmsmall & & \lcsmall & \lcsmall & \lcsmall \\
\bottomrule
\end{tabular}
\end{center}

\begin{center}
\noindent
U \hspace{2mm} Unprompted\hspace{5mm} P \hspace{2mm} Prompted

\smiley \hspace {2mm} Pleasant \textbf{Feeling} \hspace{5mm} \neutral \hspace{2mm} Indifferent \textbf{Feeling}

\tmsmall \hspace{2mm} Always in Mind Moment\hspace{5mm} \lcsmall \hspace{2mm} Sometimes in Mind Moment

\end{center}

\caption{Mental Factors arising in Mind Moments \textbf{31}--\textbf{38}.}
\label{fig:Wholesome}
\end{figure}

Before leaving the beautiful Mental Factors, let’s look at some of the Mind Moments in which they arise.\footnote{Beautiful Mental Factors can arise in Mind Moments \textbf{31}--\textbf{89}, not just in the Faultless Zone.}

\textbf{Understanding} arises in Mind Moments \textbf{31}, \textbf{32}, \textbf{35} and \textbf{36}. \textbf{Understanding} also arises in the Jhāna Mind Moments, Mind Moments \textbf{55}--\textbf{81}. 

\textbf{Compassion}, \textbf{Sympathetic joy} and each of the three abstinences can arise only when there is a suitable object, an object deserving of \textbf{Compassion}, an object deserving of \textbf{Sympathetic joy} and so on. This means that none or only one of these five Mental Factors will arise in a mundane Mind Moment.\footnote{In Supramundane Mind Moments \textbf{82}--\textbf{89}, the three abstinences arise together but \textbf{Compassion} and \textbf{Sympathetic joy} cannot arise as the object of Supramundane Mind Moments is \textit{Nibbāna}.}

\subsection*{Linkage to \textit{Satipaṭṭhāna} Sutta}

\begin{figure}[H]
\centering
\input{./Diagrams/Sati.pdf_tex}
\caption{Structure of the Satipaṭṭhāna Sutta.}
\label{fig:Sati}
\end{figure}

Please refer to the \textit{Satipaṭṭhāna} Sutta in Appendix 1. At the beginning of this lesson, I shared two different ways of looking at Mental Factors; as components of Mind Moments and as activities within a Mind Moment. Looking at Mental Factors as components is treating them as nouns while looking at Mental Factors as activities is treating them as verbs.

``\textit{Satipaṭṭhāna}" is a compound word that be broken apart in two ways; one way looks at it as a noun and the other way looks at it as a verb.

\pagebreak

We can break apart \textit{Satipaṭṭhāna} into \textit{sati} and \textit{paṭṭhāna}. \textit{Sati} means \textbf{Mindfulness}, and \textit{paṭṭhāna} is a noun meaning “foundation” or “cause.”\footnote{Extract from page 29 of \url{http://www.buddhismuskunde.uni-hamburg.de/pdf/5-personen/analayo/direct-path.pdf}: “This seems unlikely, since in the discourses contained in the Pāḷi canon the corresponding verb \textit{paṭṭhahati} never occurs together with \textit{sati}. Moreover, the noun \textit{paṭṭhāna} is not found at all in the early discourses, but comes into use only in the historically later Abhidhamma and the Commentaries.} We can also break apart \textit{Satipaṭṭhāna} into \textit{sati} and \textit{upaṭṭhāna}. \textit{Upaṭṭhāna} is a verb meaning “placing near,” “being present” or “attending to.”\footnote{Continuing from the same extract: “In contrast, the discourses frequently relate “\textit{sati}” to the verb “\textit{upaṭṭhahati},” indicating that “presence” is the etymologically correct derivation. In fact, the equivalent Sanskrit term is \textit{smṛtyupasthāna}, which shows that \textit{upasthāna}, or its Pāḷi equivalent “\textit{upaṭṭhāna}” is the correct choice for the compound.}

If we take the noun derivation of \textit{Satipaṭṭhāna}, we get “foundations of \textbf{Mindfulness}” and we emphasize the object, whereas if we take the verb derivation of \textit{Satipaṭṭhāna}, we get “being present with \textbf{Mindfulness}” or “attending with \textbf{Mindfulness}” and we emphasize the activity. In other lessons, I discuss the objects described in the \textit{Satipaṭṭhāna} Sutta. In this lesson, I will discuss the activity described in the \textit{Satipaṭṭhāna} Sutta from an Abhidhamma perspective.\footnote{The following discussion is very superficial. Meditation teachers spend many hours discussing these points during Dhamma talks given as part of meditation retreats.}

\subsubsection*{Definition}

Paragraph 3 of the \textit{Satipaṭṭhāna} Sutta defines the activity or approach to \textbf{Mindfulness} as “ardent, clearly comprehending and mindful, having overcome, in this world, covetousness and grief.” Let’s look at each of these terms individually.

The Commentary explains that the Pāḷi word translated as “ardent” is a synonym for the Mental Factor of \textbf{Energy}. The Pāḷi word is derived from a word meaning “fire,” so the implication is a glowing \textbf{Energy} that burns through the defilements. We must practise with a burning \textbf{Energy}.\footnote{I feel that the phrase “burning \textbf{Energy}” captures the urgency of practice better than the word “ardent.”}

The Commentary explains that the Pāḷi word translated as “clearly comprehending” is a form of the Mental Factor of \textbf{Understanding}.

There is another Sutta\footnote{SN 47.4: \url{http://suttacentral.net/en/sn47.4}} describing \textit{satipaṭṭhāna} in which the phrase “having overcome covetousness and grief” is replaced by “unified, concentrated with one-pointed mind,” in other words, the Mental Factor of \textbf{One-pointedness} or concentration.

So if we include the preceding paragraph which has the quote, “this is the only way,” a statement representing the Mental Factor of \textbf{Faith}, we can see that the practice of \textit{satipaṭṭhāna} includes the activities covered by \textbf{Faith}, \textbf{Energy}, \textbf{Mindfulness}, concentration and \textbf{Understanding}. In the Suttas, these are called the five spiritual faculties and when developed, they become the five spiritual powers.\footnote{Chapter 48 of the \textit{Saṃyutta Nikāya} includes Suttas on spiritual faculties (\textit{indriya}) and Chapter 50 of the \textit{Saṃyutta Nikāya} includes Suttas on spiritual powers (\textit{bala}).}

Finally, to understand the expression “in this world,” we should refer to the Sutta\footnote{AN 4.45: \url{http://www.accesstoinsight.org/tipitaka/an/an04/an04.045.than.html}} which says, “it is just within this fathom-long body, with its perception and intellect, that I declare that there is the world, the origination of the world, the cessation of the world, and the path of practice leading to the cessation of the world.”

\subsubsection*{Refrain}

If you look at the structure of the \textit{Satipaṭṭhāna} Sutta, you will notice that a standard paragraph is repeated after every exercise. I am referring to paragraph 9 which is repeated at paragraph 11, paragraph 13, paragraph 16 and so on. Though it is not explicitly shown in this version, this paragraph is also repeated after each of the cemetery contemplations. This standard paragraph, which is called the “refrain,” is actually repeated 21 times in the \textit{Satipaṭṭhāna} Sutta. If you are like me, you tend to skip over all this repetition when reading a Sutta. “Yeah, yeah, I’ve read this before. What is the next new thing?” Perhaps the reason this paragraph is repeated 21 times is that it is so important!

The refrain describes the process of \textit{satipaṭṭhāna} whereas the exercises describe the objects of \textit{satipaṭṭhāna}. It has been observed that the tendency to become absorbed in the content of the awareness, rather than continuing to attend to the process of awareness, often causes western meditators to progress more slowly than their eastern counterparts.\footnote{\url{http://www.atpweb.org/jtparchive/trps-16-84-01-025.pdf}}

The refrain has four themes: “internally/externally,” “origination/dissolution,” “to the extent necessary just for knowledge and \textbf{Mindfulness}” and “lives detached, and clings to nothing in the world.”

According to the Abhidhamma,\footnote{Found in the \textit{Vibhaṅga} and also in Visuddhimagga XIII.109. Interestingly, the version of the \textit{Satipaṭṭhāna} Sutta found in the \textit{Vibhaṅga} (which as mentioned earlier, probably pre-dates the version in the \textit{Nikāyas}) includes the internally/externally distinction in the “definition” portion, making it part of what constitutes “right \textbf{Mindfulness}.”} “internally” means phenomena arising in oneself; externally means phenomena arising in others,\footnote{“External” phenomena such as \textbf{Feelings} and Mind Moments are observed indirectly by looking at another’s expressions, listening to their voice, etc.} and “internally and externally” means awareness of the general underlying principle. The sequence progresses from the most easily observed to the most abstract.

“Origination/dissolution/origination and dissolution” also progresses from the most easily observed to the most abstract. Phenomena are most easily observed when they initially arise. It is sometimes difficult to catch when phenomena fall away. The general principle of “origination and dissolution” is the characteristic of impermanence.\footnote{The Suttas (AN 4.94: \url{http://www.accesstoinsight.org/tipitaka/an/an04/an04.094.than.html}) describe \textit{samatha} and \textit{vipassanā} as follows: \textit{samatha} is a process of steadying, settling, unifying and composing the mind, while \textit{vipassanā} is taking \textit{sankhāra} (\textit{anicca}, \textit{dukkha}, \textit{anattā}) as an object. Therefore, observing “origination and dissolution” is a \textit{vipassanā} practice because it is seeing \textit{anicca}.}

“To the extent necessary just for knowledge and \textbf{Mindfulness}” means to observe objectively, without getting lost in associations or reactions. The use of “labelling” during meditation is a useful tool. Giving an experience a label such as “thought” or “pain” helps to depersonalize the experience; it is no longer “my thought” or “pain happening to me,” it is simply something to be observed.

Finally, “lives detached, clings to nothing in this world” is a warning against setting goals in the practice. If one is craving progress, one gets craving, not progress.\footnote{Interestingly, if a patient approaches Mindfulness Based Stress Relief with an objective of strengthening their immune system, the effect of the program on their immune system is much less than if the patient approaches the program without a goal.}

\pagebreak

\subsection*{Summary of Key Points}

\begin{itemize}

\item We can think of consciousness and the Mental Factors as being either components or as interdependent activities; each perspective gives a different insight into the nature of a Mind Moment.

\item There are 52 Mental Factors:

\begin{itemize}

\item The seven universal ethically-variable Mental Factors (\textbf{Contact}, \textbf{Feeling}, \textbf{Perception}, \textbf{Volition}, \textbf{One-pointedness}, \textbf{Attention} and \textbf{Life faculty}) arise in all Mind Moments (unwholesome, ethically-neutral and wholesome).

\item The six occasional ethically-variable Mental Factors (\textbf{Initial application}, \textbf{Sustained application}, \textbf{Certainty}, \textbf{Energy}, \textbf{Zest} and \textbf{Wish to do}) arise in some Mind Moments (unwholesome, ethically-neutral and wholesome).

\item The four universal unwholesome Mental Factors (\textbf{Delusion}, \textbf{Shamelessness}, \textbf{Recklessness} and \textbf{Restlessness}) arise in all unwholesome Mind Moments.

\item The ten occasional unwholesome Mental Factors (\textbf{Attachment}, \textbf{Wrong view}, \textbf{Conceit}, \textbf{Aversion}, \textbf{Envy}, \textbf{Stinginess}, \textbf{Remorse}, \textbf{Sloth}, \textbf{Torpor} and \textbf{Doubt}) arise in some unwholesome Mind Moments.

\item The 19 universal beautiful Mental Factors (\textbf{Faith}, \textbf{Mindfulness}, \textbf{Conscience}, \textbf{Fear of blame}, \textbf{Non-attachment}, \textbf{Non-aversion}, \textbf{Equanimity}, plus the six pairs of \textbf{Tranquillity}, \textbf{Agility}, \textbf{Pliancy}, \textbf{Adaptability}, \textbf{Proficiency} and \textbf{Uprightness}) arise in all wholesome Mind Moments.

\item The six occasional beautiful Mental Factors (\textbf{Understanding}, \textbf{Compassion}, \textbf{Sympathetic joy}, \textbf{Abstinence from wrong speech}, \textbf{Abstinence from wrong action} and \textbf{Abstinence from wrong livelihood}) arise in some wholesome Mind Moments.

\end{itemize}

\item A clear understanding of how the Buddha used these terms is very helpful in better understanding the Suttas/Dhamma talks.

\item Knowing the Mental Factors helps us to \textbf{\textit{R}}ecognize the current Mind Moment; \textbf{\textit{R}}ecognize is the first step in the \textbf{\textit{R}} \textbf{\textit{A}} \textbf{\textit{D}} \textbf{\textit{I}} \textbf{\textit{CA}} \textbf{\textit{L}} process (\textbf{\textit{R}}ecognize, \textbf{\textit{A}}ccept, \textbf{\textit{D}}epersonalize, \textbf{\textit{I}}nvestigate, \textbf{\textit{C}}ontemplate \textbf{\textit{A}}\textit{nicca}/\textbf{\textit{C}}ontemplate \textbf{\textit{A}}\textit{nattā}, \textbf{\textit{L}}et go).

\end{itemize}

Finally, in my opinion, the first important thing to remember about Mental Factors is that they arise naturally, because of conditions, so we must accept them as they are. The second important thing to remember about Mental Factors is that they help us to know if the mind is in the Danger Zone or in the Faultless Zone. If the mind is in the Danger Zone, reflect upon the disadvantages of such thinking. If the mind is in the Faultless Zone, just be passively aware.

\newpage

\subsection*{Questions \& Answers}

\question{Is all \textbf{Attachment} unwholesome, even attachment to the Dhamma?}

To get across the river of suffering, one needs a raft which is the Noble Eightfold Path. Having crossed the river and experienced \textit{Nibbāna}, there is no need to continue to cling to the raft. The Buddha said,\footnote{MN 22: \url{http://www.accesstoinsight.org/tipitaka/mn/mn.022.than.html\#raft}} “I have taught the Dhamma compared to a raft for the purpose of crossing over, not for the purpose of holding onto. Understanding the Dhamma as taught compared to a raft, you should let go even of Dhammas, to say nothing of non-Dhammas.” Until one has crossed the river, the “letting go of the raft” remains a future possibility; one should hold onto the raft tightly until one has already crossed the river. There are some who reject the idea of formal meditation practice on the grounds that this reinforces the view of a “Self who practises.” In my opinion, at my stage of spiritual development, the structure provided by formal meditation practice is extremely useful. At my stage of spiritual development, I don’t get too worried about being attached to the Dhamma.

\question{Is my \textbf{Attachment} to my spouse and children unwholesome?}

A layman once asked the Buddha to teach him the origin and the ending of stress (\textit{dukkha}).\footnote{SN 42.11: \url{ http://www.accesstoinsight.org/tipitaka/sn/sn42/sn42.011.than.html }} The Buddha replied, “Are there any people who if they were murdered or imprisoned, would cause you to have stress?  Are there any people who if they were murdered or imprisoned, would not cause you to have stress? What is the difference between these two groups of people?” The layman answered, “If my friends or family were murdered or imprisoned, this would cause me stress. I would not have stress if people I did not know were murdered or imprisoned.” The Buddha concluded, “You can see that \textbf{Attachment} is a condition for stress to arise.”

The word ``unwholesome" may seem a bit extreme, but if your only purpose in life was spiritual development, you could see your spouse and children as obstructions. The Bodhisatta named his son ``\textit{Rāhula}" which means ``bond" because he viewed his son as binding him to lay life. To become a monk or nun, one must give up one's family.

\question{How important is aspiration to spiritual development?}

This question was asked in a Sutta.\footnote{MN 126: \url{http://www.accesstoinsight.org/tipitaka/mn/mn.126.than.html}} This Sutta explains that it is proper practice that leads to spiritual development, not aspiration. Improper practice, irrespective of the underlying aspiration, can never lead to spiritual development. Proper practice, irrespective of the underlying aspiration, leads to spiritual development.\footnote{In other words, the desire for awakening does not get in the way of awakening.}

A \textbf{Wish to do} is strong if the associated \textbf{Energy} is strong and the associated beautiful or unwholesome Mental Factors are also strong. Otherwise, the \textbf{Wish to do} is weak. Sometimes people have the \textbf{Wish to do} something but the \textbf{Volition} is not strong; “The spirit is willing, but the flesh is weak.”\footnote{\url{https://en.wiktionary.org/wiki/the_spirit_is_willing_but_the_flesh_is_weak}}

\pagebreak

\question{What is the ``Buddhist" way to control anger?}

Bhikkhu Visuddhācāra's book “Curbing Anger Spreading Love” suggests the following:\footnote{\url{http://www.ebookdb.org/iread.php?id=G433G739G11837G1G8273469}}

\begin{itemize}
\item First Rule: Mindfulness (\textit{sati}) is the first and best guard against anger and all unwholesome states of mind. When we watch anger, we are not paying attention to the trigger.

\item Firm resolution in maintaining calmness; make an effort to be calm at all times, before triggers arise.

\item Consider the Buddha's fine example; the Buddha did not get angry at anybody. He did not even get angry with Devadatta who tried to kill him.

\item Consider that we must all die one day. A moment of anger is a wasted moment. Reflection on death (\textit{maraṇānussati}) raises one’s sense of urgency for spiritual development. ``There are those who do not realize that one day we all must die. But those who do realize this settle their quarrels."\footnote{Dhammapada verse 6: \url{http://www.accesstoinsight.org/tipitaka/kn/dhp/dhp.01.budd.html}}

\item Consider the harmful effects of anger on oneself: “What is the point of your getting angry with him? Will not this kamma of yours that has anger as its source lead to your own harm? By doing this you are like a man who wants to hit another and picks up a burning ember or excrement in his hand and so first burns himself or makes himself stink.”\footnote{Visuddhimagga IX.23 (See footnote 2 for link.)}

\item Look into a mirror; an angry person is a very ugly person.

\item Consider their good points; look at the big picture, don't focus on the negative.

\item Freeze! Whenever anger arises, we should freeze like a block of wood and not react.

\item Nobody is free from blame: people in glass houses should not throw stones.

\item Why are we angry? We get angry because we have an ego, a BIG ego!

\item Who is angry? There is no self (\textit{anattā}), only processes of mind and matter. 

\item Consider that we are all related. Almost everybody has been our relative in a past life.

\item Forgiveness: Forgive others and forgive ourselves. ``Hatred is never appeased by hatred in this world. By non-hatred alone is hatred appeased. This is a law eternal."\footnote{Dhammapada verse 5: \url{http://www.accesstoinsight.org/tipitaka/kn/dhp/dhp.01.budd.html}}

\item Review the benefits of loving kindness (\textit{mettā}): One sleeps easily, wakes easily, dreams no evil dreams. One is dear to human beings, dear to non-human beings. The \textit{Devas} protect one. Neither fire, poison, nor weapons can touch one. One's mind gains concentration quickly. One's complexion is bright. One dies unconfused and, if penetrating no higher, is headed for the Brahma worlds.\footnote{AN 11.16: \url{http://www.accesstoinsight.org/tipitaka/an/an11/an11.016.than.html}}

\item Give a gift: an unexpected way to break down barriers.

\end{itemize}