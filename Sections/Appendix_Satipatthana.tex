\section{Appendix 1 -- \textit{Satipaṭṭhāna} Sutta}

Translated from the Pāḷi by Ñāṇasatta Thera (footnotes by Ñāṇasatta Thera, paragraph numbers and Pāḷi accents added).\footnote{\url{http://www.accesstoinsight.org/tipitaka/mn/mn.010.nysa.html}}

\begin{enumerate}
\item Thus have I heard. At one time the Blessed One was living among the Kurus, at Kammāsadamma, a market town of the Kuru people. There the Blessed One addressed the bhikkhu thus: ``Monks,” and they replied to him, ``Venerable Sir.” The Blessed One spoke as follows:
\item This is the only way, monks, for the purification of beings, for the overcoming of sorrow and lamentation, for the destruction of suffering and grief, for reaching the right path, for the attainment of Nibbāna, namely, the four foundations of mindfulness. What are the four?
\item Herein (in this teaching) a monk lives contemplating the body in the body,\footnote{The repetition of the phrases ‘contemplating the body in the body,’ ‘feelings in feelings,’ etc. is meant to impress upon the meditator the importance of remaining aware whether, in the sustained attention directed upon a single chosen object, one is still keeping to it, and has not strayed into the field of another contemplation. For instance, when contemplating any bodily process, a meditator may unwittingly be side-tracked into a consideration of his feelings connected with that bodily process. He should then be clearly aware that he has left his original subject, and is engaged in the contemplation of feeling.} ardent, clearly comprehending and mindful, having overcome, in this world, covetousness and grief; he lives contemplating feelings in feelings, ardent, clearly comprehending and mindful, having overcome, in this world, covetousness and grief; he lives contemplating consciousness in consciousness,\footnote{Mind (Pāḷi citta, also consciousness or viññāṇa) in this connection means the states of mind or units in the stream of mind of momentary duration. Mental objects, dhamma, are the mental contents or factors of consciousness making up the single states of mind.} ardent, clearly comprehending and mindful, having overcome, in this world, covetousness and grief; he lives contemplating mental objects in mental objects, ardent, clearly comprehending and mindful, having overcome, in this world, covetousness and grief.
\end{enumerate}

\subsection*{The Contemplation of the Body}
\subsubsection*{Mindfulness of Breathing}
\begin{enumerate}[resume]
\item And how does a monk live contemplating the body in the body?
\item Herein, monks, a monk, having gone to the forest, to the foot of a tree or to an empty place, sits down with his legs crossed, keeps his body erect and his mindfulness alert.\footnote{Literally, “setting up mindfulness in front.”}
\item Ever mindful he breathes in, mindful he breathes out. Breathing in a long breath, he knows, “I am breathing in a long breath;” breathing out a long breath, he knows, “I am breathing out a long breath;” breathing in a short breath, he knows, “I am breathing in a short breath;” breathing out a short breath, he knows, “I am breathing out a short breath.”

\pagebreak

\item “Experiencing the whole (breath-) body, I shall breathe in,” thus he trains himself. “Experiencing the whole (breath-) body, I shall breathe out,” thus he trains himself. “Calming the activity of the (breath-) body, I shall breathe in,” thus he trains himself. “Calming the activity of the (breath-) body, I shall breathe out,” thus he trains himself.
\item Just as a skillful turner or turner's apprentice, making a long turn, knows, ``I am making a long turn,” or making a short turn, knows, ``I am making a short turn,” just so the monk, breathing in a long breath, knows, ``I am breathing in a long breath; breathing out a long breath, he knows, ``I am breathing out a long breath;” breathing in
a short breath, he knows, “I am breathing in a short breath;” breathing out a short breath, he knows, “I am breathing out a short breath.” “Experiencing the whole (breath-) body, I shall breathe in,” thus he trains himself. “Experiencing the whole (breath-) body, I shall breathe out,” thus he trains himself. “Calming the activity of the (breath-) body, I shall breathe in,” thus he trains himself. “Calming the activity of the (breath-) body, I shall breathe out,” thus he trains himself.
\item Thus he lives contemplating the body in the body internally, or he lives contemplating the body in the body externally, or he lives contemplating the body in the body internally and externally.\footnote{‘Internally’: contemplating his own breathing; ‘externally’: contemplating another’s breathing; ‘internally and externally’: contemplating one’s own and another’s breathing, alternately, with uninterrupted attention. In the beginning one pays attention to one’s own breathing only, and it is only in advanced stages that for the sake of practicing insight, one by inference at times pays attention also to another person’s process of breathing.} He lives contemplating origination factors\footnote{The origination factors (samudaya-dhamma), that is, the conditions of the origination of the breath-body; these are: the body in its entirety, nasal aperture and mind.} in the body, or he lives contemplating dissolution factors\footnote{The conditions of the dissolution of the breath-body are: the destruction of the body and of the nasal aperture, and the ceasing of mental activity.} in the body, or he lives contemplating origination-and-dissolution factors\footnote{The contemplation of both, alternately.} in the body. Or his mindfulness is established with the thought: “The body exists,”\footnote{That is, only impersonal bodily processes exist, without a self, soul, spirit or abiding essence or substance. The corresponding phrase in other contemplations should be understood accordingly.} to the extent necessary just for knowledge and mindfulness, and he lives detached,\footnote{Detached from craving and wrong view.} and clings to nothing in the world. Thus also, monks, a monk lives contemplating the body in the body.
\end{enumerate}
\subsubsection*{Postures of the Body}
\begin{enumerate}[resume]
\item And further, monks, a monk knows, when he is going, “I am going;” he knows, when he is standing, “I am standing;” he knows, when he is sitting, “I am sitting;” he knows, when he is lying down, “I am lying down;” or just as his body is disposed so he knows it.
\pagebreak
\item Thus he lives contemplating the body in the body internally, or he lives contemplating the body in the body externally, or he lives contemplating the body in the body internally and externally. He lives contemplating origination factors in the body, or he lives contemplating dissolution factors in the body, or he lives contemplating origination-and-dissolution factors in the body.\footnote{All contemplations of the body, excepting the preceding one, have as factors of origination: ignorance, craving, kamma, food, and the general characteristic of originating; the factors of dissolution are: disappearance of ignorance, craving, kamma, food, and the general characteristic of dissolving.} Or his mindfulness is established with the thought: “The body exists,” to the extent necessary just for knowledge and mindfulness, and he lives detached, and clings to nothing in the world. Thus also, monks, a monk lives contemplating the body in the body.
\end{enumerate}
\subsubsection*{Mindfulness with Clear Comprehension}
\begin{enumerate}[resume]
\item And further, monks, a monk, in going forward and back, applies clear comprehension; in looking straight on and looking away, he applies clear comprehension; in bending and in stretching, he applies clear comprehension; in wearing robes and carrying the bowl, he applies clear comprehension; in eating, drinking, chewing and savoring, he applies clear comprehension; in walking, in standing, in sitting, in falling asleep, in waking, in speaking and in keeping silence, he applies clear comprehension.
\item Thus he lives contemplating the body in the body...
\end{enumerate}
\subsubsection*{Reflection on the Repulsiveness of the Body}
\begin{enumerate}[resume]
\item And further, monks, a monk reflects on this very body enveloped by the skin and full of manifold impurity, from the soles up, and from the top of the head-hairs down, thinking thus: “There are in this body hair of the head, hair of the body, nails, teeth, skin, flesh, sinews, bones, marrow, kidney, heart, liver, midriff, spleen, lungs, intestines, mesentery, gorge, feces, bile, phlegm, pus, blood, sweat, fat, tears, grease, saliva, nasal mucus, synovial fluid, urine.”
\item Just as if there were a double-mouthed provision bag full of various kinds of grain such as hill paddy, paddy, green gram, cow-peas, sesamum, and husked rice, and a man with sound eyes, having opened that bag, were to take stock of the contents thus: “This is hill paddy, this is paddy, this is green gram, this is cow-pea, this is sesamum, this is husked rice.” Just so, monks, a monk reflects on this very body enveloped by the skin and full of manifold impurity, from the soles up, and from the top of the head-hairs down, thinking thus: “There are in this body hair of the head, hair of the body, nails, teeth, skin, flesh, sinews, bones, marrow, kidney, heart, liver, midriff, spleen, lungs, intestines, mesentery, gorge, feces, bile, phlegm, pus, blood, sweat, fat, tears, grease, saliva, nasal mucus, synovial fluid, urine.”
\item Thus he lives contemplating the body in the body...
\end{enumerate}

\pagebreak

\subsubsection*{Reflection on the Material Elements}
\begin{enumerate}[resume]
\item And further, monks, a monk reflects on this very body, however it be placed or disposed, by way of the material elements: “There are in this body the element of earth, the element of water, the element of fire, the element of wind.”\footnote{The so-called ‘elements’ are the primary qualities of matter, explained by Buddhist tradition as solidity (earth), adhesion (water), caloricity (fire) and motion (wind or air).}
\item Just as if, monks, a clever cow-butcher or his apprentice, having slaughtered a cow and divided it into portions, should be sitting at the junction of four high roads, in the same way, a monk reflects on this very body, as it is placed or disposed, by way of the material elements: “There are in this body the elements of earth, water, fire, and wind.”
\item Thus he lives contemplating the body in the body...
\end{enumerate}
\subsubsection*{Nine Cemetery Contemplations}
\begin{enumerate}[resume]
\item And further, monks, as if a monk sees a body dead one, two, or three days; swollen, blue and festering, thrown in the charnel ground, he then applies this perception to his own body thus: “Verily, also my own body is of the same nature; such it will become and will not escape it.”
\item Thus he lives contemplating the body in the body internally, or he lives contemplating the body in the body externally, or he lives contemplating the body in the body internally and externally. He lives contemplating origination-factors in the body, or he lives contemplating dissolution factors in the body, or he lives contemplating origination-and-dissolution-factors in the body. Or his mindfulness is established with the thought: “The body exists,” to the extent necessary just for knowledge and mindfulness, and he lives detached, and clings to nothing in the world. Thus also, monks, a monk lives contemplating the body in the body.
\item And further, monks, as if a monk sees a body thrown in the charnel ground, being eaten by crows, hawks, vultures, dogs, jackals or by different kinds of worms, he then applies this perception to his own body thus: “Verily, also my own body is of the same nature; such it will become and will not escape it.”
\item Thus he lives contemplating the body in the body...
\item And further, monks, as if a monk sees a body thrown in the charnel ground and reduced to a skeleton with some flesh and blood attached to it, held together by the tendons...
\item And further, monks, as if a monk sees a body thrown in the charnel ground and reduced to a skeleton blood-besmeared and without flesh, held together by the tendons...
\item And further, monks, as if a monk sees a body thrown in the charnel ground and reduced to a skeleton without flesh and blood, held together by the tendons...
\item And further, monks, as if a monk sees a body thrown in the charnel ground and reduced to disconnected bones, scattered in all directions; here a bone of the hand, there a bone of the foot, a shin bone, a thigh bone, the pelvis, spine and skull...
\item And further, monks, as if a monk sees a body thrown in the charnel ground, reduced to bleached bones of conchlike color...
\item And further, monks, as if a monk sees a body thrown in the charnel ground reduced to bones, more than a year-old, lying in a heap...
\item And further, monks, as if a monk sees a body thrown in the charnel ground, reduced to bones gone rotten and become dust, he then applies this perception to his own body thus: “Verily, also my own body is of the same nature; such it will become and will not escape it.”
\item Thus he lives contemplating the body in the body internally, or he lives contemplating the body in the body externally, or he lives contemplating the body in the body internally and externally. He lives contemplating origination factors in the body, or he lives contemplating dissolution factors in the body, or he lives contemplating origination-and-dissolution factors in the body. Or his mindfulness is established with the thought: “The body exists,” to the extent necessary just for knowledge and mindfulness, and he lives detached, and clings to nothing in the world. Thus also, monks, a monk lives contemplating the body in the body.
\end{enumerate}

\subsection*{The Contemplation of Feeling}
\begin{enumerate}[resume]
\item And how, monks, does a monk live contemplating feelings in feelings?
\item Herein, monks, a monk when experiencing a pleasant feeling knows, “I experience a pleasant feeling;” when experiencing a painful feeling, he knows, “I experience a painful feeling;” when experiencing a neither-pleasant-nor-painful feeling, he knows, “I experience a neither-pleasant-nor-painful feeling.” When experiencing a pleasant worldly feeling, he knows, “I experience a pleasant worldly feeling;” when experiencing a pleasant spiritual feeling, he knows, “I experience a pleasant spiritual feeling;” when experiencing a painful worldly feeling, he knows, “I experience a painful worldly feeling;” when experiencing a painful spiritual feeling, he knows, “I experience a painful spiritual feeling;” when experiencing a neither-pleasant-nor-painful worldly feeling, he knows, “I experience a neither-pleasant-nor-painful worldly feeling;” when experiencing a neither-pleasant-nor-painful spiritual feeling, he knows, “I experience a neither-pleasant-nor-painful spiritual feeling.”
\item Thus he lives contemplating feelings in feelings internally, or he lives contemplating feelings in feelings externally, or he lives contemplating feelings in feelings internally and externally. He lives contemplating origination factors in feelings, or he lives contemplating dissolution factors in feelings, or he lives contemplating origination-and-dissolution factors in feelings.\footnote{The factors of origination are here: ignorance, craving, kamma, and sense-impression, and the general characteristic of originating; the factors of dissolution are: the disappearance of the four, and the general characteristic of dissolving.} Or his mindfulness is established with the thought, “Feeling exists,” to the extent necessary just for knowledge and mindfulness, and he lives detached, and clings to nothing in the world. Thus, monks, a monk lives contemplating feelings in feelings.
\end{enumerate}
\subsection*{The Contemplation of Consciousness}
\begin{enumerate}[resume]
\item And how, monks, does a monk live contemplating consciousness in consciousness?
\item Herein, monks, a monk knows the consciousness with lust, as with lust; the consciousness without lust, as without lust; the consciousness with hate, as with hate; the consciousness without hate, as without hate; the consciousness with ignorance, as with ignorance; the consciousness without ignorance, as without ignorance; the shrunken state of consciousness, as the shrunken state;\footnote{This refers to a rigid and indolent state of mind.} the distracted state of consciousness, as the distracted state;\footnote{This refers to a restless mind.} the developed state of consciousness as the developed state;\footnote{The consciousness of the meditative absorptions of the fine-corporeal and uncorporeal sphere (\textit{rūpa-arūpa-jhāna}).} the undeveloped state of consciousness as the undeveloped state;\footnote{The ordinary consciousness of the sensuous state of existence (\textit{kāmāvacara}).} the state of consciousness with some other mental state superior to it, as the state with something mentally higher;\footnote{The consciousness of the sensuous state of existence, having other mental states superior to it.} the state of consciousness with no other mental state superior to it, as the state with nothing mentally higher;\footnote{The consciousness of the fine-corporeal and the uncorporeal spheres, having no mundane mental state superior to it.} the concentrated state of consciousness, as the concentrated state; the unconcentrated state of consciousness, as the unconcentrated state; the freed state of consciousness, as the freed state;\footnote{Temporarily freed from the defilements either through the methodical practice of insight (\textit{vipassanā}) freeing from single evil states by force of their opposites, or through the meditative absorptions (\textit{jhāna}).} and the unfreed state of consciousness as the unfreed state.
\item Thus he lives contemplating consciousness in consciousness internally, or he lives contemplating consciousness in consciousness externally, or he lives contemplating  consciousness in consciousness internally and externally. He lives contemplating origination factors in consciousness, or he lives contemplating dissolution-factors in consciousness, or he lives contemplating origination-and-dissolution factors in consciousness.\footnote{The factors of origination consist here of ignorance, craving, kamma, body-and-mind (\textit{nāma-rūpa}), and the general characteristic of originating; the factors of dissolution are: the disappearance of ignorance, etc., and the general characteristic of dissolving.} Or his mindfulness is established with the thought, “Consciousness exists,” to the extent necessary just for knowledge and mindfulness, and he lives detached, and clings to nothing in the world. Thus, monks, a monk lives contemplating consciousness in consciousness.
\end{enumerate}
\subsection*{The Contemplation of Mental Objects}
\subsubsection*{Five Hindrances}
\begin{enumerate}[resume]
\item And how, monks, does a monk live contemplating mental objects in mental objects?
\item Herein, monks, a monk lives contemplating mental objects in the mental objects of the five hindrances.
\item How, monks, does a monk live contemplating mental objects in the mental objects of the five hindrances?
\pagebreak

\item Herein, monks, when sense-desire is present, a monk knows, “There is sense-desire in me,” or when sense-desire is not present, he knows, “There is no sense-desire in me.” He knows how the arising of the non-arisen sense-desire comes to be; he knows how the abandoning of the arisen sense-desire comes to be; and he knows how the non-arising in the future of the abandoned sense-desire comes to be.
\item When anger is present, he knows, “There is anger in me,” or when anger is not present, he knows, “There is no anger in me.” He knows how the arising of the non-arisen anger comes to be; he knows how the abandoning of the arisen anger comes to be; and he knows how the non-arising in the future of the abandoned anger comes to be.
\item When sloth and torpor are present, he knows, “There are sloth and torpor in me,” or when sloth and torpor are not present, he knows, “There are no sloth and torpor in me.” He knows how the arising of the non-arisen sloth and torpor comes to be; he knows how the abandoning of the arisen sloth and torpor comes to be; and he knows how the non-arising in the future of the abandoned sloth and torpor comes to be.
\item When agitation and remorse are present, he knows, “There are agitation and remorse in me,” or when agitation and remorse are not present, he knows, “There are no agitation and remorse in me.” He knows how the arising of the non-arisen agitation and remorse comes to be; he knows how the abandoning of the arisen agitation and remorse comes to be; and he knows how the non-arising in the future of the abandoned agitation and remorse comes to be.
\item When doubt is present, he knows, “There is doubt in me,” or when doubt is not present, he knows, “There is no doubt in me.” He knows how the arising of the non-arisen doubt comes to be; he knows how the abandoning of the arisen doubt comes to be; and he knows how the non-arising in the future of the abandoned doubt comes to be.
\item Thus he lives contemplating mental objects in mental objects internally, or he lives contemplating mental objects in mental objects externally, or he lives contemplating mental objects in mental objects internally and externally. He lives contemplating origination factors in mental objects, or he lives contemplating dissolution factors in mental objects, or he lives contemplating origination-and-dissolution factors in mental objects.\footnote{The factors of origination are here the conditions which produce the hindrances, such as wrong reflection, etc., the factors of dissolution are the conditions which remove the hindrances, e.g., right reflection.} Or his mindfulness is established with the thought, “Mental objects exist,” to the extent necessary just for knowledge and mindfulness, and he lives detached, and clings to nothing in the world. Thus also, monks, a monk lives contemplating mental objects in the mental objects of the five hindrances.
\end{enumerate}
\subsubsection*{Five Aggregates of Clinging}
\begin{enumerate}[resume]
\item And further, monks, a monk lives contemplating mental objects in the mental objects of the five aggregates of clinging.\footnote{These five groups or aggregates constitute the so-called personality. By making them objects of clinging, existence, in the form of repeated births and deaths, is perpetuated.}
\item How, monks, does a monk live contemplating mental objects in the mental objects of the five aggregates of clinging?
\item Herein, monks, a monk thinks, “Thus is material form; thus is the arising of material form; and thus is the disappearance of material form. Thus is feeling; thus is the arising of feeling; and thus is the disappearance of feeling. Thus is perception; thus is the arising of perception; and thus is the disappearance of perception. Thus are formations; thus is the arising of formations; and thus is the disappearance of formations. Thus is consciousness; thus is the arising of consciousness; and thus is the disappearance of consciousness.”
\item Thus he lives contemplating mental objects in mental objects internally, or he lives contemplating mental objects in mental objects externally, or he lives contemplating mental objects in mental objects internally and externally. He lives contemplating origination factors in mental objects, or he lives contemplating dissolution factors in mental objects, or he lives contemplating origination-and-dissolution factors in mental objects.\footnote{The origination-and-dissolution factors of the five aggregates: for material form, the same as for the postures; for feeling, the same as for the contemplation of feeling; for perception and formations, the same as for feeling; for consciousness, the same as for the contemplation of consciousness.} Or his mindfulness is established with the thought, “Mental objects exist,” to the extent necessary just for knowledge and mindfulness, and he lives detached, and clings to nothing in the world. Thus also, monks, a monk lives contemplating mental objects in the mental objects of the five aggregates of clinging.
\end{enumerate}
\subsubsection*{Six Internal and Six External Sense Bases}
\begin{enumerate}[resume]
\item And further, monks, a monk lives contemplating mental objects in the mental objects of the six internal and the six external sense-bases.
\item How, monks, does a monk live contemplating mental objects in the mental objects of the six internal and the six external sense-bases?
\item Herein, monks, a monk knows the eye and visual forms and the fetter that arises dependent on both (the eye and forms);\footnote{The usual enumeration of the ten principal fetters (saṃyojana), as given in the Discourse Collection (Sutta Piṭaka), is as follows: (1) self-illusion, (2) skepticism, (3) attachment to rules and rituals, (4) sensual lust, (5) ill-will, (6) craving for fine-corporeal existence, (7) craving for incorporeal existence, (8) conceit, (9) restlessness, (10) ignorance.} he knows how the arising of the non-arisen fetter comes to be; he knows how the abandoning of the arisen fetter comes to be; and he knows how the non-arising in the future of the abandoned fetter comes to be.
\item He knows the ear and sounds... the nose and smells... the tongue and flavors... the body and tactual objects... the mind and mental objects, and the fetter that arises dependent on both; he knows how the arising of the non-arisen fetter comes to be; he knows how the abandoning of the arisen fetter comes to be; and he knows how the non-arising in the future of the abandoned fetter comes to be.
\pagebreak
\item Thus he lives contemplating mental objects in mental objects internally, or he lives contemplating mental objects in mental objects externally, or he lives contemplating mental objects in mental objects internally and externally. He lives contemplating origination factors in mental objects, or he lives contemplating dissolution factors in mental objects, or he lives contemplating origination-and-dissolution factors in mental objects.\footnote{Origination factors of the ten physical sense-bases are ignorance, craving, kamma, food, and the general characteristic of originating; dissolution factors: the general characteristic of dissolving and the disappearance of ignorance, etc. The origination-and-dissolution factors of the mind-base are the same as those of feeling.} Or his mindfulness is established with the thought, “Mental objects exist,” to the extent necessary just for knowledge and mindfulness, and he lives detached, and clings to nothing in the world. Thus, monks, a monk lives contemplating mental objects in the mental objects of the six internal and the six external sense-bases.
\end{enumerate}
\subsubsection*{Seven Factors of Enlightenment}
\begin{enumerate}[resume]
\item And further, monks, a monk lives contemplating mental objects in the mental objects of the seven factors of enlightenment.
\item How, monks, does a monk live contemplating mental objects in the mental objects of the seven factors of enlightenment?
\item Herein, monks, when the enlightenment-factor of mindfulness is present, the monk knows, “The enlightenment-factor of mindfulness is in me,” or when the enlightenment-factor of mindfulness is absent, he knows, “The enlightenment-factor of mindfulness is not in me;” and he knows how the arising of the non-arisen enlightenment-factor of mindfulness comes to be; and how perfection in the development of the arisen enlightenment-factor of mindfulness comes to be.
\item When the enlightenment-factor of the investigation of mental objects is present, the monk knows, “The enlightenment-factor of the investigation of mental objects is in me;” when the enlightenment-factor of the investigation of mental objects is absent, he knows, “The enlightenment-factor of the investigation of mental objects is not in me;” and he knows how the arising of the non-arisen enlightenment-factor of the investigation of mental objects comes to be, and how perfection in the development of the arisen enlightenment-factor of the investigation of mental objects comes to be.
\item When the enlightenment-factor of energy is present, he knows, “The enlightenment-factor of energy is in me;” when the enlightenment-factor of energy is absent, he knows, “The enlightenment-factor of energy is not in me;” and he knows how the arising of the non-arisen enlightenment-factor of energy comes to be, and how perfection in the development of the arisen enlightenment-factor of energy comes to be.
\item When the enlightenment-factor of joy is present, he knows, “The enlightenment-factor of joy is in me;” when the enlightenment-factor of joy is absent, he knows, “The enlightenment-factor of joy is not in me;” and he knows how the arising of the non-arisen enlightenment-factor of joy comes to be, and how perfection in the development of the arisen enlightenment-factor of joy comes to be.
\pagebreak
\item When the enlightenment-factor of tranquility is present, he knows, “The enlightenment-factor of tranquility is in me;” when the enlightenment-factor of tranquility is absent, he knows, “The enlightenment-factor of tranquility is not in me;” and he knows how the arising of the non-arisen enlightenment-factor of tranquility comes to be, and how perfection in the development of the arisen enlightenment-factor of tranquility comes to be.
\item When the enlightenment-factor of concentration is present, he knows, “The enlightenment-factor of concentration is in me;” when the enlightenment-factor of concentration is absent, he knows, “The enlightenment-factor of concentration is not in me;” and he knows how the arising of the non-arisen enlightenment-factor of concentration comes to be, and how perfection in the development of the arisen enlightenment-factor of concentration comes to be.
\item When the enlightenment-factor of equanimity is present, he knows, “The enlightenment-factor of equanimity is in me;” when the enlightenment-factor of equanimity is absent, he knows, “The enlightenment-factor of equanimity is not in me;” and he knows how the arising of the non-arisen enlightenment-factor of equanimity comes to be, and how perfection in the development of the arisen enlightenment-factor of equanimity comes to be.
\item Thus he lives contemplating mental objects in mental objects internally, or he lives contemplating mental objects in mental objects externally, or he lives contemplating mental objects in mental objects internally and externally. He lives contemplating origination-factors in mental objects, or he lives contemplating dissolution-factors in mental objects, or he lives contemplating origination-and-dissolution-factors in mental objects.\footnote{Just the conditions conducive to the origination and dissolution of the factors of enlightenment comprise the origination-and-dissolution factors here.} Or his mindfulness is established with the thought, “Mental objects exist,” to the extent necessary just for knowledge and mindfulness, and he lives detached, and clings to nothing in the world. Thus, monks, a monk lives contemplating mental objects in the mental objects of the seven factors of enlightenment.
\end{enumerate}
\subsubsection*{Four Noble Truths}
\begin{enumerate}[resume]
\item And further, monks, a monk lives contemplating mental objects in the mental objects of the four noble truths.
\item How, monks, does a monk live contemplating mental objects in the mental objects of the four noble truths?
\item Herein, monks, a monk knows, “This is suffering,” according to reality; he knows, “This is the origin of suffering,” according to reality; he knows, “This is the cessation of suffering,” according to reality; he knows “This is the road leading to the cessation of suffering,” according to reality.
\pagebreak
\item Thus he lives contemplating mental objects in mental objects internally, or he lives contemplating mental objects in mental objects externally, or he lives contemplating mental objects in mental objects internally and externally. He lives contemplating origination-factors in mental objects, or he lives contemplating dissolution-factors in mental objects, or he lives contemplating origination-and-dissolution-factors in mental objects.\footnote{The origination-and-dissolution factors of the truths should be understood as the arising and passing of suffering, craving, and the path; the truth of cessation is not to be included in this contemplation since it has neither origination nor dissolution.} Or his mindfulness is established with the thought, “Mental objects exist,” to the extent necessary just for knowledge and mindfulness, and he lives detached, and clings to nothing in the world. Thus, monks, a monk lives contemplating mental objects in the mental objects of the four noble truths.
\item Verily, monks, whosoever practices these four foundations of mindfulness in this manner for seven years, then one of these two fruits may be expected by him: highest knowledge (arahantship) here and now, or if some remainder of clinging is yet present, the state of non-returning.\footnote{That is, the non-returning to the world of sensuality. This is the last stage before the attainment of the final goal of arahantship.}
\item O monks, let alone seven years. Should any person practice these four foundations of mindfulness in this manner for six years... five years... four years... three years... two years... one year, then one of these two fruits may be expected by him: highest knowledge here and now, or if some remainder of clinging is yet present, the state of non-returning.
\item O monks, let alone a year. Should any person practice these four foundations of mindfulness in this manner for seven months... six months... five months... four months... three months... two months... a month... half a month, then one of these two fruits may be expected by him: highest knowledge here and now, or if some remainder of clinging is yet present, the state of non-returning.
\item O monks, let alone half a month. Should any person practice these four foundations of mindfulness in this manner for a week, then one of these two fruits may be expected by him: highest knowledge here and now, or if some remainder of clinging is yet present, the state of non-returning.
\item Because of this it was said: “This is the only way, monks, for the purification of beings, for the overcoming of sorrow and lamentation, for the destruction of suffering and grief, for reaching the right path, for the attainment of Nibbāna, namely the four foundations of mindfulness.”
\item Thus spoke the Blessed One. Satisfied, the monks approved of his words.

\end{enumerate}