\section{Realms of Existence}

Welcome to the sixth lesson of this Practical Abhidhamma Course in which we will discuss the Realms of Existence.\footnote{\url{http://en.wikipedia.org/wiki/Buddhist_cosmology_of_the_Theravada_school},\linebreak “The Four Planes of Existence in Theravāda Buddhism:” \url{http://www.bps.lk/olib/wh/wh462.pdf}, See Chapter 5 of “A Comprehensive Manual of Abhidhamma” (see Footnote 2 for link).}

Some Buddhists, particularly Western Buddhists, find it difficult to accept the details regarding Realms of Existence.  To them, Realms of Existence are the stuff of legend rather than something that can be verified through personal experience. I have been asked, “As a Buddhist, do I need to believe in Realms of Existence?” My answer is, “Belief in rebirth is not negotiable. In my opinion, belief in details regarding Realms of Existence is optional because these details do not impact doctrine or practice.”

According to the Abhidhamma, the external universe is an outer reflection of the internal cosmos of the mind.\footnote{Quote from page 188 of “A Comprehensive Manual of Abhidhamma" (see Footnote 2 for link).} When the mind has \textbf{Aversion}, it is like burning in Hell. When the mind has \textbf{Delusion}, it is like an animal driven by instincts. When the mind has by \textbf{Attachment}, it is like being a hungry ghost. Similarly, a wholesome mind is like a \textit{Deva} and a mind in jhāna is like being in the Fine Material Plane or Immaterial Plane.

\begin{figure}[H]
\centering
\renewcommand{\arraystretch}{1.1}
\setlength{\tabcolsep}{0mm}
\noindent\begin{tabular}{C{0.1\textwidth} C{0.2\textwidth} C{0.23\textwidth} L{0.28\textwidth}}
\toprule
\tableheader{Realm} & \multicolumn{2}{c}{\tableheader{Plane}} & \\
\midrule
\textit{28}--\textit{31} & \multicolumn{2}{c}{Immaterial Plane} & Immaterial jhānas \\
\cmidrule{2-4}
\textit{21}--\textit{27} & \multicolumn{2}{c}{\multirow{4}{0.43\textwidth}{\centering Fine Material Plane}} & Brahma (5\textsuperscript{th} jhāna)\\
\textit{18}--\textit{20} & & & Brahma (4\textsuperscript{th} jhāna)\\
\textit{15}--\textit{17} & & & Brahma (2\textsuperscript{nd} \& 3\textsuperscript{rd} jhāna)\\
\textit{12}--\textit{14} & & & Brahma (1\textsuperscript{st} jhāna)\\
\cmidrule{2-4}
\textit{6}--\textit{11} & \multirow{3} {0.2\textwidth} {\centering Sensuous Plane} & \multirow{2} {0.23\textwidth} {Happy Destinations} & \textit{Deva}\\
\textit{5} & & & Human\\
\cmidrule{3-4}
\textit{1}--\textit{4} & & Woeful States & Hell, Animal, \textit{Peta}, \textit{Asura}\\
\bottomrule
\end{tabular}
\caption[]{Overview of the 31 realms of existence.\footnotemark}
\label{Realms}
\end{figure}

\footnotetext{The Pure Abodes are Realms \textit{23}--\textit{27}.}

The Suttas list three types of beings: beings in the sensuous plane, beings in the fine material plane and beings in the immaterial plane.\footnote{MN 9: \url{http://www.accesstoinsight.org/tipitaka/mn/mn.009.ntbb.html\#bhava}} Beings in the sensuous planes are generally attached to sense objects\footnote{If a being in a sensuous plane becomes an Anāgāmī or Arahat, he will no longer be attached to sense objects.} whereas beings in other planes spend most of their time in jhāna.

\pagebreak

\subsection*{Woeful States}

\begin{figure}[H]
\centering
\renewcommand{\arraystretch}{1.1}
\setlength{\tabcolsep}{0mm}
\noindent\begin{tabular}{L{0.05\textwidth} C{0.2\textwidth} C{0.22\textwidth} C{0.15\textwidth} C{0.2\textwidth} C{0.09\textwidth} C{0.09\textwidth}}
\toprule
 & \multirow{2}{.2\textwidth}{\centering \vspace{-5mm}\tableheader{Name}}
 & \multirow{2}{0.22\textwidth}{\centering \raisebox{-5mm}{\parbox{.22\textwidth}{\centering \tableheader{Cause of rebirth\newline into this Realm}}}}
 & \multirow{2}{0.15\textwidth}{\centering \raisebox{-5mm}{\parbox{.15\textwidth}{\centering \tableheader{Life-continuum}}}}
 & \multirow{2}{0.18\textwidth}{\centering \vspace{-5mm}\tableheader{Lifespan}}
 & \multicolumn{2}{c}{\centering \tableheader{Destination}}
 \\
 & & & & & \tablesubheader{Non-saints} & \tablesubheader{Saints}
 \\
\midrule
\textit{4}
& \textit{Asura} & -- & \multirow{4}{0.15\textwidth}{\raisebox{-22mm}{\parbox{0.15\textwidth}{\centering \textbf{19}}}}
& \multirow{4}{0.18\textwidth}{\raisebox{-22mm}{\parbox{0.18\textwidth}{\centering Indefinite}}}
& \multirow{4}{0.09\textwidth}{\raisebox{-22mm}{\parbox{0.09\textwidth}{\centering \textit{1}--\textit{11}}}}
& \multirow{4}{0.09\textwidth}{\raisebox{-26mm}{\parbox{0.09\textwidth}{\centering --}}}
\\[6mm]
\textit{3} & \textit{Peta}\newline (Hungry Ghosts) & -- & & & &
\\[6mm]
\textit{2} & Animal & Behaving like\newline an animal & & & &
\\[6mm]
\textit{1} & Hell & Five heinous deeds & & & &
\\[6mm]
\bottomrule
\end{tabular}

\caption{Realms 1--4, the Woeful States. For all Woeful States, one cause of rebirth into this Realm is ``completed'' unwholesome kamma.}
\label{fig:Woeful1}
\end{figure}

The Woeful States include four realms: Hell, Animal, \textit{Peta} and \textit{Asura}. One cause of rebirth common to all of the Woeful States is “completed” unwholesome kamma from a previous existence. There are many types of unwholesome kamma but the Suttas identify those that can cause rebirth in the Woeful States as killing, stealing, sexual misconduct, lying, divisive speech, abusive speech, idle talk, covetousness, ill will and \textbf{Wrong view}.\footnote{See AN 10.177: \url{http://www.accesstoinsight.org/tipitaka/an/an10/an10.177.than.html} and MN 41: \url{http://www.accesstoinsight.org/tipitaka/mn/mn.041.nymo.html}}

The Abhidhamma Commentary lists conditions that must be met for kamma to be “completed,” to be sufficiently weighty to be able to cause rebirth in the Woeful States.\footnote{Details can be found in the Atthasālinī, pages 128--134.} For example, for killing to be a “completed” kamma, there must be life, knowledge of that life, intent to kill, effort to kill and consequential death. We will discuss the details during our lesson on kamma.

The life-continuum Mind Moment for all beings in the Woeful States is \textbf{19}. A being’s lifespan in the Woeful State depends on the weightiness of their kamma. Beings with weightier kamma will have a longer lifespan in the Woeful State. 

After their lifespan in a Woeful state is over, beings will be reborn into realm \textit{1} to realm \textit{11}. There are stories of an animal reborn as a \textit{Deva},\footnote{In \textit{Vimānavatthu} 852--88, a frog dies listening to the Buddha’s voice and is reborn into realm \textit{7}.} but most of the time, beings in the Woeful States are reborn back into the Woeful States because while in a Woeful State, the mind is consumed by \textbf{Attachment}, \textbf{Aversion} and \textbf{Delusion}, and these thoughts create more unwholesome kamma. Saints are never reborn into the Woeful States and it is not possible to become a saint while in a Woeful State. Figure \ref{fig:Woeful1} indicates this by leaving the column blank.

\pagebreak

\begin{figure}[H]
\centering
\vspace{2mm}
\input{./Diagrams/Hell.pdf_tex}
\caption{Structure of Hells according to the Commentary. See footnote for details.}
\label{fig:Hells}
\end{figure}

Let's discuss each of the realms, starting with Hell.\footnote{\url{http://en.wikipedia.org/wiki/Naraka_(Buddhism)}} According to the Suttas, there are five heinous deeds that guarantee a rebirth in Hell in the next life: killing one’s mother, killing one’s father, killing an Arahat, wounding a Buddha and causing a split in the Sangha.\footnote{AN 5.29: \url{http://www.accesstoinsight.org/tipitaka/an/an05/an05.129.than.html} \\ Performing any of these five heinous deeds makes it impossible to attain sainthood in that same lifetime and results in rebirth in the worst (\textit{Avīci}) Hell in the next existence.} Hell-beings are subject to painful suffering and according to the Commentary, there are multiple levels of Hell. \footnote{According to the Commentary, there are eight great Hells of increasing intensity of torment. 
\begin{itemize}[noitemsep,topsep=0pt]
\item \textit{Sañjīva}: The least severe type of Hell where Hell-guardians cut the Hell-beings with glowing weapons.
\item \textit{Kāḷasutta}: Hell-beings are placed on a floor of heated iron, marked with a black thread. Hell-guardians plane the Hell-beings with adzes along the markings.
\item \textit{Sanghāta}: Hell-beings are constantly being crushed by huge fiery rocks.
\item \textit{Roruva}: Hell-beings have noxious gases blown into their bodies.
\item \textit{Mahā Roruva}: Hell-beings have flames blown into their bodies.
\item \textit{Tāpana}: Hell-beings are pierced by red-hot stakes and remain motionless.
\item \textit{Mahā Tāpana}: Hell-guardians force the Hell-beings to climb up a burning iron mountain. Strong winds force the Hell-beings to fall from the mountain and be impaled on the red-hot stakes below.
\item \textit{Avīci}: The worst type of Hell for those who have committed the five heinous acts. There is no space between the Hell-beings and the flames. 
\end{itemize}
Each of the eight great Hells is square with a door on each side. Each door leads to five minor Hells for a total of 168 Hells. The five minor Hells are \textit{Gūtha} (excrement) Hell, \textit{Kukkuḷa} (ember) Hell, \textit{Sim Pālivana} (silk-cotton tree) Hell, \textit{Asipattavana} (sword-leafed forest) Hell and \textit{Vettarani} (river of caustic water) Hell.} The Hell-being experiences a series of increasingly nasty Hells until the unwholesome kamma that caused rebirth in Hell has exhausted its result. 

The Buddha explained that when a being arrives in Hell or moves between Hells, he is met by a compassionate judge\footnote{King Yama: \url{http://en.wikipedia.org/wiki/Yama_(East_Asia)}} who asks, “Did you not see a baby, an old person, a sick person, a condemned person, a dead person? Did these sights not create in you a sense of urgency to do good?” to create a sense of spiritual urgency to generate wholesome kamma and reduce the time that the Hell-being must spend in Hell.\footnote{MN 130: \url{http://www.accesstoinsight.org/tipitaka/mn/mn.130.than.html}}

\pagebreak

Realm \textit{2} is the animal kingdom. Based on the kamma that caused rebirth, some animals such as household pets have a relatively easy life, and some animals have a difficult life. The Buddha explained that behaving like an animal will lead to rebirth in the animal realm, and the belief that behaving like an animal could lead to a fortunate rebirth is a \textbf{Wrong view} that could lead to rebirth in Hell.\footnote{MN 57: \url{http://www.accesstoinsight.org/tipitaka/mn/mn.057.nymo.html}}

Realm \textit{3} is the \textit{Peta} realm.\footnote{\url{http://en.wikipedia.org/wiki/Preta}} \textit{Peta} are hungry ghosts who coexist with us in the human realm but are invisible to most people. The \textit{Tipiṭaka} includes a book dedicated to stories of \textit{Peta} and the kamma that resulted in their unfortunate rebirth into Realm \textit{3}.\footnote{Petavatthu: \url{http://en.wikipedia.org/wiki/Petavatthu}} One of the Suttas in this book explains that there are \textit{Peta} who depend on food and drink offered by relatives living in the human realm.\footnote{Pv 1.5: \url{http://www.accesstoinsight.org/tipitaka/kn/pv/pv.1.05.than.html}} The Buddha explained that only deceased relatives who have been reborn into the Peta realm are able to receive offerings dedicated to them.\footnote{AN 10.177: \url{http://www.accesstoinsight.org/tipitaka/an/an10/an10.177.than.html}} There are also other types of \textit{Peta} that are unable to receive offerings; they always suffer from hunger and thirst.\footnote{According to the Commentary, there are four kinds of \textit{Peta}: 

\begin{itemize}[nosep]
\item \textit{Paradattupajīvika-peta}: This sort of \textit{Peta} lives on the \textit{dakkhiṇā} (sacrificial gifts) of relatives. Buddhism encourages merit-making by offering food, clothing, shelter, etc. to virtuous persons such as monastics and then dedicating the merit acquired to deceased relatives. The \textit{dakkhiṇā} will become fruitful to the \textit{Peta} when three conditions are met:
%\begin{itemize}[noitemsep,topsep=0pt]
\begin{itemize}[nosep]
\item The \textit{dakkhiṇā} is given to a virtuous person
\item The merit of the \textit{dakkhiṇā} is dedicated to a deceased relative
\item His deceased relative has been born as a \textit{Paradattupajīvika-peta}
\end{itemize}
\item [ ]If any of these three conditions is lacking, the deceased relative will not be able to enjoy the outcome of the \textit{dakkhiṇā}, but the performer of the \textit{dakkhiṇā} will still receive the good result of his meritorious action.
\item \textit{Khuppipsika-peta}: This sort of \textit{Peta} suffers from hunger and thirst and will suffer as a \textit{Peta} as long as his evil kamma lasts.
\item \textit{Nijjhāmataõhikā-peta}: This \textit{Peta’s} suffering is caused by his own craving (\textit{taṇhā}). Fire burns in his mouth as long as his evil kamma lasts.
\item \textit{Kālakancika-peta}: This sort of \textit{Peta} has a very tall body that appears to be like a dry leaf with only skin covering the skeleton. His eyes protrude like those of the crab and his mouth is extremely small. He suffers from hunger and thirst like other types of \textit{Peta}. 
\vspace{-4mm}
\end{itemize}
}

Next is the \textit{Asura} realm.\footnote{\url{http://en.wikipedia.org/wiki/Asura_(Buddhism)}} The Suttas do not include the \textit{Asura} realm when listing realms; the \textit{Asura} realm was added by the Commentaries.\footnote{MN 97: \url{http://www.accesstoinsight.org/tipitaka/mn/mn.097.than.html}. \textit{Asura} realm is mentioned in Visuddhimagga XIII.93 (see Footnote 2 for link).} In the Suttas and Commentaries, the term \textit{Asura}, is sometimes translated as ``Titan" and is applied to three types of beings: a rebellious group of \textit{Deva},\footnote{AN 9.39: \url{http://www.accesstoinsight.org/tipitaka/an/an09/an09.039.than.html}. The leader of the \textit{Asuras} is \textit{Vepacitti} (\url{http://www.accesstoinsight.org/tipitaka/sn/sn35/sn35.207.than.html\#vepacitti}).} a type of \textit{Peta},\footnote{\textit{Kālakancika-peta} described above.} and a type of Hell-being.\footnote{These \textit{Asura} live in the \textit{Lokantarika-niraya} realm, situated between the human world, the hell world and the heaven world. It is a dark sea of acid water surrounded by rocky mountains; no light can reach this place. The Asura hang themselves on cliffs like bats. They are tortured by hunger and thirst as there is no food for them. While moving along the cliff they sometimes come across each other. Thinking that they have come across food, they jump upon each other and start fighting. As soon as they start fighting, they let loose their grip on the cliff and as a result they fall into the sea below and their bodies melt away just like salt melting away in water.}

\pagebreak

\pagebreak

\begin{figure}[H]
\centering

\setlength{\tabcolsep}{0pt}
\renewcommand{\arraystretch}{1.1}
\noindent\begin{tabular}{C{\dimexpr0.5\textwidth-2\tabcolsep} C{\dimexpr0.5\textwidth-2\tabcolsep}}

\noindent\begin{tabular}{p{.26\textwidth}
R{.077\textwidth} |
p{.039\textwidth}}
\toprule
\multirow{8}{.28\textwidth}{\tablesubheader{\textbf{Attachment}-rooted}} & \hpadright{\textbf{1}} & \tmcommon
\\
& \hpadright{\textbf{2}} & \tmcommon
\\
& \hpadright{\textbf{3}} & \tmcommon
\\
& \hpadright{\textbf{4}} & \tmcommon
\\
& \hpadright{\textbf{5}} & \tmcommon
\\
& \hpadright{\textbf{6}} & \tmcommon
\\
& \hpadright{\textbf{7}} & \tmcommon
\\
& \hpadright{\textbf{8}} & \tmcommon
\\
\midrule
\multirow{2}{.28\textwidth}{\tablesubheader{\textbf{Aversion}-rooted}} & \hpadright{\textbf{9}} & \tmcommon
\\
& \hpadright{\textbf{10}} & \tmcommon
\\
\midrule
\multirow{2}{.28\textwidth}{\tablesubheader{\textbf{Delusion}-rooted}} & \hpadright{\textbf{11}} & \tmcommon
\\
& \hpadright{\textbf{12}} & \tmcommon
\\
\midrule
\multirow{7}{.28\textwidth}{\tablesubheader{Past unwholesome\linebreak resultant}} & \hpadright{\textbf{13}} & \tm
\\
& \hpadright{\textbf{14}} & \tm
\\
& \hpadright{\textbf{15}} & \tm
\\
& \hpadright{\textbf{16}} & \tm
\\
& \hpadright{\textbf{17}} & \tm
\\
& \hpadright{\textbf{18}} & \tm
\\
& \hpadright{\textbf{19}} & \tm
\\
\midrule
\multirow{8}{.28\textwidth}{\tablesubheader{Past wholesome\linebreak resultant}} & \hpadright{\textbf{20}} & \tm
\\
& \hpadright{\textbf{21}} & \tm
\\
& \hpadright{\textbf{22}} & \tm
\\
& \hpadright{\textbf{23}} & \tm
\\
& \hpadright{\textbf{24}} & \tm
\\
& \hpadright{\textbf{25}} & \tm
\\
& \hpadright{\textbf{26}} & \tm
\\
& \hpadright{\textbf{27}} & \tm
\\
\midrule
\multirow{3}{.28\textwidth}{\tablesubheader{Functional}} & \hpadright{\textbf{28}} & \tm
\\
& \hpadright{\textbf{29}} & \tm
\\
& \hpadright{\textbf{30}} & 
\\
\bottomrule
\end{tabular}

&
\vspace{34.5mm}
\begin{tabular}{p{.05\textwidth} p{.05\textwidth}
p{0.16\textwidth}
R{.077\textwidth} |
p{.039\textwidth}}
\toprule
\multicolumn{2}{C{0.1\textwidth}}{\multirow{10}{.1\textwidth}{\vspace{-8mm}\rotatebox[origin=l]{90}{\tableheader{Sense Sphere}}}} & \tablesubheader{Wholesome} & \hpadright{\textbf{31}--\textbf{38}} & \tm
\\
\cmidrule{3-5}
& & \multirow{8}{.26\textwidth}{\tablesubheader{Resultant}} & \hpadright{\textbf{39}} & 
\\
& & & \hpadright{\textbf{40}} &
\\
& & & \hpadright{\textbf{41}} &
\\
& & & \hpadright{\textbf{42}} &
\\
& & & \hpadright{\textbf{43}} &
\\
& & & \hpadright{\textbf{44}} &
\\
& & & \hpadright{\textbf{45}} &
\\
& & & \hpadright{\textbf{46}} &
\\
\cmidrule{3-5}
& & \tablesubheader{Functional} & \hpadright{\textbf{47}--\textbf{54}} &
\\
\midrule
\multicolumn{2}{C{0.1\textwidth}}{\multirow{3}{0.1\textwidth}{\rotatebox[]{90}{\parbox{17mm}{\centering \tableheader{Fine\\ Material\\ Sphere}}}}} & \tablesubheader{Wholesome} & \hpadright{\textbf{55}--\textbf{59}} &
\\
& & \tablesubheader{Resultant} & \hpadright{\textbf{60}--\textbf{64}} &
\\
& & \tablesubheader{Functional} & \hpadright{\textbf{65}--\textbf{69}} &
\\\midrule
\multicolumn{2}{C{0.1\textwidth}}{\multirow{3}{0.1\textwidth}{\rotatebox[]{90}{\parbox{17mm}{\centering \tableheader{Im-\\ material\\ Sphere}}}}} & \tablesubheader{Wholesome} & \hpadright{\textbf{70}--\textbf{73}} & 
\\
& & \tablesubheader{Resultant} & \hpadright{\textbf{74}--\textbf{77}} & 
\\
& & \tablesubheader{Functional} & \hpadright{\textbf{78}--\textbf{81}} & 
\\\midrule
\multirow{8}{.05\textwidth}{\rotatebox[]{90}{\tableheader{Supramundane}}} & \multirow{4}{.05\textwidth}{\rotatebox[]{90}{\tablesubheader{Path}}} & \tablesubheader{Sotāpanna} & \hpadright{\textbf{82}} & 
\\
& & \tablesubheader{Sakadāgāmī} & \hpadright{\textbf{83}} & 
\\
& & \tablesubheader{Anāgāmī} & \hpadright{\textbf{84}} & 
\\
& & \tablesubheader{Arahat} & \hpadright{\textbf{85}} & 
\\\cmidrule{2-5}
& \multirow{4}{.05\textwidth}{\rotatebox[]{90}{\tablesubheader{Fruit}}} & \tablesubheader{Sotāpanna} & \hpadright{\textbf{86}} &
\\
& & \tablesubheader{Sakadāgāmī} & \hpadright{\textbf{87}} & 
\\
& & \tablesubheader{Anāgāmī} & \hpadright{\textbf{88}} & 
\\
& & \tablesubheader{Arahat} & \hpadright{\textbf{89}} & 
\\
\bottomrule
\end{tabular}
\\
\end{tabular}

\begin{center}
\tmcommon\hspace{2mm} Common kamma-creating Mind Moment \hspace{5mm} \tm\hspace{2mm} Mind Moment that can arise
\end{center}

\caption{Mind Moments in the Woeful States; the common kamma-creating Mind Moments are \textbf{1}--\textbf{12}.}
\label{fig:Woeful}
\end{figure}

Figure \ref{fig:Woeful} shows that for beings in the Woeful States, only Mind Moments \textbf{1}--\textbf{29} and Mind Moments \textbf{31}--\textbf{38} can arise. The most common kamma-creating Mind Moments for beings in Woeful States are \textbf{1}--\textbf{12}, the Danger Zone. Jhāna and supramundane Mind Moments are not possible for beings in the Woeful States.

\pagebreak

\subsection*{Happy Destinations}

\begin{figure}[H]
\centering

\setlength{\tabcolsep}{0pt}
\renewcommand{\arraystretch}{1.1}

\noindent\begin{tabular}{L{0.05\textwidth} C{.2\textwidth} C{0.25\textwidth} C{0.1\textwidth} C{0.15\textwidth} C{0.17\textwidth} C{0.08\textwidth}}
\toprule
 & \multirow{2}{.2\textwidth}{\centering \tableheader{Name}}
 & \multirow{2}{0.25\textwidth}{\centering \tableheader{Cause of rebirth\newline into this Realm}}
 & \multirow{2}{0.1\textwidth}{\centering \tableheader{Life-cont.}}
 & \multirow{2}{0.15\textwidth}{\centering \tableheader{Lifespan}}
 & \multicolumn{2}{c}{\centering \tableheader{Destination}}
 \\
 & & & & & \tablesubheader{Non-saints} & \tablesubheader{Saints}
 \\
\midrule
\textit{11}
 & Gods wielding power of creation of others & \multirow{7}{0.25\textwidth}{\hspace{.01\textwidth}\parbox{0.23\textwidth}{\centering
\vspace{-4mm}3-rooted superior kamma \textrightarrow \newline
being with life-continuum of\newline
\textbf{39}, \textbf{40}, \textbf{43}, \textbf{44}\newline (3 roots)\newline \vspace{5mm}
3-rooted superior kamma or 2-rooted superior kamma \textrightarrow\newline
being with life-continuum of\newline
\textbf{41}, \textbf{42}, \textbf{45}, \textbf{46}\newline (2 roots)\newline \vspace{5mm}
2-rooted inferior kamma \textrightarrow \newline
being with life-continuum of\newline
\textbf{27} (0 roots)
}}
 & \multirow{5}{0.1\textwidth}{\raisebox{-29mm}{\parbox{0.1\textwidth}{\centering \textbf{39}, \textbf{40}, \textbf{43}, \textbf{44}}}}
 & 9216\newline mil. years
 & \multirow{7}{0.15\textwidth}{\raisebox{-44mm}{\parbox{0.15\textwidth}{\centering
Being with life-continuum of\newline
\textbf{39}, \textbf{40}, \textbf{43}, \textbf{44} \textrightarrow \newline \textit{1}--\textit{22}, \textit{28}--\textit{31}\newline \vspace{5mm}
Being with life-continuum of\newline
\textbf{27}, \textbf{41}, \textbf{42}, \textbf{45}, \textbf{46} \textrightarrow \newline
\textit{1}--\textit{11}
}}}
 & \multirow{7}{0.1\textwidth}{\raisebox{-48mm}{\parbox{0.1\textwidth}{\centering \textit{5}--\textit{21} \textit{23}--\textit{31}}}}
\\[9mm]
\textit{10} & Gods delighting in creation & & & 2304\newline mil. years & &
\\[9mm]
\textit{9} & Heaven of the contented Gods & & & 576\newline mil. years & &
\\[9mm]
\textit{8} & Heaven of the Yāma Gods & & & 144\newline mil. years & &
\\[9mm]
\textit{7} & Heaven of the 33 Gods & & & 36\newline mil. years & &
\\[9mm]\cmidrule{4-4}
\textit{6} & Heaven of the Four Great Kings & & \multirow{2}{0.1\textwidth}{\raisebox{-8mm}{\parbox{0.1\textwidth}{\centering \textbf{27}, \textbf{39}--\textbf{46}}}} & 9 mil.\newline years / Indefinite & &
\\[9mm]
\textit{5} & Human & & & Indefinite & &
\\[9mm]
\bottomrule
\end{tabular}

\caption{Realms \textit{5}--\textit{11}, the Happy Destinations. Realm \textit{5} is the Human Realm and Realms \textit{6}--\textit{11} are the \textit{Deva} Realms.}
\label{fig:Happy1}
\end{figure}

The Happy Destinations include seven realms; the human realm and six \textit{Deva} realms.

If the rebirth-linking kamma from the previous existence is “3-rooted superior kamma,” then the life-continuum Mind Moment in the Happy Destinations will be one of \textbf{39}, \textbf{40}, \textbf{43} or \textbf{44}. As can be seen in Figure \ref{fig:Faultless}, these four life-continuum Mind Moments have three roots including the root of \textbf{Understanding}.

What is “3-rooted” kamma? As seen in Figure \ref{fig:Faultless}, kamma generated by Mind Moments \textbf{31}, \textbf{32}, \textbf{35} or \textbf{36} is 3-rooted because these Mind Moments are associated with \textbf{Understanding}. On the other hand, kamma generated by Mind Moments \textbf{33}, \textbf{34}, \textbf{37} or \textbf{38} is 2-rooted kamma because these Mind Moments are not associated with \textbf{Understanding}.

What differentiates “superior” kamma from “inferior” kamma are the Mind Moments shortly before and shortly after the wholesome kamma-creating Mind Moment. For a kamma to be “superior,” there must be a wholesome Mind Moment shortly before, and a wholesome Mind Moment shortly after, otherwise the kamma is “inferior.” 

\pagebreak

For example, if I make a donation reluctantly, unwholesome reluctance arises shortly before the donation and the kamma is “inferior.” If I make a donation and then regret it, unwholesome regret arises shortly after the donation and the kamma of the donation is “inferior.” If I prepare the donation with joy, donate and then share the merit of the donation, the donation is supported before and after by other wholesome Mind Moments and the kamma is “superior.” The Buddha encouraged his son to reflect before, during and after an action; the Buddha encouraged his son to create superior kamma.\footnote{MN 61: \url{http://www.accesstoinsight.org/tipitaka/mn/mn.061.than.html}}

If the rebirth-linking kamma from the previous existence is “3-rooted superior kamma,” then the life-continuum Mind Moment in the Happy Destinations will have three roots. If the rebirth-linking kamma from the previous existence is “3-rooted inferior kamma” or “2-rooted superior kamma,” then the life-continuum Mind Moment will have two roots. If the rebirth-linking kamma from the previous existence is “2-rooted inferior kamma,” then the life-continuum Mind Moment will have no roots.

Beings in Realms \textit{7}--\textit{11} will have 3-rooted life-continuum Mind Moments, while beings in the Human Realm or Realm \textit{6} may have life-continuum Mind Moments with 0, 2 or 3 roots. Beings with 3-rooted life-continuum Mind Moments can be reborn into the Sensuous Plane, the Fine Material Plane or the Immaterial Plane. Beings whose life-continuum Mind Moment has 0 or 2 roots will be reborn into the Sensuous Plane.

The lowest of the Happy Destinations is Realm \textit{5}, the Human realm. The Pāḷi word for this realm is “\textit{manussa}” which literally means “abundance of mind.” The Human realm is the perfect place for spiritual development. The minds of beings in the Woeful States are consumed with \textbf{Attachment}, \textbf{Aversion} and \textbf{Delusion}, so there is little opportunity for spiritual development. The minds of \textit{Deva} are occupied with sensual bliss, so there is little motivation for spiritual development. In the Human realm, suffering, sickness, old age and death can create a sense of spiritual urgency. At the same time, the Human realm has joy and happiness, the teachings of the Buddha are available and exalted states of mind are possible.

The Buddha said, “Imagine the whole world was an ocean and a single piece of wood with a hole was floating on the surface. There is a blind turtle in the ocean that comes to the surface once every 100 years. Is it likely that the turtle would put its head through the hole in the piece of wood?” The monks replied, “It would be a very rare event.” The Buddha replied, “It is also rare event to have a human rebirth at a time that there is a Buddha and the Dhamma shines brightly. Therefore, monks, an exertion should be made to understand the Four Noble Truths!”\footnote{SN 56.48: \url{http://www.accesstoinsight.org/tipitaka/sn/sn56/sn56.048.than.html}}

Realms \textit{6}--\textit{11} are \textit{Deva} realms. Those who can see \textit{Devas} describe them as brightly shining. The Buddha encouraged “\textit{Devatānussati},” the practice of the “recollection of \textit{Devas}.”\footnote{AN 1.301, DN 33, DN 34, AN 6.9, AN 6.10, AN 6.25 (this Sutta explains \textit{Devatānussati}), etc.} This is not worshipping \textit{Devas}, but rather reviewing the faith, virtuous behaviour, learning, generosity and wisdom of the meditator, and reflecting how these same qualities caused \textit{Devas} to be reborn into \textit{Deva} realms. The Buddha said that \textit{Devas}, along with parents, family, customers and ascetics are worthy of respect.\footnote{AN 5.58: \url{http://suttacentral.net/en/an5.58}} The Ratana Sutta\footnote{Sn 2.1: \url{http://www.accesstoinsight.org/tipitaka/kn/snp/snp.2.01.piya.html}} is directed to the \textit{Devas}; the Buddha asks the \textit{Devas} to protect human beings because human beings share merit with the \textit{Devas}. Whereas \textit{Petas} depend on relatives to share food and drink, \textit{Devas} are happy when they see humans performing meritorious deeds.

\pagebreak

Realm \textit{6}, Heaven of Four Great Kings,\footnote{DN 32: \url{http://www.accesstoinsight.org/tipitaka/dn/dn.32.0.piya.html}\newline See also \url{http://en.wikipedia.org/wiki/Four_Heavenly_Kings}} is called \textit{Catumahārājika} and has four divisions, each ruled by its own guardian deity and inhabited by a different class of demi-Gods.\footnote{East: \textit{Dhataraṭṭha}, South: \textit{Virūḷhaka}, West: \textit{Virūpakkha}, North: \textit{Vessavaṇa}/\textit{Kuvera}.} To the East, there are celestial musicians.\footnote{\textit{Gandhabbas}: \url{http://en.wikipedia.org/wiki/Gandharva}} To the South, there are gnomes who take care of forests, mountains and hidden treasures.\footnote{\textit{Kumbhaṇḍas}: \url{http://en.wikipedia.org/wiki/Kumbhanda}} To the West, there are \textit{Nāgas}, dragon-like creatures.\footnote{\url{http://en.wikipedia.org/wiki/Naga}} To the North, there are \textit{Yakkhas}.\footnote{\url{http://en.wikipedia.org/wiki/Yaksha}} The Pāḷi Text Society Dictionary defines a \textit{Yakkha} as a “non-human being (ogre, ghost) that sometimes helps and sometimes hinders humans.” Some modern scholars believe that \textit{Yakkhas} were actually humans, members of displaced aboriginal tribes who lived outside Indian society.\footnote{See entry in volume 8 of “Encyclopaedia of Buddhism,” published by the Government of Sri Lanka.}

Some of the \textit{Devas} in Realm \textit{6} are earthbound and live on mountains, in pagodas and in public houses like temples. Some of the \textit{Devas} in Realm \textit{6} live on top of trees; when their trees are chopped down they have to shift to unoccupied ones. The \textit{Devas} from Realm 6 who live on the earth and on trees have an indefinite lifespan. The \textit{Devas} in Realm \textit{6} who have mansions in the sky have a lifespan of 9 million years. The \textit{Tipiṭaka} includes a book dedicated to stories of the heavenly mansions of \textit{Devas} in Realm \textit{6} and Realm \textit{7}, and the kamma that resulted in the rebirth into these realms.\footnote{\textit{Vimānavatthu}: \url{http://en.wikipedia.org/wiki/Vimanavatthu}. The grandeur of the mansion described in the \textit{Vimānavatthu} depends on the kamma of the owner.}

Realm \textit{7} is the Heaven of the 33 Gods; in Pāḷi, it is called \textit{Tāvatiṃsa}. Realm \textit{6} and Realm \textit{7} share a common space. Sakka is the King of Realm \textit{7}, and the four Kings of Realm \textit{6} are among Sakka’s retinue.\footnote{\url{http://en.wikipedia.org/wiki/Sakra_(Buddhism)} sometimes also called \textit{Indra}.} According to the Commentary, there was a group of thirty-three men who collectively dedicated their efforts to the happiness and well-being of other people. They passed their whole life with such wholesome actions that after death they were reborn into Realm \textit{7}.\footnote{\url{http://www.tipitaka.net/tipitaka/dhp/verseload.php?verse=030}}

Except for Realm \textit{6} and Realm \textit{7}, \textit{Devas} of higher realms are invisible to \textit{Devas} of lower realms, and \textit{Devas} cannot travel to realms higher than their own but can descend into a lower realm at will. The Buddha taught the Abhidhamma in \textit{Tāvatiṃsa} Heaven in gratitude to his mother. The Buddha chose \textit{Tāvatiṃsa} to teach the Abhidhamma because \textit{Tāvatiṃsa} is accessible to \textit{Devas} of all realms; lower as well as higher heavens. The Buddha wanted his sermon to benefit not only his mother, but also \textit{Devas} of other realms who could learn from his teachings.

The \textit{Devas} of Realm \textit{8} live without hardship, and the \textit{Devas} of Realm \textit{9} can always enjoy the pleasures of life. In Pāḷi, Realm \textit{9} is called \textit{Tusita} heaven. All Bodhisatta are reborn into the Human Realm from Realm \textit{9} to become a Buddha, and the Buddha’s mother, who died seven days after giving birth to the Buddha, was reborn into Realm \textit{9}. The \textit{Devas} of Realm \textit{10} use their minds to create objects of sense-pleasure. The \textit{Devas} of Realm \textit{11} don’t even have to bother creating objects of sense-pleasure; others create the objects of sense-pleasure for them and these \textit{Devas} simply enjoy them.

\subsubsection*{Māra}

Māra lives in Realm \textit{11}.\footnote{Māra literally means “the killer”, he is sometimes called \textit{Namuci}, literally “the opponent of liberation.” See entry for “\textit{Māra}” in “Buddhist Dictionary” (see Footnote 2 for link).} Māra tries to disrupt spiritual progress by causing distractions, either by interrupting the meditation of the Buddha or monks, or by interfering with the Buddha’s preaching.\footnote{The ten armies of Māra are listed in Sn 3.2: \url{http://www.accesstoinsight.org/tipitaka/kn/snp/snp.3.02.than.html}. See also “Letter from Māra," (\url{http://www.bps.lk/olib/wh/wh461.pdf}) a satirical, allegorical, yet profound essay, in which Māra writes a long letter to his ten armies with instructions on how to keep his subjects trapped in his realm of birth and death. Māra is portrayed as a modern CEO of \textit{Saṃsāra} who views his dominions through a computer monitor.} Once Māra is recognized, he has no choice but to retreat. In other words, when the light of \textbf{Understanding} is directed onto Māra, Māra loses his power and disappears.\footnote{SN 4.8: \url{http://www.accesstoinsight.org/tipitaka/sn/sn04/sn04.008.than.html}\\SN 4.13: \url{http://www.accesstoinsight.org/tipitaka/sn/sn04/sn04.013.than.html}\\SN 4.19: \url{http://www.accesstoinsight.org/tipitaka/sn/sn04/sn04.019.than.html}\\SN 4.20: \url{http://www.accesstoinsight.org/tipitaka/sn/sn04/sn04.020.than.html}}

The Commentaries speak of a “fivefold Māra:" Māra as a \textit{Deva}, Māra as a personification of the defilements, Māra as a personification of the aggregates, Māra as a personification of kammic formations and Māra as a personification of death. So according to the Commentaries, Māra is sometimes used as a literary device in the Suttas.

I believe that other supernatural beings were also sometimes used as a literary device in the Suttas. For example Sakka, the King of Realm \textit{7}, was a powerful \textit{Deva} who was highly respected by non-Buddhists. When the Suttas say that Sakka came to pay respect to the Buddha\footnote{\url{http://www.tipitaka.net/tipitaka/dhp/verseload.php?verse=206}} or asked questions to the Buddha,\footnote{DN 21: \url{http://www.accesstoinsight.org/tipitaka/dn/dn.21.2x.than.html}} this implied that the Buddha is superior to gods from other belief systems.

Figure \ref{fig:Happy} shows that beings in the Happy Destinations whose life-continuum Mind Moments have 0 or 2 roots cannot experience jhāna (Mind Moments \textbf{55}--\textbf{59}, \textbf{70}--\textbf{73}), nor can they attain sainthood (Mind Moment \textbf{82}), but 3-rooted beings can experience jhāna and attain sainthood. For non-saints, the commonly arising kamma-creating Mind Moments include both Mind Moments \textbf{1}--\textbf{12}, the Danger Zone and Mind Moments \textbf{31}--\textbf{38}, the Faultless Zone.

A Sotāpanna cannot experience Mind Moment \textbf{11}, which is associated with \textbf{Doubt}, or Mind Moments associated with \textbf{Wrong view} (Mind Moments \textbf{1}, \textbf{2}, \textbf{5} and \textbf{6}). \textbf{Wrong view} has been uprooted in a Sotāpanna. 

Reminds me of a joke: one becomes a Sotāpanna when your karma runs over your dogma. 

An Anāgāmī cannot experience \textbf{Aversion}-rooted Mind Moments (Mind Moments \textbf{9} and \textbf{10}) and an Arahat cannot experience any unwholesome Mind Moments. For saints, the commonly arising kamma-creating Mind Moments are \textbf{31}--\textbf{38}, the Faultless Zone.

\pagebreak

\begin{figure}[H]
\centering

\setlength{\tabcolsep}{0pt}
\renewcommand{\arraystretch}{1.1}

\noindent\begin{tabular}{p{.26\textwidth}
R{.077\textwidth} |
p{.039\textwidth}
p{.039\textwidth}
p{.039\textwidth}
p{.039\textwidth}
p{.039\textwidth}
p{.039\textwidth}}
\toprule
&
& \tablevsubheaderhack{0 / 2 Roots (Non-saint)}
& \tablevsubheaderhack{3 Roots (Non-saint)}
& \tablevsubheaderhack{Sotāpanna}
& \tablevsubheaderhack{Sakadāgāmī}
& \tablevsubheaderhack{Anāgāmī}
& \tablevsubheaderhack{Arahat}
\\
\midrule
\multirow{8}{.28\textwidth}{\tablesubheader{\textbf{Attachment}-rooted}} & \hpadright{\textbf{1}} & \tmcommon & \tmcommon & & & &
\\
& \hpadright{\textbf{2}} & \tmcommon & \tmcommon & & & &
\\
& \hpadright{\textbf{3}} & \tmcommon & \tmcommon & \tm & \tm & \tm &
\\
& \hpadright{\textbf{4}} & \tmcommon & \tmcommon & \tm & \tm & \tm &
\\
& \hpadright{\textbf{5}} & \tmcommon & \tmcommon & & & &
\\
& \hpadright{\textbf{6}} & \tmcommon & \tmcommon & & & &
\\
& \hpadright{\textbf{7}} & \tmcommon & \tmcommon & \tm & \tm & \tm &
\\
& \hpadright{\textbf{8}} & \tmcommon & \tmcommon & \tm & \tm & \tm &
\\
\midrule
\multirow{2}{.28\textwidth}{\tablesubheader{\textbf{Aversion}-rooted}} & \hpadright{\textbf{9}} & \tmcommon & \tmcommon & \tm & \tm & &
\\
& \hpadright{\textbf{10}} & \tmcommon & \tmcommon & & & &
\\
\midrule
\multirow{2}{.28\textwidth}{\tablesubheader{\textbf{Delusion}-rooted}} & \hpadright{\textbf{11}} & \tmcommon & \tmcommon & & & &
\\
& \hpadright{\textbf{12}} & \tmcommon & \tmcommon & \tm & \tm & \tm & 
\\
\midrule
\multirow{7}{.28\textwidth}{\tablesubheader{Past unwholesome\linebreak resultant}} & \hpadright{\textbf{13}} & \tm & \tm & \tm & \tm & \tm & \tm
\\
& \hpadright{\textbf{14}} & \tm & \tm & \tm & \tm & \tm & \tm
\\
& \hpadright{\textbf{15}} & \tm & \tm & \tm & \tm & \tm & \tm
\\
& \hpadright{\textbf{16}} & \tm & \tm & \tm & \tm & \tm & \tm
\\
& \hpadright{\textbf{17}} & \tm & \tm & \tm & \tm & \tm & \tm
\\
& \hpadright{\textbf{18}} & \tm & \tm & \tm & \tm & \tm & \tm
\\
& \hpadright{\textbf{19}} & \tm & \tm & \tm & \tm & \tm & \tm
\\
\midrule
\multirow{8}{.28\textwidth}{\tablesubheader{Past wholesome\linebreak resultant}} & \hpadright{\textbf{20}} & \tm & \tm & \tm & \tm & \tm & \tm
\\
& \hpadright{\textbf{21}} & \tm & \tm & \tm & \tm & \tm & \tm
\\
& \hpadright{\textbf{22}} & \tm & \tm & \tm & \tm & \tm & \tm
\\
& \hpadright{\textbf{23}} & \tm & \tm & \tm & \tm & \tm & \tm
\\
& \hpadright{\textbf{24}} & \tm & \tm & \tm & \tm & \tm & \tm
\\
& \hpadright{\textbf{25}} & \tm & \tm & \tm & \tm & \tm & \tm
\\
& \hpadright{\textbf{26}} & \tm & \tm & \tm & \tm & \tm & \tm
\\
& \hpadright{\textbf{27}} & \tm & \tm & \tm & \tm & \tm & \tm
\\
\midrule
\multirow{3}{.28\textwidth}{\tablesubheader{Functional}} & \hpadright{\textbf{28}} & \tm & \tm & \tm & \tm & \tm & \tm
\\
& \hpadright{\textbf{29}} & \tm & \tm & \tm & \tm & \tm & \tm
\\
& \hpadright{\textbf{30}} & & & & & & \tm
\\
\bottomrule
\end{tabular}
\caption{This figure is continued on the next page.}

\end{figure}

\clearpage

\begin{figure}[H]
\centering
\ContinuedFloat

\setlength{\tabcolsep}{0pt}
\renewcommand{\arraystretch}{1.1}

\noindent\begin{tabular}{p{.05\textwidth} p{.05\textwidth}
p{0.16\textwidth}
R{.077\textwidth} |
p{.039\textwidth}
p{.039\textwidth}
p{.039\textwidth}
p{.039\textwidth}
p{.039\textwidth}
p{.039\textwidth}}
\toprule
& &
&
& \tablevsubheaderhack{0 / 2 Roots (Non-saint)}
& \tablevsubheaderhack{3 Roots (Non-saint)}
& \tablevsubheaderhack{Sotāpanna}
& \tablevsubheaderhack{Sakadāgāmī}
& \tablevsubheaderhack{Anāgāmī}
& \tablevsubheaderhack{Arahat}
\\
\midrule
\multicolumn{2}{C{0.1\textwidth}}{\multirow{10}{.1\textwidth}{\vspace{-8mm}\rotatebox[origin=l]{90}{\tableheader{Sense Sphere}}}} & \tablesubheader{Wholesome} & \hpadright{\textbf{31}--\textbf{38}} & \tmcommon & \tmcommon & \tmcommon& \tmcommon& \tmcommon &
\\
\cmidrule{3-10}
& & \multirow{8}{.26\textwidth}{\tablesubheader{Resultant}} & \hpadright{\textbf{39}} & & \tm & \tm & \tm & \tm & \tm
\\
& & & \hpadright{\textbf{40}} & & \tm & \tm & \tm & \tm & \tm
\\
& & & \hpadright{\textbf{41}} & \tm & \tm & \tm & \tm & \tm & \tm
\\
& & & \hpadright{\textbf{42}} & \tm & \tm & \tm & \tm & \tm & \tm
\\
& & & \hpadright{\textbf{43}} & & \tm & \tm & \tm & \tm & \tm
\\
& & & \hpadright{\textbf{44}} & & \tm & \tm & \tm & \tm & \tm
\\
& & & \hpadright{\textbf{45}} & \tm & \tm & \tm & \tm & \tm & \tm
\\
& & & \hpadright{\textbf{46}} & \tm & \tm & \tm & \tm & \tm & \tm
\\
\cmidrule{3-10}
& & \tablesubheader{Functional} & \hpadright{\textbf{47}--\textbf{54}} & & & & & & \tm
\\
\midrule
\multicolumn{2}{C{0.1\textwidth}}{\multirow{3}{0.1\textwidth}{\rotatebox[]{90}{\parbox{17mm}{\centering \tableheader{Fine\\ Material\\ Sphere}}}}} & \tablesubheader{Wholesome} & \hpadright{\textbf{55}--\textbf{59}} & & \tm & \tm & \tm & \tm & 
\\
& & \tablesubheader{Resultant} & \hpadright{\textbf{60}--\textbf{64}} & & & & & &
\\
& & \tablesubheader{Functional} & \hpadright{\textbf{65}--\textbf{69}} & & & & & & \tm
\\\midrule
\multicolumn{2}{C{0.1\textwidth}}{\multirow{3}{0.1\textwidth}{\rotatebox[]{90}{\parbox{17mm}{\centering \tableheader{Im-\\ material\\ Sphere}}}}} & \tablesubheader{Wholesome} & \hpadright{\textbf{70}--\textbf{73}} & & \tm & \tm & \tm & \tm &
\\
& & \tablesubheader{Resultant} & \hpadright{\textbf{74}--\textbf{77}} & & & & & &
\\
& & \tablesubheader{Functional} & \hpadright{\textbf{78}--\textbf{81}} & & & & & & \tm
\\\midrule
\multirow{8}{.05\textwidth}{\rotatebox[]{90}{\tableheader{Supramundane}}} & \multirow{4}{.05\textwidth}{\rotatebox[]{90}{\tablesubheader{Path}}} & \tablesubheader{Sotāpanna} & \hpadright{\textbf{82}} & & \tm & & & &
\\
& & \tablesubheader{Sakadāgāmī} & \hpadright{\textbf{83}} & & & \tm & & &
\\
& & \tablesubheader{Anāgāmī} & \hpadright{\textbf{84}} & & & & \tm & &
\\
& & \tablesubheader{Arahat} & \hpadright{\textbf{85}} & & & & & \tm &
\\\cmidrule{2-10}
& \multirow{4}{.05\textwidth}{\rotatebox[]{90}{\tablesubheader{Fruit}}} & \tablesubheader{Sotāpanna} & \hpadright{\textbf{86}} & & & \tm & & &
\\
& & \tablesubheader{Sakadāgāmī} & \hpadright{\textbf{87}} & & & & \tm & &
\\
& & \tablesubheader{Anāgāmī} & \hpadright{\textbf{88}} & & & & & \tm &
\\
& & \tablesubheader{Arahat} & \hpadright{\textbf{89}} & & & & & & \tm
\\
\bottomrule
\end{tabular}
\begin{center}
\tmcommon\hspace{2mm} Common kamma-creating Mind Moment \hspace{5mm} \tm\hspace{2mm} Mind Moment that can arise
\end{center}
\caption{Mind Moments in the Happy Destinations; the common kamma-creating Mind Moments for non-saints (0, 2 or 3 roots) are \textbf{1}--\textbf{12} and \textbf{31}--\textbf{38}, while the common kamma-creating Mind Moments for saints are \textbf{31}--\textbf{38}.}
\label{fig:Happy}
\end{figure}

\subsection*{Fine Material Plane}

\begin{figure}[H]
\centering

\setlength{\tabcolsep}{0pt}
\renewcommand{\arraystretch}{1.1}

\noindent\begin{tabular}{L{0.05\textwidth} C{0.18\textwidth} C{0.13\textwidth} C{0.21\textwidth} C{0.1\textwidth} C{0.17\textwidth} C{0.08\textwidth} C{0.08\textwidth}}
\toprule
 & \raisebox{-8mm}{\parbox{0.18\textwidth}{\centering \tableheader{Name}}}
 & \raisebox{-8mm}{\parbox{0.13\textwidth}{\centering \tableheader{Jhāna}}}
 & \multirow{2}{0.21\textwidth}{\raisebox{-5mm}{\parbox{0.21\textwidth}{\centering \tableheader{Cause of rebirth into this Realm}}}}
 & \multirow{2}{0.1\textwidth}{\raisebox{-5mm}{\parbox{0.1\textwidth}{\centering \tableheader{Life-cont.}}}}
 & \multirow{2}{0.17\textwidth}{\raisebox{-4mm}{\parbox{0.17\textwidth}{\centering \tableheader{Lifespan}}}}
 & \multicolumn{2}{c}{\centering \tableheader{Destination}}
 \\
 & & & & & & \tablesubheader{Non-saints} & \tablesubheader{Saints}
 \\
\midrule
\textit{27} & Highest Pure Abode & \multirow{7}{0.13\textwidth}{\raisebox{-29mm}{\parbox{0.13\textwidth}{\centering 5\textsuperscript{th} Jhāna}}} & Understanding & \multirow{7}{0.1\textwidth}{\raisebox{-29mm}{\parbox{0.1\textwidth}{\centering \textbf{64}}}} & 16,000\newline Great aeons & - & -
\\
\textit{26} & Clear-sighted Abode & & Concentration & & 8,000\newline Great aeons & - & 27
\\
\textit{25} & Beautiful Abode & & Mindfulness & & 4,000\newline Great aeons & - & 26, 27
\\
\textit{24} & Serene Abode & & Energy & & 2,000\newline Great aeons & - & 2\textit{5}--\textit{27}
\\
\textit{23} & Durable Abode & & Faith & & 1,000\newline Great aeons & - & \textit{24}--\textit{27}
\\
\textit{22} & Gods without consciousness & & Dispassion to the mind & & 500\newline Great aeons & \textit{5}--\textit{11} & -
\\
\textit{21} & Gods of\newline great reward & & - & & 500\newline Great aeons & \textit{5}--\textit{22}, \textit{28}--\textit{31} & 21, \textit{23}--\textit{31}
\\\cmidrule{3-8}
\textit{20} & Gods of\newline steady aura & \multirow{3}{0.13\textwidth}{\raisebox{-8mm}{\parbox{0.13\textwidth}{\centering 4\textsuperscript{th} Jhāna}}} & Highest Degree & \multirow{3}{0.1\textwidth}{\raisebox{-9mm}{\parbox{0.1\textwidth}{\centering \textbf{63}}}} & 64\newline Great aeons & \textit{5}--\textit{22}, \textit{28}--\textit{31} & \textit{20}--\textit{21}, \textit{23}--\textit{31}
\\
\textit{19} & Gods of\newline infinite aura & & Medium Degree & & 32\newline Great aeons & \textit{5}--\textit{22}, \textit{28}--\textit{31} & \textit{19}--\textit{21}, \textit{23}--\textit{31}
\\
\textit{18} & Gods of\newline minor aura & & Minor Degree & & 16\newline Great aeons & \textit{5}--\textit{22}, \textit{28}--\textit{31} & \textit{18}--\textit{21}, \textit{23}--\textit{31}
\\\cmidrule{3-8}
\textit{17} & Gods of\newline radiant luster & \multirow{3}{0.13\textwidth}{\raisebox{-11mm}{\parbox{0.13\textwidth}{\centering 2\textsuperscript{nd} \& 3\textsuperscript{rd} Jhāna}}} & Highest Degree & \multirow{3}{0.1\textwidth}{\raisebox{-11mm}{\parbox{0.1\textwidth}{\centering 3\textsuperscript{rd} Jhāna\newline \textbf{62}\newline \vspace{5mm}2\textsuperscript{nd} Jhāna\newline \textbf{61}}}} & 8\newline Great aeons & \textit{5}--\textit{22}, \textit{28}--\textit{31} & \textit{17}--\textit{21}, \textit{23}--\textit{31}
\\
\textit{16} & Gods of\newline infinite luster & & Medium Degree & & 4\newline Great aeons & \textit{5}--\textit{22}, \textit{28}--\textit{31} & \textit{16}--\textit{21}, \textit{23}--\textit{31}
\\
\textit{15} & Gods of\newline minor luster & & Minor Degree & & 2\newline Great aeons & \textit{5}--\textit{22}, \textit{28}--\textit{31} & \textit{15}--\textit{21}, \textit{23}--\textit{31}
\\\cmidrule{3-8}
\textit{14} & Great Brahmās & \multirow{3}{0.13\textwidth}{\raisebox{-8mm}{\parbox{0.13\textwidth}{\centering 1\textsuperscript{st} Jhāna}}} & Highest Degree & \multirow{3}{0.1\textwidth}{\raisebox{-8mm}{\parbox{0.1\textwidth}{\centering \textbf{60}}}} & 1\newline Incalc. aeon & \textit{5}--\textit{22}, \textit{28}--\textit{31} & \textit{14}--\textit{21}, \textit{23}--\textit{31}
\\
\textit{13} & Ministers of Brahmā & & Medium Degree & & $\sfrac{1}{2}$ \newline Incalc. aeon & \textit{5}--\textit{22}, \textit{28}--\textit{31} & \textit{13}--\textit{21}, \textit{23}--\textit{31}
\\
\textit{12} & Retinue of Brahmā & & Minor Degree & & $\sfrac{1}{3}$ \newline Incalc. aeon & \textit{5}--\textit{22}, \textit{28}--\textit{31} & \textit{12}--\textit{21}, \textit{23}--\textit{31}
\\
\bottomrule
\end{tabular}

\caption{Realms \textit{12}--\textit{27}, the Fine Material Plane. Beings must attain a fine material jhāna in a previous life to be reborn here and beings in these realms spend most of their time experiencing a fine material jhāna.}
\label{fig:Fine1}
\end{figure}

The Fine Material Plane includes 16 realms: three related to the first jhāna, three related to the second or third jhāna, three related to the fourth jhāna and seven related to the fifth jhāna. The cause of rebirth into the Fine Material Plane is the attainment of a jhāna, Mind Moment \textbf{55}--\textbf{59}, in a previous life. After a very long lifespan, a non-saint is reborn from the Fine Material Plane into one of the Happy Destinations or higher. Once a saint has been reborn into the Fine Material Plane, they can only be reborn back into the Fine Material Plane or higher, until they become an Arahat.

In one Sutta, the Buddha explained how the \textbf{Wrong view} of a Creator God arises.\footnote{DN 1: \url{http://www.accesstoinsight.org/tipitaka/dn/dn.01.0.bodh.html\#paragraph-39}} The world goes through cycles of contracting and expanding, and that when the world starts to expand after the “big bang,” beings from another world-system are reborn into Realm \textit{14} as a Great \textit{Brahmā} in the newly formed world. After a while, the Great \textit{Brahmā} in Realm \textit{14} gets lonely and wishes for companionship. This is a condition for beings from other world-systems to be reborn into Realm \textit{12} and Realm \textit{13} to accompany the Great \textit{Brahmā} in Realm \textit{14}. 

When this happens, the Great \textit{Brahmā} is convinced that he is the All-seeing, All-powerful Creator God and the Supreme Being. He believes this because he arose first in this world and the other beings arose because he wanted them to arise. The beings in Realm \textit{12} and Realm \textit{13} are also convinced that the Great \textit{Brahmā} is the Creator God. These beings have shorter lifespans than the Great \textit{Brahmā}, and they are reborn in progressively lower realms: the \textit{Deva} realms, the Human realm and the Woeful States. As they progress to these lower realms, many retain this memory of the Great \textit{Brahmā} as being the All-seeing, All-powerful Creator God. The Buddha explained that this is why many humans believe in a Creator God.

In another Sutta, a monk with the ability to visit heavenly realms has a philosophical question.\footnote{DN 11: \url{http://www.accesstoinsight.org/tipitaka/dn/dn.11.0.than.html}} When he visits Realm \textit{6}, the \textit{Devas} say they do not know the answer and suggest that he ask the \textit{Devas} in Realm \textit{7}. The \textit{Devas} in Realm \textit{7} cannot answer and direct him to Realm \textit{8}. This happens repeatedly until the monk asks the Great \textit{Brahmā} in Realm \textit{14}. The Great \textit{Brahmā} says, “I am the All-seeing, All-powerful Creator.” The monk replies, “That is not what I asked” and repeats his question. Again, the Great \textit{Brahmā} avoids the question and says, “I am the All-seeing, All-powerful Creator.” A third time, the monk repeats his philosophical question. Finally, the Great \textit{Brahmā} takes the monk aside and says, “The beings in the lower realms believe there is nothing that I do not see, nothing that I do not know. That is why I answered you as I did. In fact, I do not know that answer to your question but I could not admit this in the presence of the other beings. The Buddha will know the answer to your question.”

World-systems are created and destroyed in a cyclical pattern over an extremely long time-frame called an “incalculable aeon.” The four phases can be described as “big bang,” “expanding universe,” “contracting universe” and “big crunch.” The lifespan of a Great \textit{Brahmā} matches with the lifespan of a world-system, one “incalculable aeon.” A Great \textit{Brahmā} comes into existence when a world-system arises and after an incalculable aeon, the world-system is destroyed up to and including Realm \textit{14} where the Great \textit{Brahmā} resides. According to the Commentary,\footnote{See Commentary to AN 4.156.} the realms up to Realm \textit{14} are destroyed seven times in a row by fire, and then realms up to Realm \textit{17} are destroyed by water. Once Realm \textit{17} has been destroyed by water seven times, the realms up to Realm \textit{20} are destroyed by wind. This cycle of 64 destructions of world-systems then repeats itself. The lifespans of beings in Realm \textit{15} and above are measured in “great aeons;” a “great aeon” is four times the duration of an “incalculable aeon.”

\begin{figure}[H]
\centering
\input{./Diagrams/Worlds.pdf_tex}
\caption{World-systems are destroyed up to Realm 14 by fire seven times in a row and then destroyed up to Realm 17 by water. This pattern repeats itself seven times and for the eighth repetition, the world-system is destroyed up to Realm 20 by wind and the cycle repeats.}
\label{fig:Worlds}
\end{figure}

Beings who attain the fifth jhāna together with a dispassion toward the mind can be reborn into Realm \textit{22}. For their entire lifespan here, the beings have no mind, only matter; they are like statues. When their lifespan in Realm \textit{22} expires, kamma from a previous rebirth causes them to be reborn into some other realm.

Only Anāgāmī are born into Realms \textit{23}--\textit{27}. Once reborn into these Pure Abodes, the Anāgāmī will continue to be reborn into the Pure Abodes until they become an Arahat. Anāgāmī who have attained the fifth jhāna will be reborn into Realms \textit{23}--\textit{27} according to their dominant faculty: \textbf{Faith}, \textbf{Energy}, \textbf{Mindfulness}, concentration or \textbf{Understanding}. Anāgāmī who have not attained the fifth jhāna can be reborn in any realm in the Fine Material Plane or the Immaterial Plane. Anāgāmī cannot be reborn into the Sensuous Plane as they have eradicated any \textbf{Attachment} to sense objects.

Figure \ref{fig:Fine} shows that the Mind Moments in the Fine Material Plane are similar to those arising in the Happy Destinations for 3-rooted non-saints and for saints. There are two exceptions; there is no \textbf{Aversion} in the Fine Material Plane, and beings in the Fine Material Plane lack the “coarse” senses of smelling, tasting and tactile sense. The most common kamma-creating Mind Moment for beings in the Fine Material Plane will be the one Fine Material Sphere jhāna corresponding to that realm (Mind Moment \textbf{55}--\textbf{59}).\footnote{For example, the most common kamma-creating Mind Moment for beings in Realms \textit{12}--\textit{14} will be \textbf{55}.}

\pagebreak

\begin{figure}[H]
\centering
\noindent\begin{tabular}{L{\dimexpr.4\textwidth-2\tabcolsep} R{\dimexpr.6\textwidth-2\tabcolsep}}

\setlength{\tabcolsep}{0pt}
\renewcommand{\arraystretch}{1.1}

\noindent\begin{tabular}{p{.15\textwidth}
R{.047\textwidth} |
p{.039\textwidth}
p{.039\textwidth}
p{.039\textwidth}
p{.039\textwidth}
p{.039\textwidth}}
\toprule
&
& \tablevsubheaderhack{Non-saint}
& \tablevsubheaderhack{Sotāpanna}
& \tablevsubheaderhack{Sakadāgāmī}
& \tablevsubheaderhack{Anāgāmī}
& \tablevsubheaderhack{Arahat}
\\
\midrule
\multirow{8}{.15\textwidth}{\tablesubheader{\textbf{Attachment}\newline -rooted}} & \hpadright{\textbf{1}} & \tm & & & &
\\
& \hpadright{\textbf{2}} & \tm & & & &
\\
& \hpadright{\textbf{3}} & \tm & \tm & \tm & \tm &
\\
& \hpadright{\textbf{4}} & \tm & \tm & \tm & \tm &
\\
& \hpadright{\textbf{5}} & \tm & & & &
\\
& \hpadright{\textbf{6}} & \tm & & & &
\\
& \hpadright{\textbf{7}} & \tm & \tm & \tm & \tm &
\\
& \hpadright{\textbf{8}} & \tm & \tm & \tm & \tm &
\\
\midrule
\multirow{2}{.15\textwidth}{\tablesubheader{\textbf{Aversion}\newline -rooted}} & \hpadright{\textbf{9}} & & & & &
\\
& \hpadright{\textbf{10}} & & & & &
\\
\midrule
\multirow{2}{.15\textwidth}{\tablesubheader{\textbf{Delusion}\newline -rooted}} & \hpadright{\textbf{11}} & \tm & & & &
\\
& \hpadright{\textbf{12}} & \tm & \tm & \tm & \tm &
\\
\midrule
\multirow{7}{.15\textwidth}{\tablesubheader{Past unwholesome\linebreak resultant}} & \hpadright{\textbf{13}} & \tm & \tm & \tm & \tm & \tm
\\
& \hpadright{\textbf{14}} & \tm & \tm & \tm & \tm & \tm
\\
& \hpadright{\textbf{15}} & \multicolumn{5}{C{.195\textwidth}}{\multirow{3}{.195\textwidth}{\centering No nose, tongue\newline or body}}
\\
& \hpadright{\textbf{16}} & & & & &
\\
& \hpadright{\textbf{17}} & & & & &
\\
& \hpadright{\textbf{18}} & \tm & \tm & \tm & \tm & \tm
\\
& \hpadright{\textbf{19}} & \tm & \tm & \tm & \tm & \tm
\\
\midrule
\multirow{8}{.15\textwidth}{\tablesubheader{Past wholesome\linebreak resultant}} & \hpadright{\textbf{20}} & \tm & \tm & \tm & \tm & \tm
\\
& \hpadright{\textbf{21}} & \tm & \tm & \tm & \tm & \tm
\\
& \hpadright{\textbf{22}} & \multicolumn{5}{C{.195\textwidth}}{\multirow{3}{.195\textwidth}{\centering No nose, tongue\newline or body}}
\\
& \hpadright{\textbf{23}} & & & & &
\\
& \hpadright{\textbf{24}} & & & & &
\\
& \hpadright{\textbf{25}} & \tm & \tm & \tm & \tm & \tm
\\
& \hpadright{\textbf{26}} & \tm & \tm & \tm & \tm & \tm
\\
& \hpadright{\textbf{27}} & \tm & \tm & \tm & \tm & \tm
\\
\midrule
\multirow{3}{.15\textwidth}{\tablesubheader{Functional}} & \hpadright{\textbf{28}} & \tm & \tm & \tm & \tm & \tm
\\
& \hpadright{\textbf{29}} & \tm & \tm & \tm & \tm & \tm
\\
& \hpadright{\textbf{30}} & & & & & \tm
\\
\bottomrule
\end{tabular}

&

\setlength{\tabcolsep}{0pt}
\renewcommand{\arraystretch}{1.1}
\vspace{30mm}
\noindent\begin{tabular}{p{.05\textwidth} p{.05\textwidth}
p{0.16\textwidth}
R{.077\textwidth} |
p{.039\textwidth}
p{.039\textwidth}
p{.039\textwidth}
p{.039\textwidth}
p{.039\textwidth}}
\toprule
& &
&
& \tablevsubheaderhack{Non-saint}
& \tablevsubheaderhack{Sotāpanna}
& \tablevsubheaderhack{Sakadāgāmī}
& \tablevsubheaderhack{Anāgāmī}
& \tablevsubheaderhack{Arahat}
\\
\midrule
\multicolumn{2}{C{0.1\textwidth}}{\multirow{10}{.1\textwidth}{\vspace{-8mm}\rotatebox[origin=l]{90}{\tableheader{Sense Sphere}}}} & \tablesubheader{Wholesome} & \hpadright{\textbf{31}--\textbf{38}} & \tm & \tm & \tm & \tm &
\\
\cmidrule{3-9}
& & \multirow{8}{.26\textwidth}{\tablesubheader{Resultant}} & \hpadright{\textbf{39}} & & & & &
\\
& & & \hpadright{\textbf{40}} & & & & &
\\
& & & \hpadright{\textbf{41}} & & & & &
\\
& & & \hpadright{\textbf{42}} & & & & &
\\
& & & \hpadright{\textbf{43}} & & & & &
\\
& & & \hpadright{\textbf{44}} & & & & &
\\
& & & \hpadright{\textbf{45}} & & & & &
\\
& & & \hpadright{\textbf{46}} & & & & &
\\
\cmidrule{3-9}
& & \tablesubheader{Functional} & \hpadright{\textbf{47}--\textbf{54}} & & & & & \tm
\\
\midrule
\multicolumn{2}{C{0.1\textwidth}}{\multirow{3}{0.1\textwidth}{\rotatebox[]{90}{\parbox{17mm}{\centering \tableheader{Fine\\ Material\\ Sphere}}}}} & \tablesubheader{Wholesome} & \hpadright{\textbf{55}--\textbf{59}} & \tmcommon & \tmcommon & \tmcommon & \tmcommon &
\\
& & \tablesubheader{Resultant} & \hpadright{\textbf{60}--\textbf{64}} & \tm & \tm & \tm & \tm & \tm
\\
& & \tablesubheader{Functional} & \hpadright{\textbf{65}--\textbf{69}} & & & & & \tm
\\\midrule
\multicolumn{2}{C{0.1\textwidth}}{\multirow{3}{0.1\textwidth}{\rotatebox[]{90}{\parbox{17mm}{\centering \tableheader{Im-\\ material\\ Sphere}}}}} & \tablesubheader{Wholesome} & \hpadright{\textbf{70}--\textbf{73}} & \tm & \tm & \tm & \tm &
\\
& & \tablesubheader{Resultant} & \hpadright{\textbf{74}--\textbf{77}} & & & & &
\\
& & \tablesubheader{Functional} & \hpadright{\textbf{78}--\textbf{81}} & & & & & \tm
\\\midrule
\multirow{8}{.05\textwidth}{\rotatebox[]{90}{\tableheader{Supramundane}}} & \multirow{4}{.05\textwidth}{\rotatebox[]{90}{\tablesubheader{Path}}} & \tablesubheader{Sotāpanna} & \hpadright{\textbf{82}} & \tm & & & &
\\
& & \tablesubheader{Sakadāgāmī} & \hpadright{\textbf{83}} & & \tm & & &
\\
& & \tablesubheader{Anāgāmī} & \hpadright{\textbf{84}} & & & \tm & &
\\
& & \tablesubheader{Arahat} & \hpadright{\textbf{85}} & & & & \tm &
\\\cmidrule{2-9}
& \multirow{4}{.05\textwidth}{\rotatebox[]{90}{\tablesubheader{Fruit}}} & \tablesubheader{Sotāpanna} & \hpadright{\textbf{86}} & & \tm & & &
\\
& & \tablesubheader{Sakadāgāmī} & \hpadright{\textbf{87}} & & & \tm & &
\\
& & \tablesubheader{Anāgāmī} & \hpadright{\textbf{88}} & & & & \tm &
\\
& & \tablesubheader{Arahat} & \hpadright{\textbf{89}} & & & & & \tm
\\
\bottomrule
\end{tabular}
\\
\end{tabular}
\begin{center}
\tmcommon\hspace{2mm} Common kamma-creating Mind Moment \hspace{5mm} \tm\hspace{2mm} Mind Moment that can arise
\end{center}
\caption{Mind Moments in the Fine Material Plane; the one common kamma-creating Mind Moment is \textbf{55}, \textbf{56}, \textbf{57}, \textbf{58} or \textbf{59}, corresponding to that realm.}
\label{fig:Fine}
\end{figure}

\subsection*{Immaterial Plane}

\begin{figure}[H]
\centering

\setlength{\tabcolsep}{0pt}
\renewcommand{\arraystretch}{1.1}

\noindent\begin{tabular}{L{0.05\textwidth} C{0.18\textwidth} C{0.13\textwidth} C{0.21\textwidth} C{0.1\textwidth} C{0.17\textwidth} C{0.08\textwidth} C{0.08\textwidth}}
\toprule
 & \multirow{2}{0.18\textwidth}{\raisebox{-4mm}{\parbox{0.18\textwidth}{\centering \tableheader{Name}}}}
 & \multirow{2}{0.13\textwidth}{\raisebox{-4mm}{\parbox{0.13\textwidth}{\centering \tableheader{Jhāna}}}}
 & \multirow{2}{0.21\textwidth}{\raisebox{-5mm}{\parbox{0.21\textwidth}{\centering \tableheader{Cause of rebirth into this Realm}}}}
 & \multirow{2}{0.1\textwidth}{\raisebox{-5mm}{\parbox{0.1\textwidth}{\centering \tableheader{Life-cont.}}}}
 & \multirow{2}{0.17\textwidth}{\raisebox{-4mm}{\parbox{0.17\textwidth}{\centering \tableheader{Lifespan}}}}
 & \multicolumn{2}{c}{\centering \tableheader{Destination}}
 \\
 & & & & & & \tablesubheader{Non-saints} & \tablesubheader{Saints}
 \\
\midrule
\textit{31} & Neither perception nor non-perception & \multirow{4}{0.13\textwidth}[-10mm]{\centering Immaterial Jhāna} & Mind Moment \textbf{73} & \textbf{77} & 84,000\newline Great aeons & 5--11, 31 & 31
\\
\textit{30} & Nothingness & & Mind Moment \textbf{72} & \textbf{76} & 60,000\newline Great aeons & 5--11, 30, 31 & 30, 31
\\
\textit{29} & Infinite consciousness & & Mind Moment \textbf{71} & \textbf{75} & 40,000\newline Great aeons & 5--11, 29-31 & 29--31
\\
\textit{28} & Infinite space & & Mind Moment \textbf{70} & \textbf{74} & 20,000\newline Great aeons & 5--11, 28-31 & 28--31
\\
\bottomrule
\end{tabular}

\caption{Realms \textit{28}--\textit{31}, the Immaterial Plane. Beings must attain an immaterial jhāna in a previous life to be reborn here and beings in these realms spend most of their time experiencing an immaterial jhāna.}
\label{fig:Immaterial1}
\end{figure}

The Immaterial Plane includes four realms, named after the four formless jhānas. The cause of rebirth into the Immaterial Plane is the attainment of the corresponding jhāna, Mind Moment \textbf{70}--\textbf{73}, in a previous life.\footnote{For example, attaining Mind Moment \textbf{70} in a previous life is required to be reborn into Realm \textit{28}.} After an exceptionally long lifespan, a non-saint is reborn from the Immaterial Plane into one of the Happy Destinations or back into the Immaterial Plane. If a saint is reborn into the Immaterial Plane, he continues to be reborn into the Immaterial Plane until he becomes an Arahat.

Figure \ref{fig:Immaterial} shows that the Mind Moments in the Immaterial Plane are similar to those arising in the Fine Material Plane. One exception is that none of the Mind Moments associated with sensing can arise in the Immaterial Plane, as beings in the Immaterial Plane are pure mind with no body and no senses. Without senses, beings in the Immaterial Plane are unable to see the Buddha or hear the Dhamma. To explain how mind exists without a supporting body, the Commentary uses the analogy of a bar flung into the air. For a certain period, the bar remains in the air without support. The one most common kamma-creating Mind Moment for beings in the Immaterial Plane will be the Immaterial Sphere jhāna corresponding to that realm (Mind Moment \textbf{70}--\textbf{73}).\footnote{For example, the most common kamma-creating Mind Moment for beings in Realm \textit{28} will be \textbf{70}.}

\begin{figure}[H]
\centering
\noindent\begin{tabular}{L{\dimexpr.4\textwidth-2\tabcolsep} R{\dimexpr.6\textwidth-2\tabcolsep}}

\setlength{\tabcolsep}{0pt}
\renewcommand{\arraystretch}{1.1}

\noindent\begin{tabular}{p{.15\textwidth}
R{.047\textwidth} |
p{.039\textwidth}
p{.039\textwidth}
p{.039\textwidth}
p{.039\textwidth}
p{.039\textwidth}}
\toprule
&
& \tablevsubheaderhack{Non-saint}
& \tablevsubheaderhack{Sotāpanna}
& \tablevsubheaderhack{Sakadāgāmī}
& \tablevsubheaderhack{Anāgāmī}
& \tablevsubheaderhack{Arahat}
\\
\midrule
\multirow{8}{.15\textwidth}{\tablesubheader{\textbf{Attachment}\newline -rooted}} & \hpadright{\textbf{1}} & \tm & & & &
\\
& \hpadright{\textbf{2}} & \tm & & & &
\\
& \hpadright{\textbf{3}} & \tm & \tm & \tm & \tm &
\\
& \hpadright{\textbf{4}} & \tm & \tm & \tm & \tm &
\\
& \hpadright{\textbf{5}} & \tm & & & &
\\
& \hpadright{\textbf{6}} & \tm & & & &
\\
& \hpadright{\textbf{7}} & \tm & \tm & \tm & \tm &
\\
& \hpadright{\textbf{8}} & \tm & \tm & \tm & \tm &
\\
\midrule
\multirow{2}{.15\textwidth}{\tablesubheader{\textbf{Aversion}\newline -rooted}} & \hpadright{\textbf{9}} & & & & &
\\
& \hpadright{\textbf{10}} & & & & &
\\
\midrule
\multirow{2}{.15\textwidth}{\tablesubheader{\textbf{Delusion}\newline -rooted}} & \hpadright{\textbf{11}} & \tm & & & &
\\
& \hpadright{\textbf{12}} & \tm & \tm & \tm & \tm &
\\
\midrule
\multirow{7}{.15\textwidth}{\tablesubheader{Past unwholesome\linebreak resultant}} & \hpadright{\textbf{13}} & \multicolumn{5}{C{.195\textwidth}}{\multirow{7}{.195\textwidth}{\centering Only mind}}
\\
& \hpadright{\textbf{14}} & & & & &
\\
& \hpadright{\textbf{15}} & & & & &
\\
& \hpadright{\textbf{16}} & & & & &
\\
& \hpadright{\textbf{17}} & & & & &
\\
& \hpadright{\textbf{18}} & & & & &
\\
& \hpadright{\textbf{19}} & & & & &
\\
\midrule
\multirow{8}{.15\textwidth}{\tablesubheader{Past wholesome\linebreak resultant}} & \hpadright{\textbf{20}} & \multicolumn{5}{C{.195\textwidth}}{\multirow{8}{.195\textwidth}{\centering Only mind}}
\\
& \hpadright{\textbf{21}} & & & & &
\\
& \hpadright{\textbf{22}} & & & & &
\\
& \hpadright{\textbf{23}} & & & & &
\\
& \hpadright{\textbf{24}} & & & & &
\\
& \hpadright{\textbf{25}} & & & & &
\\
& \hpadright{\textbf{26}} & & & & &
\\
& \hpadright{\textbf{27}} & & & & &
\\
\midrule
\multirow{3}{.15\textwidth}{\tablesubheader{Functional}} & \hpadright{\textbf{28}} & & & &
\\
& \hpadright{\textbf{29}} & \tm & \tm & \tm & \tm & \tm
\\
& \hpadright{\textbf{30}} & & & & &
\\
\bottomrule
\end{tabular}

&

\setlength{\tabcolsep}{0pt}
\renewcommand{\arraystretch}{1.1}
\vspace{30mm}
\noindent\begin{tabular}{p{.05\textwidth} p{.05\textwidth}
p{0.16\textwidth}
R{.077\textwidth} |
p{.039\textwidth}
p{.039\textwidth}
p{.039\textwidth}
p{.039\textwidth}
p{.039\textwidth}}
\toprule
& &
&
& \tablevsubheaderhack{Non-saint}
& \tablevsubheaderhack{Sotāpanna}
& \tablevsubheaderhack{Sakadāgāmī}
& \tablevsubheaderhack{Anāgāmī}
& \tablevsubheaderhack{Arahat}
\\
\midrule
\multicolumn{2}{C{0.1\textwidth}}{\multirow{10}{.1\textwidth}{\vspace{-8mm}\rotatebox[origin=l]{90}{\tableheader{Sense Sphere}}}} & \tablesubheader{Wholesome} & \hpadright{\textbf{31}--\textbf{38}} & \tm & \tm & \tm & \tm &
\\
\cmidrule{3-9}
& & \multirow{8}{.26\textwidth}{\tablesubheader{Resultant}} & \hpadright{\textbf{39}} & & & & &
\\
& & & \hpadright{\textbf{40}} & & & & &
\\
& & & \hpadright{\textbf{41}} & & & & &
\\
& & & \hpadright{\textbf{42}} & & & & &
\\
& & & \hpadright{\textbf{43}} & & & & &
\\
& & & \hpadright{\textbf{44}} & & & & &
\\
& & & \hpadright{\textbf{45}} & & & & &
\\
& & & \hpadright{\textbf{46}} & & & & &
\\
\cmidrule{3-9}
& & \tablesubheader{Functional} & \hpadright{\textbf{47}--\textbf{54}} & & & & & \tm
\\
\midrule
\multicolumn{2}{C{0.1\textwidth}}{\multirow{3}{0.1\textwidth}{\rotatebox[]{90}{\parbox{17mm}{\centering \tableheader{Fine\\ Material\\ Sphere}}}}} & \tablesubheader{Wholesome} & \hpadright{\textbf{55}--\textbf{59}} & & & & &
\\
& & \tablesubheader{Resultant} & \hpadright{\textbf{60}--\textbf{64}} & & & & &
\\
& & \tablesubheader{Functional} & \hpadright{\textbf{65}--\textbf{69}} & & & & &
\\\midrule
\multicolumn{2}{C{0.1\textwidth}}{\multirow{3}{0.1\textwidth}{\rotatebox[]{90}{\parbox{17mm}{\centering \tableheader{Im-\\ material\\ Sphere}}}}} & \tablesubheader{Wholesome} & \hpadright{\textbf{70}--\textbf{73}} & \tmcommon & \tmcommon & \tmcommon & \tmcommon &
\\
& & \tablesubheader{Resultant} & \hpadright{\textbf{74}--\textbf{77}} & \tm & \tm & \tm & \tm & \tm
\\
& & \tablesubheader{Functional} & \hpadright{\textbf{78}--\textbf{81}} & & & & & \tm
\\\midrule
\multirow{8}{.05\textwidth}{\rotatebox[]{90}{\tableheader{Supramundane}}} & \multirow{4}{.05\textwidth}{\rotatebox[]{90}{\tablesubheader{Path}}} & \tablesubheader{Sotāpanna} & \hpadright{\textbf{82}} & \tm & & & &
\\
& & \tablesubheader{Sakadāgāmī} & \hpadright{\textbf{83}} & & \tm & & &
\\
& & \tablesubheader{Anāgāmī} & \hpadright{\textbf{84}} & & & \tm & &
\\
& & \tablesubheader{Arahat} & \hpadright{\textbf{85}} & & & & \tm &
\\\cmidrule{2-9}
& \multirow{4}{.05\textwidth}{\rotatebox[]{90}{\tablesubheader{Fruit}}} & \tablesubheader{Sotāpanna} & \hpadright{\textbf{86}} & & \tm & & &
\\
& & \tablesubheader{Sakadāgāmī} & \hpadright{\textbf{87}} & & & \tm & &
\\
& & \tablesubheader{Anāgāmī} & \hpadright{\textbf{88}} & & & & \tm &
\\
& & \tablesubheader{Arahat} & \hpadright{\textbf{89}} & & & & & \tm
\\
\bottomrule
\end{tabular}
\\
\end{tabular}
\begin{center}
\tmcommon\hspace{2mm} Common kamma-creating Mind Moment \hspace{5mm} \tm\hspace{2mm} Mind Moment that can arise
\end{center}
\caption{Mind Moments in the Immaterial Plane; the one common kamma-creating Mind Moment is \textbf{70}, \textbf{71}, \textbf{72} or \textbf{73}, corresponding to that realm.}
\label{fig:Immaterial}
\end{figure}

\pagebreak

\subsection*{Summary of Key Points}

\begin{itemize}

\item Belief in rebirth in not negotiable, but in my opinion, belief in details regarding Realms of Existence is optional because these details do not impact doctrine or practice.

\item The Woeful States include four realms: Hell, Animal, \textit{Peta} and \textit{Asura}.

\begin{itemize}

\item Unwholesome kamma can cause rebirth into the Woeful States.

\item Saints are never reborn in Woeful States, and beings in Woeful States cannot become saints.

\item Beings in Woeful States cannot attain jhāna.

\end{itemize}

\item Human rebirth (Realm \textit{5}) is the lowest among the Happy Destinations.

\begin{itemize}

\item Human realm is the ideal place for spiritual development.

\item Past wholesome rebirth-linking kamma can have either 3 or 2 roots (with or without \textbf{Understanding}) and can be either “superior” or “inferior” (supported or unsupported before and after).

\begin{itemize}

\item Past 3-rooted superior rebirth-linking kamma produces present 3-rooted life-continuum in realms \textit{5}--\textit{11} (can achieve sainthood or jhāna).

\item Past 3-rooted inferior rebirth-linking kamma or past 2-rooted inferior rebirth-linking kamma produces present 2-rooted life-continuum in realms \textit{5} or \textit{6} (cannot achieve sainthood or jhāna).

\item Past 2-rooted inferior rebirth-linking kamma produces present rootless life-continuum in human realm (congenitally disabled).

\end{itemize}

\end{itemize}

\item The \textit{Deva} realms (Realms \textit{6}--\textit{11}) are also Happy Destinations.

\begin{itemize}

\item \textit{Deva} realms include the earth-bound \textit{Devas}, Sakka (king of \textit{Devas}) and Māra (personification of temptation who disappears when recognized).

\end{itemize}

\item The \textit{Brahmā} realms (Realms \textit{12}--\textit{31}) are the result of jhāna meditation.

\begin{itemize}

\item \textit{Brahmā} realms include “the Great Brahmā” (Creator God whose lifespan is the duration of a universe) and the Pure Abodes.

\item Beings in \textit{Brahmā} realms spend most of their time in jhāna.

\item Beings in realm \textit{12}--\textit{27} have no nose, tongue or body (eyes and ears only), beings in realm \textit{28}--\textit{31} are mind-only.

\end{itemize}

\end{itemize}

Finally, in my opinion, the most important thing to remember about Realms of Existence is that supernatural beings are like spices in a meal. They add flavour, but are not essential to the nutritional value of the meal. Belief in supernatural beings is not a requirement for following the Buddha’s path.

\newpage

\subsection*{Questions \& Answers}

\question{Is there a correlation between our present Mind Moment and Realms of Existence?}

Yes, there is a clear correlation. A mind that is consumed by anger is burning, painful and unpleasant, as if the mind is in the Hell realm. A mind that is consumed by \textbf{Delusion} is working on instinct without \textbf{Understanding}, as if the mind is in the animal realm. A mind that is consumed by \textbf{Attachment} suffers from insatiable hunger, like the mind of hungry ghosts. A mind that is quarrelsome and argumentative is dark, like the mind of an Asura. The mind that is wholesome shines brightly, as the \textit{Devas} shine. The meditating mind is deeply absorbed and stable, like \textit{Brahmas}.

\question{Do the heaven and Hell realms really exist or are they metaphorical? Is belief in these realms important to spiritual development?}

I do not have a strong opinion as to whether the heaven and Hell realms really exist or are metaphorical. I consider many of the details, especially those found in the Commentaries, to be legendary. In my opinion, belief in rebirth does not require belief in heaven and Hell realms.

Shortly before his \textit{parinibbāna}, the Buddha was asked if there were saints in any other religious tradition.\footnote{DN 16: \url{http://www.accesstoinsight.org/tipitaka/dn/dn.16.1-6.vaji.html\#para-5-60}} The Buddha replied that any religious tradition that included the Noble Eightfold Path could have saints. To me, this means that spiritual development involves the Noble Eightfold Path. The Noble Eightfold Path does not include any supernatural elements and does not depend on belief in heaven and Hell realms.

\question{Does performing the ten wholesome actions and observing the five precepts guarantee that one will not be reborn into woeful states?}

Performing the ten wholesome actions and observing the five precepts does not provide a guarantee, but does increase one’s chances of a favourable rebirth. Even if performing the ten wholesome actions and observing the five precepts does not condition favourable rebirth in the next life, the wholesome kamma generated will have a positive impact in this life, in the next life and in future lives. We will discuss this in more detail in the next lesson, which will cover kamma.

\question{Why is Māra found in a \textit{Deva} realm?}

In my opinion, Māra was placed in the topmost \textit{Deva} realm to reflect the immense power that the defilements can have (Māra being a personification of the defilements). Though they were born in a \textit{Deva} realm due to past kamma, \textit{Devas} have many of the same weaknesses as humans. \textit{Brahmas}, \textit{Devas} and humans all turn to the Buddha for advice on spiritual development; this is why one of the descriptions of the Buddha is ``teacher of gods and men" (\textit{satthā deva-manussānaṃ}). 

\pagebreak

\question{How do we know if our life-continuum Mind Moment has two roots or three roots? If our life-continuum Mind Moment has two roots, should we bother meditating?}

In my opinion, a person whose life-continuum Mind Moment has two roots may not be spiritually inclined at all (we all know people like this). Since you have taken an interest in the Buddha's teaching, this suggests to me that your Mind Moment has three roots; it includes the root of \textbf{Understanding}.

According to the Abhidhamma, if your life-continuum Mind Moment has two roots, then you will not be able to attain jhāna or achieve sainthood in this lifetime. This does not mean that meditation would not be extremely valuable. There are many benefits to meditation that do not involve jhāna or sainthood. Meditation in this lifetime is laying a strong foundation for this life and for future lives.

\question{In what manner are beings reborn in other realms?}

The Buddha mentions that beings can be born from an egg, from a womb, from moisture or spontaneously.\footnote{MN 12: \url{http://www.accesstoinsight.org/tipitaka/mn/mn.012.ntbb.html}} Animals can be born from an egg, from a womb or from moisture. Beings born into realms other than the animal realm and human realm are born spontaneously.

