\documentclass[a4 paper, 12pt]{article}
\usepackage[margin=10mm]{geometry}
\usepackage{textcomp}
\usepackage[font={small,it}]{caption}
\usepackage{endnotes}

\makeatletter
\def\enoteheading{\section*{\@mkboth{\MakeUppercase{\notesname}}{\MakeUppercase{\notesname}}}%
  \mbox{}\par\vskip-0\baselineskip\noindent\rule{.5\textwidth}{0.4pt}\par\vskip-0\baselineskip}
\makeatother
\usepackage{enumitem}
\usepackage{float}
\usepackage{fontspec}
\usepackage{titlesec}
\newfontfamily{\sectionfont}[Ligatures=TeX]{Arial}
\titleformat*{\section}{\sectionfont\LARGE}
\titleformat*{\subsection}{\sectionfont\Large}
\titleformat*{\subsubsection}{\sectionfont\large}
\setmainfont[Ligatures=TeX]{Times New Roman}
\makeatletter
\renewcommand\@makefntext[1]{%
\noindent\makebox[0em][r]{\@makefnmark}#1}
\makeatother
\usepackage{xfrac}
\usepackage{xcolor}
\usepackage[hidelinks]{hyperref}
\hypersetup{pdfauthor={rob.abhidhamma@gmail.com}, pdfsubject={Abhidhamma}, pdftitle={Practical Abhidhamma Course}}
\hypersetup{colorlinks=true, linkcolor=black, urlcolor=blue}
\usepackage{multicol}
\newcommand{\hpadright}[1]{#1 \hspace{2mm}}
\newcommand{\question}[1]{\subsubsection*{#1}}
\usepackage{wasysym}
\newcommand{\tmcommon}{\hspace{2.5mm}\CIRCLE}
\newcommand{\tm}{\hspace{2.5mm}\Circle}
\usepackage{booktabs}
\usepackage{multicol}
\usepackage{multirow}
\usepackage{array}
\newcolumntype{L}[1]{>{\raggedright\let\newline\\\arraybackslash\hspace{0pt}}m{#1}}
\newcolumntype{C}[1]{>{\centering\let\newline\\\arraybackslash\hspace{0pt}}m{#1}}
\newcolumntype{R}[1]{>{\raggedleft\let\newline\\\arraybackslash\hspace{0pt}}m{#1}}
\newcommand{\tableheader}[1]{{\sectionfont \textbf{#1}}}
\newcommand{\tablesubheader}[1]{{\sectionfont #1}}
\newcommand{\tablevsubheader}[1]{\hspace{2.5mm}\rotatebox[origin=l]{90}{\tablesubheader{#1}\hspace{1mm}}}
\newcommand{\tablevsubheaderhack}[1]{\hspace{2mm}\rotatebox[origin=l]{90}{\tablesubheader{#1}\hspace{1mm}}}

\begin{document}
\pagestyle{empty}

\section*{Handout 7. Kamma and Natural Decisive Support}

\subsection*{Penetrative Sutta (AN 6.63)}

It is volition (\textit{cetanā}) that I call kamma. For having willed, one acts by body, speech or mind.

\subsection*{Shorter Exposition of Kamma Sutta (MN 135)}

Subha: “One meets with short-lived and long-lived people, sick and healthy people, ugly and beautiful people, insignificant and influential people, poor and rich people, low-born and high-born people, stupid and wise people. What is the reason, what is the condition, why superiority and inferiority are met with among human beings?”

Buddha: “Beings are owners of their kamma, heirs of their kamma; they originate from their kamma, are bound to their kamma and have their kamma as their refuge.”
A person who kills is reborn in an unhappy destination, or if reborn as human they have a short life. A person who refrains from killing is reborn as a god, or if reborn as a human they have a long life.

… injures others \textrightarrow \hspace{1mm} sickly

… angry \textrightarrow \hspace{1mm} ugly

… envious \textrightarrow \hspace{1mm} insignificant

… stingy \textrightarrow \hspace{1mm} poor

… arrogant \textrightarrow \hspace{1mm} low-born

… does not reflect on spiritual matters \textrightarrow \hspace{1mm} stupid

\subsection*{Kamma Results Sutta (AN 8.40)}

A person who steals is reborn in an unhappy destination, or if reborn as human they will tend to lose wealth.

… commits sexual misconduct \textrightarrow \hspace{1mm} tend to face ill-will and rivalry

… lies \textrightarrow \hspace{1mm} tend to face false accusations

… slanders \textrightarrow \hspace{1mm} tend to be divided from their friends

… uses harsh speech \textrightarrow \hspace{1mm} tend to hear disagreeable sounds

… indulges in frivolous talk \textrightarrow \hspace{1mm} tend to have others distrusting one’s words

… indulges in intoxicants \textrightarrow \hspace{1mm} tend to mental derangement

\subsection*{Salt Crystal Sutta (AN 3.99)}

There is the case where a trifling evil deed done by a certain individual takes him to hell. There is the case where the very same sort of trifling deed done by another individual is experienced in the here-and-now, and for the most part barely appears for a moment.

There is an individual who is undeveloped in contemplating the body, undeveloped in virtue, undeveloped in mind, undeveloped in discernment: restricted, small-hearted, dwelling with suffering. A trifling evil deed done by this sort of individual takes him to hell.

There is an individual who is developed in contemplating the body, developed in virtue, developed in mind, developed in discernment: unrestricted, large-hearted, dwelling with the immeasurable. A trifling evil deed done by this sort of individual is experienced in the here-and-now, and for the most part barely appears for a moment. 

A salt crystal in a small cup makes the water undrinkable and a salt crystal in the River Ganges has no effect. In the same way a trifling evil deed takes one individual to hell and the same sort of trifling deed done by the other individual is experienced in the here-and-now, and for the most part barely appears for a moment.

\newpage

\subsection*{Angulimāla Sutta (MN 86)}

Then Venerable Angulimāla, early in the morning, having put on his robes and carrying his outer robe and bowl, went into Sāvatthī for alms. Now at that time a clod thrown by one person hit Venerable Angulimāla on the body, a stone thrown by another person hit him on the body, and a potsherd thrown by still another person hit him on the body. So Venerable Angulimāla – his head broken open and dripping with blood, his bowl broken, and his outer robe ripped to shreds – went to the Blessed One. The Blessed One saw him coming from afar and on seeing him said to him: “Bear with it, brahmin! Bear with it! The fruit of the kamma that would have burned you in hell for many years, many hundreds of years, many thousands of years, you are now experiencing in the here-and-now!”

\subsection*{Completed Kamma}

\subsubsection*{Killing}
\ding{172} A living being \ding{173} Knowledge \ding{174} Intention \ding{175} Effort \ding{176} ⑤ ☺ Death

\subsubsection*{Stealing}
\ding{172} Property \ding{173} Knowledge \ding{174} Intention \ding{175} Effort \ding{176} Removal

\subsubsection*{Sexual Misconduct}

\ding{172} Forbidden partner (married / under guardianship) \ding{173} Intention \ding{174} Effort \ding{175} Acceptance

\subsubsection*{Lying}

\ding{172} Untrue thing \ding{173} Intention \ding{174} Effort \ding{175} Communication

\subsubsection*{Slander}

\ding{172} People to be separated \ding{173} Intention \ding{174} Effort \ding{175} Separation

\subsubsection*{Harsh Speech}

\ding{172} Person \ding{173} Intention \ding{174} Effort

\subsubsection*{Frivolous Talk}

\ding{172} Inclination toward useless talk \ding{173} Effort \ding{174} Others accept 

\subsubsection*{Covetousness}

\ding{172} Property \ding{173} Thought of “I wish it were mine”

\subsubsection*{Ill Will}

\ding{172} Another person \ding{173} Thought of “I hope that the other person is destroyed”

\subsubsection*{Wrong view}

\ding{172} Ideas of “no kamma”/“no cause”/“no cause and no result” \ding{173} Manifestation

\newpage

\subsection*{Classifications of Kamma in the Commentary}

\subsubsection*{By way of function}
\ding{172} Productive kamma \ding{173} Supportive kamma \ding{174} Obstructive kamma \ding{175} Destructive kamma

\subsubsection*{By time of ripening}

\ding{172} Immediately effective kamma \ding{173} Subsequently effective kamma \ding{174} Indefinitely effective kamma \ding{175} Defunct kamma

\subsubsection*{By order of ripening}

\ding{172} Weighty kamma \ding{173} Death proximate kamma \ding{174} Habitual kamma \ding{175} Reserve kamma

\subsubsection*{By place of ripening}

\ding{172} Unwholesome kamma \ding{173} Sense sphere wholesome kamma \ding{174} Fine material sphere kamma \ding{175} Immaterial sphere kamma

\subsection*{Causes and Results of Kamma}

\begin{tabular*}{\textwidth}{L{\dimexpr0.4\textwidth-2\tabcolsep} | L{\dimexpr0.6\textwidth-2\tabcolsep}}
\toprule
\tableheader{Kamma Causes} & \tableheader{Kamma Results} \\
\midrule

Volition as part of past unwholesome Mind Moments (\textbf{1}--\textbf{12})

\vspace{2mm}

Volition as part of past wholesome Mind Moments (\textbf{31}--\textbf{38}, \textbf{55}--\textbf{59}, \textbf{70}--\textbf{73} and \textbf{82}--\textbf{85})
&
Plane of rebirth (hell, animal, human, deva, etc.)

\vspace{2mm}

Mind Moments that process sense objects (\textbf{13}--\textbf{27})

\vspace{2mm}

Fruit Mind Moments (\textbf{86}--\textbf{89})

\vspace{2mm}

\textbf{Vital-nonad}, \textbf{Heart-base}, Gender
\vspace{2mm}

Five sense-sensitivity groups (eye, ear, nose, tongue, body)
  \\
  
\bottomrule
\end{tabular*}


\subsection*{Causes and Results of Natural Decisive Support}

\begin{tabular*}{\textwidth}{L{\dimexpr0.4\textwidth-2\tabcolsep} | L{\dimexpr0.6\textwidth-2\tabcolsep}}
\toprule
\tableheader{Natural Decisive Support Causes} & \tableheader{Natural Decisive Support Results} \\
\midrule

Past “strong” experiences:
\begin{itemize}
\item Mind Moments
\item Rūpa
\item Concepts
\end{itemize}

&
All current Mind Moments:
\begin{itemize}
\item Life-continuum Mind Moments
\item Mind Moments that process sense objects (\textbf{13}--\textbf{27})
\item Mind Moment that decides reaction (\textbf{29})
\item Mind Moments that create new kamma, including strength of volition (weightiness of kamma)\vspace*{-\baselineskip}
\end{itemize}\\
\bottomrule
\end{tabular*}

\subsection*{Implications of Natural Decisive Support}

Natural Decisive Support is how the current Mind Moment is impacted by accumulations, habits, defilements, \textit{pārami}, vows, tendencies, the environment, your mood or recent events. Don’t try to control the mind, train the mind by creating “strong” experiences through:

\begin{itemize}

\item Repeated wholesome volition (dāna, precepts, meditation, studying the Dhamma, wise attention, etc.)

\item Associating with good friends in the dhamma (\textit{kalyāṇa-mitta})

\item Suitable environment, suitable food

\end{itemize}


\end{document}
